\section{Related Work}\label{sec:related}

\subsection{Smart Contract Indexing and Analysis}
Several research efforts have focused on extracting and indexing blockchain data to improve accessibility. Third and Domingue~\cite{b4} proposed using Linked Data principles to create a semantic index for Ethereum, mapping blockchain entities to an ontology. The Graph Protocol~\cite{b5} offers a decentralized indexing solution where ``subgraphs'' can be defined to query specific on-chain data via a GraphQL API. While powerful for querying transactional data, these systems do not focus on the semantic content of the contract code itself.

For code analysis, tools have emerged to make sense of compiled bytecode. The STAN system~\cite{b6} aimed to describe the functionality of closed-source contracts by analyzing their bytecode. Similarly, the eth2dgraph tool~\cite{b7}, which we utilize in our work, can decompile bytecode to extract a contract's Application Binary Interface (ABI) using heimdall-rs. However, these approaches are limited by the information loss that occurs during compilation and do not capture the high-level design intent available in source code. Furthermore, there is no present research that has explored utilizing LLM's capability to analyze source code into analyzing smart contracts' source code.

\subsection{Semantic Code Search}
The application of Natural Language Processing (NLP) to code search has gained traction. Zhang~\cite{b8} developed a ``Solidity Summarizer'' using transformer models to generate natural language comments for Solidity code, aiming to improve readability. Shi et al.~\cite{b9} proposed MM-SCS, a multi-modal search model that combines textual features with structural information from a ``Contract Elements Dependency Graph'' to improve search accuracy. These methods show the promise of semantic understanding but often require complex model training and do not integrate into a complete, end-to-end discovery system.

\subsection{Gap Analysis}
Our work builds upon these foundations but addresses a critical gap. While previous systems focused on either indexing transactions or analyzing code structure, our approach leverages the advanced reasoning capabilities of modern LLMs to create a rich, structured semantic layer directly from source code. Furthermore, our work focuses on a real data pipeline, utilizing eth2dgraph capability to extract the full blockchain data into the database. By integrating this pipeline, complete with deep semantic enrichment into a scalable architecture featuring a graph database and vector search, we provide a full-fledged discovery system that allows developers to find contracts based on what they do, not just what words they contain.