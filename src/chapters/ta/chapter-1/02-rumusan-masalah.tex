\section{Rumusan Masalah}
\label{sec:rumusan-masalah}

Berikut adalah rumusan masalah yang akan dijawab dalam penelitian ini:
\begin{enumerate}
  % \item Dengan banyaknya Smart Contracts yang ada, bagaimana menemukan Smart Contracts dengan fungsionalitas yang sesuai dengan kebutuhan dan spesifikasi yang diberikan?
  % \item Pencarian Smart Contracts masih menggunakan kata kunci, yang biasanya tidak relevan dengan arti atau fungsionalitas dari Smart Contract tersebut, sehingga bagaimana menemukan Smart Contracts dengan semantik yang sesuai?
  % \item Tidak semua Smart Contract dapat terjamin keamanannya, sehingga bagaimana mendapatkan Smart Contracts yang memiliki \textit{compliance} terhadap aturan yang baik?
  % \item Apakah ada cara yang lebih baik untuk memodelkan Smart Contracts dibandingkan model \textit{Minimal Service Model}?
  % \item Bagaimana mengevaluasi ketersesuaian hasil Smart Contracts Discovery berdasarkan semantik?
  
  % \item Bagaimana membangun arsitektur dan mengimplementasikan sistem Smart Contract Discovery berbasis semantik yang mampu menemukan Smart Contracts sesuai dengan kebutuhan pengguna secara lebih akurat dibandingkan metode berbasis kata kunci?
  \item Bagaimana membangun sebuah sistem Smart Contract Discovery yang dapat memberikan hasil pencarian berupa Smart Contract yang sesuai bagi pengembang maupun pengguna Smart Contracts jika hanya diketahui kebutuhan fungsionalitas yang diinginkan?
  % Langchain (difasilitasi LLM) -> mengerti kebutuhan fungsionalitas yang pengguna inginkan dan dapat memberikan hasil yang sesuai
  \item Bagaimana membangun sebuah sistem Smart Contract Discovery yang dapat digunakan secara luas oleh pengguna tanpa berdampak pada kinerja dari Blockchain yang digunakan?
  % Offchain dan dgraph
  \item Bagaimana membangun sebuah sistem Smart Contract Discovery yang dapat mengekstrak informasi dari Smart Contracts yang ada dan menghasilkan model yang dapat digunakan untuk menemukan Smart Contracts yang sesuai dengan kebutuhan pengguna?
  % eth2dgraph mengektstrak informasi, lalu melakukan semantic enrichment, lalu melakukan query (fokus di semantic enrichment nya)
  \item Bagaimana membangun sebuah sistem Smart Contract Discovery yang dapat mengetahui semantik dari kebutuhan dan Smart Contract yang menjawab kebutuhan tersebut?
  % Ini fokus di GIMANA MEKANISME semantic enrichment nya
\end{enumerate}

% Rumusan Masalah berisi masalah utama yang dibahas dalam tugas akhir. Rumusan masalah yang baik memiliki struktur sebagai berikut:

% \begin{enumerate}
% 	\item Penjelasan ringkas tentang kondisi/situasi yang ada sekarang terkait dengan topik utama yang dibahas Tugas Akhir.
% 	\item Pokok persoalan dari kondisi/situasi yang ada, dapat dilihat dari kelemahan atau kekurangannya. \textbf{Bagian ini merupakan inti dari rumusan masalah}.
% 	\item Elaborasi lebih lanjut yang menekankan pentingnya untuk menyelesaikan pokok persoalan tersebut.
% 	\item Usulan singkat terkait dengan solusi yang ditawarkan untuk menyelesaikan persoalan.
% \end{enumerate}

% Penting untuk diperhatikan bahwa persoalan yang dideskripsikan pada subbab ini akan dipertanggungjawabkan di bab Evaluasi apakah terselesaikan atau tidak.