\section{Rumusan Masalah}
\label{sec:rumusan-masalah}

Berikut adalah rumusan masalah yang akan dijawab dalam penelitian ini:
\begin{enumerate}

	\item Bagaimana melakukan \textit{enrichment} dengan LLM yang memberikan pengertian semantik dengan terhadap data Smart Contracts untuk membantu pengguna menemukan Smart Contracts yang sesuai dengan kebutuhan? (RM1)

	\item Bagaimana melakukan \textit{data retrieval} berbasis semantik terhadap Smart Contracts yang sudah \textit{semantically enriched}? (RM2)
	
	\item Bagaimana merancang dan mengimplementasikan sistem Smart Contract Discovery yang mampu menerima input berupa kebutuhan pengguna dalam bentuk bahasa alami dan menghasilkan daftar Smart Contract yang relevan melalui mekanisme pencarian yang efektif? (RM3)
\end{enumerate}

% Rumusan Masalah berisi masalah utama yang dibahas dalam tugas akhir. Rumusan masalah yang baik memiliki struktur sebagai berikut:

% \begin{enumerate}
% 	\item Penjelasan ringkas tentang kondisi/situasi yang ada sekarang terkait dengan topik utama yang dibahas Tugas Akhir.
% 	\item Pokok persoalan dari kondisi/situasi yang ada, dapat dilihat dari kelemahan atau kekurangannya. \textbf{Bagian ini merupakan inti dari rumusan masalah}.
% 	\item Elaborasi lebih lanjut yang menekankan pentingnya untuk menyelesaikan pokok persoalan tersebut.
% 	\item Usulan singkat terkait dengan solusi yang ditawarkan untuk menyelesaikan persoalan.
% \end{enumerate}

% Penting untuk diperhatikan bahwa persoalan yang dideskripsikan pada subbab ini akan dipertanggungjawabkan di bab Evaluasi apakah terselesaikan atau tidak.