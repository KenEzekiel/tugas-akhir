\section{Rumusan Masalah}
\label{sec:rumusan-masalah}

Berikut adalah rumusan masalah yang akan dijawab oleh penelitian ini:
% Masih belum fiks
\begin{enumerate}
  % \item Bagaimana cara menemukan Smart Contracts dalam sebuah Blockchain?
  \item Dengan banyaknya Smart Contracts yang ada, bagaimana menemukan Smart Contracts dengan fungsionalitas yang sesuai dengan kebutuhan dan spesifikasi yang diberikan?
  \item Pencarian Smart Contracts masih menggunakan kata kunci, yang biasanya tidak relevan dengan arti atau fungsionalitas dari Smart Contract tersebut, sehingga bagaimana menemukan Smart Contracts dengan semantik yang sesuai?
  % \item Tidak semua Smart Contract dapat terjamin keamanannya, sehingga bagaimana mendapatkan Smart Contracts yang memiliki \textit{compliance} terhadap aturan yang baik?
  % \item Apakah ada cara yang lebih baik untuk memodelkan Smart Contracts dibandingkan model \textit{Minimal Service Model}?
  \item Bagaimana mengevaluasi ketersesuaian hasil Smart Contracts Discovery berdasarkan semantik?
  % \item Apakah memungkinkan untuk membangun sebuah sistem Smart Contracts Discovery yang dinamis?
\end{enumerate}

% Rumusan Masalah berisi masalah utama yang dibahas dalam tugas akhir. Rumusan masalah yang baik memiliki struktur sebagai berikut:

% \begin{enumerate}
% 	\item Penjelasan ringkas tentang kondisi/situasi yang ada sekarang terkait dengan topik utama yang dibahas Tugas Akhir.
% 	\item Pokok persoalan dari kondisi/situasi yang ada, dapat dilihat dari kelemahan atau kekurangannya. \textbf{Bagian ini merupakan inti dari rumusan masalah}.
% 	\item Elaborasi lebih lanjut yang menekankan pentingnya untuk menyelesaikan pokok persoalan tersebut.
% 	\item Usulan singkat terkait dengan solusi yang ditawarkan untuk menyelesaikan persoalan.
% \end{enumerate}

% Penting untuk diperhatikan bahwa persoalan yang dideskripsikan pada subbab ini akan dipertanggungjawabkan di bab Evaluasi apakah terselesaikan atau tidak.