\section{Rumusan Masalah}
\label{sec:rumusan-masalah}

Berdasarkan kondisi saat ini dimana jumlah Smart Contracts terus bertambah dengan laju yang pesat tanpa infrastruktur yang baik dan mewadahi, berikut adalah rumusan masalah yang akan dijawab dalam penelitian ini:
\begin{enumerate}
	% \item Dengan banyaknya Smart Contracts yang ada, bagaimana menemukan Smart Contracts dengan fungsionalitas yang sesuai dengan kebutuhan dan spesifikasi yang diberikan?
	% \item Pencarian Smart Contracts masih menggunakan kata kunci, yang biasanya tidak relevan dengan arti atau fungsionalitas dari Smart Contract tersebut, sehingga bagaimana menemukan Smart Contracts dengan semantik yang sesuai?
	% \item Tidak semua Smart Contract dapat terjamin keamanannya, sehingga bagaimana mendapatkan Smart Contracts yang memiliki \textit{compliance} terhadap aturan yang baik?
	% \item Apakah ada cara yang lebih baik untuk memodelkan Smart Contracts dibandingkan model \textit{Minimal Service Model}?
	% \item Bagaimana mengevaluasi ketersesuaian hasil Smart Contracts Discovery berdasarkan semantik?

	% \item Bagaimana membangun arsitektur dan mengimplementasikan sistem Smart Contract Discovery berbasis semantik yang mampu menemukan Smart Contracts sesuai dengan kebutuhan pengguna secara lebih akurat dibandingkan metode berbasis kata kunci?

	\item Bagaimana melakukan \textit{enrichment} dengan LLM yang memberikan pengertian semantik dengan terhadap data Smart Contracts untuk membantu pengguna menemukan Smart Contracts yang sesuai dengan kebutuhan? (RM1)

	\item Bagaimana melakukan \textit{data retrieval} berbasis semantik terhadap Smart Contracts yang sudah \textit{semantically enriched}? (RM2)
	      % Data retrieval menggunakan cosine similarity
	\item Bagaimana merancang dan mengimplementasikan sistem Smart Contract Discovery yang mampu menerima input berupa kebutuhan pengguna dalam bentuk bahasa alami dan menghasilkan daftar Smart Contract yang relevan melalui mekanisme pencarian yang efektif? (RM3)
	      % DESIGN AND DEVELOPMENT
	      % mekanisme pencarian dan relevansi hasil.
	      % Langchain (difasilitasi LLM) -> mengerti kebutuhan fungsionalitas yang pengguna inginkan dan dapat memberikan hasil yang sesuai
	      % LLM untuk semantic enrichment
	      % RAG untuk retrieval

	      % sistem dapat diskalakan dan tidak mengganggu performa blockchain.
	      % Offchain dan dgraph

	      % \item Bagaimana mengekstrak informasi dari Smart Contract yang ada dan membangun model representasi yang efektif untuk mendukung pencarian dan klasifikasi? (RM4)
	      % (melalui metadata, source code, dan dokumentasi)
	      % ekstraksi data dan pembentukan model representasi dari smart contract.
	      % eth2dgraph mengektstrak informasi, lalu melakukan semantic enrichment, lalu melakukan query (fokus di semantic enrichment nya)

	      % \item Bagaimana mengembangkan modul yang dapat memahami semantik dari kebutuhan pengguna dan isi Smart Contract, sehingga memungkinkan pemetaan yang tepat antara permintaan dan Smart Contract yang tersedia? (RM5)
	      % mengintegrasikan pemahaman semantik untuk mencocokkan kebutuhan dengan konten smart contract.
	      % Ini fokus di GIMANA MEKANISME semantic enrichment nya
\end{enumerate}

% Rumusan Masalah berisi masalah utama yang dibahas dalam tugas akhir. Rumusan masalah yang baik memiliki struktur sebagai berikut:

% \begin{enumerate}
% 	\item Penjelasan ringkas tentang kondisi/situasi yang ada sekarang terkait dengan topik utama yang dibahas Tugas Akhir.
% 	\item Pokok persoalan dari kondisi/situasi yang ada, dapat dilihat dari kelemahan atau kekurangannya. \textbf{Bagian ini merupakan inti dari rumusan masalah}.
% 	\item Elaborasi lebih lanjut yang menekankan pentingnya untuk menyelesaikan pokok persoalan tersebut.
% 	\item Usulan singkat terkait dengan solusi yang ditawarkan untuk menyelesaikan persoalan.
% \end{enumerate}

% Penting untuk diperhatikan bahwa persoalan yang dideskripsikan pada subbab ini akan dipertanggungjawabkan di bab Evaluasi apakah terselesaikan atau tidak.