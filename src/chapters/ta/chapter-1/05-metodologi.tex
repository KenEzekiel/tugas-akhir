\section{Metodologi}
\label{sec:metodologi}

% Tuliskan semua tahapan yang akan dilalui selama pelaksanaan tugas akhir. Tahapan ini spesifik untuk menyelesaikan persoalan tugas akhir. Tahapan studi literatur tidak perlu dituliskan karena ini adalah pekerjaan yang harus Anda lakukan selama proses pelaksanaan tugas akhir.

Terdapat beberapa tahapan untuk melaksanakan tugas akhir secara sistematis, berikut merupakan tahapan-tahapan yang dilakukan:

\begin{enumerate}
  \item \textbf{Identifikasi permasalahan} \newline
        Pada tahapan ini, dilakukan eksplorasi dan pengumpulan informasi terkait sistem pencarian Smart Contract untuk mengidentifikasi permasalahan utama yang akan diselesaikan.
  \item \textbf{Identifikasi kebutuhan} \newline
        Permasalahan utama akan dielaborasikan menjadi kebutuhan-kebutuhan yang harus dipenuhi oleh solusi yang dirancang.
  \item \textbf{Analisis dan desain sistem} \newline
        Kebutuhan-kebutuhan akan dianalisis untuk dibuat rancangannya yang sesuai dan memenuhi kebutuhan tersebut.
  \item \textbf{Pengembangan prototipe} \newline
        Hasil dari rancangan akan diimplementasikan menggunakan teknologi yang sesuai.
  \item \textbf{Pengujian prototipe} \newline
        Hasil implementasi, yaitu prototipe, akan diuji dengan berbagai kasus nyata yang sesuai dengan kebutuhan yang akan dipenuhi untuk menjamin ketercapaian kebutuhan.
  \item \textbf{Evaluasi dan analisis kinerja prototipe} \newline
        Hasil dari pengujian akan dianalisis dan dievaluasi, untuk memberikan saran perbaikan dan juga kesimpulan dari penelitian.
\end{enumerate}

% \section{Sistematika Pembahasan}
% \label{sec:sistematika-pembahasan}

% \section{Jadwal Pelaksanaan Tugas Akhir}

% Tuliskan rencana kegiatan dan jadwal (dirinci sampai per minggu) mulai dari awal pelaksanaan Tugas Akhir I s.d. sidang tugas akhir berikut milestones dan deliverables yang harus diberikan. Jadwal ini dapat dibantu dengan membuat sebuah tabel timeline.