\section{Tujuan dan Ukuran Keberhasilan Pencapaian}
\label{sec:tujuan-ukuran-keberhasilan-pencapaian}

% Tuliskan tujuan utama dan/atau tujuan detail yang akan dicapai dalam pelaksanaan tugas akhir. Fokuskan pada hasil akhir yang ingin diperoleh setelah tugas akhir diselesaikan, terkait dengan penyelesaian persoalan pada rumusan masalah. Penting untuk diperhatikan bahwa tujuan yang dideskripsikan pada subbab ini akan dipertanggungjawabkan di akhir pelaksanaan tugas akhir apakah tercapai atau tidak. Tuliskan juga ukuran keberhasilan pencapaiannya.

Tujuan yang akan dicapai untuk tugas akhir ini adalah menghasilkan:

\begin{enumerate}
	\item \textbf{Format Semantik yang Dapat Mengakomodasi Pencarian Semantik yang Spesifik (T1)} \newline
	      Sebuah format yang dapat digunakan untuk menyimpan data deskripsi semantik dari Smart Contract yang dapat digunakan untuk melakukan pencarian semantik yang spesifik.

	\item \textbf{Teknik \textit{Semantic Enrichment} Menggunakan LLM yang Efektif (T2)} \newline
	      Implementasi teknik \textit{semantic enrichment} menggunakan \textit{Large Language Models} (LLM) untuk memperkaya deskripsi dan metadata Smart Contract, sehingga meningkatkan akurasi dan relevansi hasil pencarian.

	\item \textbf{Teknik \textit{Data Retrieval} Menggunakan RAG yang Efektif (T3)} \newline
	      Penerapan teknik \textit{Retrieval-Augmented Generation} (RAG) yang efektif untuk mengambil informasi yang relevan dari basis data Smart Contract yang telah diekstraksi dan diperkaya, untuk menghasilkan daftar Smart Contracts yang sesuai dengan kebutuhan pengguna.

	\item \textbf{Mekanisme Ekstraksi Data dari Blockchain yang Terintegrasi (T4)} \newline
	      Sebuah mekanisme yang terintegrasi dengan sistem yang dapat mengekstrak data dari Blockchain dan mengubahnya menjadi format yang dapat digunakan oleh sistem.

	\item \textbf{Sistem Smart Contract Discovery (T5)} \newline
	      Sebuah sistem yang dapat digunakan oleh pengguna, yang dapat diakses dengan beberapa cara, untuk mencari sebuah Smart Contract yang spesifik dan sesuai dengan kebutuhan.

\end{enumerate}

% dapat menghubungkan kebutuhan dari pengguna kepada fungsionalitas dari Smart Contract yang sesuai menggunakan semantik dan memberikan hasil yang sesuai kepada pengguna. Sistem ini diharapkan dapat memudahkan pencarian Smart Contracts dengan mengutamakan keterkaitan semantik, serta dapat digunakan oleh pengguna tanpa mempengaruhi kinerja dari Blockchain yang digunakan.

Ukuran keberhasilan pencapaiannya adalah seberapa sesuai Smart Contract dapat ditemukan. Metrik dari masing-masing ukuran keberhasilan adalah sebagai berikut:

\begin{enumerate}
	\item \textbf{Relevansi Hasil Pencarian Smart Contract (UK1)} \newline
	      Sistem Smart Contract Discovery dapat menghasilkan Smart Contract yang relevan sesuai semantik dan fungsionalitas yang diinginkan oleh pengguna. Metrik yang digunakan adalah akurasi.
	      % Pengukuran:
	      % Gunakan model pre-trained (misalnya, CodeBERT atau BERT) untuk menghasilkan representasi vektor dari query dan smart contract yang ditemukan. (BISA PAKAI LLM)
	      % Hitung nilai cosine similarity antara vektor query dan vektor smart contract untuk memperoleh Semantic Similarity Score.
	      % Jika dataset berlabel tidak tersedia, lakukan evaluasi manual dengan mengambil sampel hasil pencarian dan mengumpulkan umpan balik dari pengguna atau domain expert.
	      % (Alternatif) Gunakan metrik ekstrinsik seperti Click-Through Rate (CTR) jika sistem sudah di-deploy.

	\item \textbf{Kualitas Semantik Data Smart Contract (UK2)} \newline
	      Format semantik yang digunakan untuk menyimpan data dapat mengakomodasi pencarian Smart Contract berdasarkan semantik secara spesifik. Selain itu, data yang dihasilkan dari \textit{semantic enrichment} dapat mendeskripsikan Smart Contract dengan baik. Metrik yang digunakan adalah \textit{Semantic Expresiveness Score}.

	\item \textbf{Kemiripan Hasil Pencarian (UK3)} \newline
	      Sistem Smart Contract Discovery dapat menghasilkan hasil pencarian yang memiliki kemiripan antar satu sama lain dengan query yang sama. Metrik yang digunakan adalah \textit{Semantic Similarity Score}.

	\item \textbf{Konsistensi Hasil Pencarian (UK4)} \newline
	      Sistem Smart Contract Discovery dapat menghasilkan hasil pencarian yang konsisten dengan query yang serupa. Metrik yang digunakan adalah \textit{Jaccard Index} pada K hasil teratas.

	      % \item \textbf{Dampak Terhadap Kinerja Blockchain (UK2)} \newline 
	      % Sistem Smart Contract Discovery dapat berjalan tanpa mempengaruhi kinerja dari Blockchain yang digunakan. Metrik yang digunakan adalah \textit{overhead} penggunaan sumber daya pada Blockchain.
	      % Ini sudah pasti tercapai, karena didesain agar seperti itu, validasi nya gimana?
	      %   Pengukuran:
	      % Lakukan benchmarking pada jaringan blockchain dalam kondisi normal (tanpa integrasi sistem discovery) dan setelah integrasi.
	      % Bandingkan metrik seperti waktu konfirmasi transaksi, beban jaringan, atau total penggunaan gas untuk mengidentifikasi apakah sistem discovery memberikan dampak negatif terhadap kinerja blockchain.
	      % Pengukuran ini dapat dilakukan di lingkungan testnet dengan simulasi beban yang representatif.
	      % \item \textbf{Kinerja dan Skalabilitas Sistem (UK3)} \newline 
	      % Sistem Smart Contract Discovery dapat berjalan dengan latensi rendah dan skalabilitas yang baik. Metrik yang digunakan adalah \textit{Response Time}, \textit{Throughput}, dan \textit{Resource Utilization}.
	      %   Pengukuran:
	      % Gunakan tools performance testing (misalnya, Apache JMeter atau Gatling) untuk mensimulasikan beban pengguna dan mengukur response time sistem.
	      % Ukur throughput dengan mencatat jumlah permintaan pencarian yang diproses dalam jangka waktu tertentu.
	      % Pantau resource utilization menggunakan monitoring tools standar seperti Prometheus dan Grafana untuk memastikan sistem dapat menangani beban secara efisien, terutama pada kondisi puncak.
\end{enumerate}