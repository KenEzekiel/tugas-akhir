\section{Tujuan dan Ukuran Keberhasilan Pencapaian}
\label{sec:tujuan-ukuran-keberhasilan-pencapaian}

% Tuliskan tujuan utama dan/atau tujuan detail yang akan dicapai dalam pelaksanaan tugas akhir. Fokuskan pada hasil akhir yang ingin diperoleh setelah tugas akhir diselesaikan, terkait dengan penyelesaian persoalan pada rumusan masalah. Penting untuk diperhatikan bahwa tujuan yang dideskripsikan pada subbab ini akan dipertanggungjawabkan di akhir pelaksanaan tugas akhir apakah tercapai atau tidak. Tuliskan juga ukuran keberhasilan pencapaiannya.

Tujuan yang akan dicapai untuk tugas akhir ini adalah menghasilkan:

\begin{enumerate}
	\item \textbf{Format Semantik yang Dapat Mengakomodasi Pencarian Semantik yang Spesifik (T1)} \newline
	      Sebuah format yang dapat digunakan untuk menyimpan data deskripsi semantik dari Smart Contract yang dapat digunakan untuk melakukan pencarian semantik yang spesifik.

	\item \textbf{Teknik \textit{Semantic Enrichment} Menggunakan LLM yang Efektif (T2)} \newline
	      Implementasi teknik \textit{semantic enrichment} menggunakan \textit{Large Language Models} (LLM) untuk memperkaya deskripsi dan metadata Smart Contract, sehingga meningkatkan akurasi dan relevansi hasil pencarian.

	\item \textbf{Teknik \textit{Data Retrieval} Menggunakan Vector Embeddings yang Efektif (T3)} \newline
	      Penerapan teknik \textit{data retrieval} menggunakan \textit{vector embeddings} yang efektif untuk mengambil informasi yang relevan dari basis data Smart Contract yang telah diekstraksi dan diperkaya, untuk menghasilkan daftar Smart Contracts yang sesuai dengan kebutuhan pengguna.

	\item \textbf{Mekanisme Ekstraksi Data dari Blockchain yang Terintegrasi (T4)} \newline
	      Sebuah mekanisme yang terintegrasi dengan sistem yang dapat mengekstrak data dari Blockchain dan mengubahnya menjadi format yang dapat digunakan oleh sistem.

	\item \textbf{Sistem Smart Contract Discovery (T5)} \newline
	      Sebuah sistem yang dapat digunakan oleh pengguna, yang dapat diakses dengan beberapa cara, untuk mencari sebuah Smart Contract yang spesifik dan sesuai dengan kebutuhan.

\end{enumerate}

% dapat menghubungkan kebutuhan dari pengguna kepada fungsionalitas dari Smart Contract yang sesuai menggunakan semantik dan memberikan hasil yang sesuai kepada pengguna. Sistem ini diharapkan dapat memudahkan pencarian Smart Contracts dengan mengutamakan keterkaitan semantik, serta dapat digunakan oleh pengguna tanpa mempengaruhi kinerja dari Blockchain yang digunakan.

Ukuran keberhasilan pencapaiannya adalah seberapa sesuai Smart Contract dapat ditemukan. Metrik dari masing-masing ukuran keberhasilan adalah sebagai berikut:

\begin{enumerate}
	\item \textbf{Relevansi Hasil Pencarian Smart Contract (UK1)} \newline
	      Sistem Smart Contract Discovery dapat menghasilkan Smart Contract yang relevan sesuai semantik dan fungsionalitas yang diinginkan oleh pengguna. Metrik yang digunakan adalah presisi.

	\item \textbf{Perbandingan Hasil dengan Sistem Pencarian Berbasis Kata Kunci (UK2)} \newline
	      Sistem Smart Contract Discovery dapat menghasilkan hasil pencarian yang lebih relevan dibandingkan dengan sistem pencarian berbasis kata kunci. Metrik yang digunakan adalah peningkatan presisi dibandingkan dengan sistem pencarian berbasis kata kunci.
\end{enumerate}