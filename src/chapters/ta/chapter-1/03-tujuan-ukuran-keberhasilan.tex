\section{Tujuan dan Ukuran Keberhasilan Pencapaian}
\label{sec:tujuan-ukuran-keberhasilan-pencapaian}

% Tuliskan tujuan utama dan/atau tujuan detail yang akan dicapai dalam pelaksanaan tugas akhir. Fokuskan pada hasil akhir yang ingin diperoleh setelah tugas akhir diselesaikan, terkait dengan penyelesaian persoalan pada rumusan masalah. Penting untuk diperhatikan bahwa tujuan yang dideskripsikan pada subbab ini akan dipertanggungjawabkan di akhir pelaksanaan tugas akhir apakah tercapai atau tidak. Tuliskan juga ukuran keberhasilan pencapaiannya.

Tujuan yang akan dicapai untuk tugas akhir ini adalah mengembangkan sebuah sistem Smart Contract Discovery yang mampu: 

\begin{enumerate}
  \item \textbf{Menghubungkan Kebutuhan Pengguna dengan Fungsionalitas Smart Contract} \newline
  Menggunakan pendekatan semantik untuk menerjemahkan dan memetakan kebutuhan fungsionalitas dari pengguna dengan fitur dan fungsi Smart Contract yang relevan, sehingga dapat menghasilkan daftar Smart Contract yang sesuai dengan kebutuhan pengguna.
  \item \textbf{Meningkatkan Relevansi dan Akurasi Hasil Pencarian} \newline
  Menerapkan teknik ekstraksi informasi, representasi semantik, serta mekanisme pencarian untuk memastikan hasil pencarian tidak hanya relevan tetapi juga akurat.
  \item \textbf{Menjamin Efisiensi dan Skalabilitas Sistem} \newline
  Merancang arsitektur sistem yang tidak membebani kinerja jaringan Blockchain tetapi dapat digunakan secara luas.
\end{enumerate}

% dapat menghubungkan kebutuhan dari pengguna kepada fungsionalitas dari Smart Contract yang sesuai menggunakan semantik dan memberikan hasil yang sesuai kepada pengguna. Sistem ini diharapkan dapat memudahkan pencarian Smart Contracts dengan mengutamakan keterkaitan semantik, serta dapat digunakan oleh pengguna tanpa mempengaruhi kinerja dari Blockchain yang digunakan.

Ukuran keberhasilan pencapaiannya adalah seberapa sesuai Smart Contract dapat ditemukan. Metrik dari masing-masing ukuran keberhasilan adalah sebagai berikut:

\begin{enumerate}
  \item \textbf{Relevansi Hasil Pencarian Smart Contract} \newline 
  Sistem Smart Contract Discovery dapat menghasilkan Smart Contract yang relevan sesuai semantik dan fungsionalitas yang diinginkan oleh pengguna. Metrik yang digunakan adalah \textit{Semantic Similarity Score}. 
  % Pengukuran:
  % Gunakan model pre-trained (misalnya, CodeBERT atau BERT) untuk menghasilkan representasi vektor dari query dan smart contract yang ditemukan. (BISA PAKAI LLM)
  % Hitung nilai cosine similarity antara vektor query dan vektor smart contract untuk memperoleh Semantic Similarity Score.
  % Jika dataset berlabel tidak tersedia, lakukan evaluasi manual dengan mengambil sampel hasil pencarian dan mengumpulkan umpan balik dari pengguna atau domain expert.
  % (Alternatif) Gunakan metrik ekstrinsik seperti Click-Through Rate (CTR) jika sistem sudah di-deploy.
  
  \item \textbf{Dampak Terhadap Kinerja Blockchain} \newline 
  Sistem Smart Contract Discovery dapat berjalan tanpa mempengaruhi kinerja dari Blockchain yang digunakan. Metrik yang digunakan adalah \textit{overhead} penggunaan sumber daya pada Blockchain.
  % Ini sudah pasti tercapai, karena didesain agar seperti itu, validasi nya gimana?
  %   Pengukuran:
  % Lakukan benchmarking pada jaringan blockchain dalam kondisi normal (tanpa integrasi sistem discovery) dan setelah integrasi.
  % Bandingkan metrik seperti waktu konfirmasi transaksi, beban jaringan, atau total penggunaan gas untuk mengidentifikasi apakah sistem discovery memberikan dampak negatif terhadap kinerja blockchain.
  % Pengukuran ini dapat dilakukan di lingkungan testnet dengan simulasi beban yang representatif.
  \item \textbf{Kinerja dan Skalabilitas Sistem} \newline 
  Sistem Smart Contract Discovery dapat berjalan dengan latensi rendah dan skalabilitas yang baik. Metrik yang digunakan adalah \textit{Response Time}, \textit{Throughput}, dan \textit{Resource Utilization}.
  %   Pengukuran:
  % Gunakan tools performance testing (misalnya, Apache JMeter atau Gatling) untuk mensimulasikan beban pengguna dan mengukur response time sistem.
  % Ukur throughput dengan mencatat jumlah permintaan pencarian yang diproses dalam jangka waktu tertentu.
  % Pantau resource utilization menggunakan monitoring tools standar seperti Prometheus dan Grafana untuk memastikan sistem dapat menangani beban secara efisien, terutama pada kondisi puncak.
\end{enumerate}