\section{Sistematika Pembahasan}

Pembahasan pada laporan ini akan dibagi menjadi lima bagian yang terdiri dari:

\begin{enumerate}
	\item \textbf{Pendahuluan} \\
	      Bab I akan menjelaskan gagasan utama dari topik tugas akhir ini yang terdiri dari latar belakang dan tantangan yang ada pada saat ini, perumusan masalah dari tantangan-tantangan yang ada, penentuan tujuan, batasan permasalahan, metodologi pengembangan, dan sistematika pembahasan mengenai proses pengembangan solusi.
	\item \textbf{Studi Literatur} \\
	      Bab II akan menjelaskan terkait studi literatur yang menjadi landasan untuk pengetahuan dan pengembangan dari solusi.
	\item \textbf{Analisis Persoalan dan Rancangan Solusi} \\
	      Bab III akan menjelaskan hasil analisis dari masalah yang ditemukan pada kondisi yang ada pada saat ini diikuti dengan dampaknya, kebutuhan yang muncul dari analisis tersebut, dan ajuan solusi yang diusulkan untuk menyelesaikan masalah yang ada. Selain itu, bab ini juga akan menjelaskan alternatif solusi yang dipertimbangkan dan alasan mengapa solusi yang diusulkan dipilih.
	\item \textbf{Implementasi dan Pengujian} \\
	      Bab IV akan menjelaskan terkait hasil implementasi dan rancangan yang telah dibuat, baik dari segi arsitektur dan teknologi yang digunakan, hingga hasil pengujian yang dilakukan untuk memastikan kualitas dari solusi yang diusulkan.
	\item \textbf{Kesimpulan dan Saran} \\
	      Bab V menjadi penutup dari laporan ini, yang akan menjelaskan kesimpulan dari hasil pengembangan solusi yang telah dilakukan, serta saran dari penulis untuk pengembangan selanjutnya.
\end{enumerate}