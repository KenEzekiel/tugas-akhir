\section{Batasan Masalah}
\label{sec:batasan-masalah}

% Tuliskan batasan-batasan yang diambil dalam pelaksanaan tugas akhir. Batasan ini dapat dihindari (tidak perlu ada) jika topik/judul tugas akhir dibuat cukup spesifik.

Batasan masalah yang diambil adalah sebagai berikut:

\begin{enumerate}
	\item \textbf{Pencarian Hanya pada Smart Contract dengan \textit{Source Code} yang Tersedia dan Terverifikasi (BM1)} \newline
	      Alasan dari batasan ini adalah untuk berfokus kepada \textit{proof-of-concept} dari sistem Smart Contract Discovery, bukan kepada aspek teknis dari \textit{decompilation} Bytecode. Selain itu, Smart Contracts yang tidak terverifikasi tidak dianjurkan untuk digunakan karena risiko keamanan yang tinggi.

	\item \textbf{Analisis dan Implementasi Difokuskan pada Blockchain Ethereum Mainnet (BM2)} \newline
	      Alasan dari batasan ini adalah karena Ethereum mainnet merupakan ekosistem yang paling banyak digunakan dan menyediakan data yang cukup. Pembatasan ini bertujuan untuk menyederhanakan analisis dan pengembangan.

	\item \textbf{Fokus Tugas Akhir berada pada \textit{Pipeline} Utama Sebagai \textit{Proof-of-concept} (BM3)} \newline
	      Batasan ini muncul karena \textit{pipeline} dapat dilakukan ekstensi dengan mudah untuk melakukan \textit{improvement}. Sehingga, yang akan dikembangkan pada tugas akhir ini adalah \textit{pipeline} utama sebagai \textit{proof-of-concept}.
\end{enumerate}