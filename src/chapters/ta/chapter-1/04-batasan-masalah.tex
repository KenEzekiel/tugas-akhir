\section{Batasan Masalah}
\label{sec:batasan-masalah}

% Tuliskan batasan-batasan yang diambil dalam pelaksanaan tugas akhir. Batasan ini dapat dihindari (tidak perlu ada) jika topik/judul tugas akhir dibuat cukup spesifik.

Batasan masalah yang diambil adalah sebagai berikut:

\begin{enumerate}
	\item \textbf{Pencarian Hanya pada Smart Contract dengan \textit{Source Code} yang Tersedia dan Terverifikasi (BM1)} \newline
	      Alasan dari batasan ini adalah untuk berfokus kepada \textit{proof-of-concept} dari sistem Smart Contract Discovery, bukan kepada aspek teknis dari \textit{decompilation} Bytecode. Selain itu, Smart Contracts yang tidak terverifikasi tidak dianjurkan untuk digunakan karena risiko keamanan yang tinggi.
	\item \textbf{Analisis dan Implementasi Difokuskan pada Blockchain Ethereum Mainnet (BM2)} \newline
	      Alasan dari batasan ini adalah karena Ethereum mainnet merupakan ekosistem yang paling banyak digunakan dan menyediakan data yang cukup. Pembatasan ini bertujuan untuk menyederhanakan analisis dan pengembangan.
	% \item \textbf{Pemodelan Fungsionalitas Smart Contract Hanya pada Fungsionalitas Umum (BM3)} \newline
	%       Alasan dari batasan ini adalah untuk mengembangkan model pencarian yang lebih efisien dan relevan untuk berbagai jenis Smart Contract. Hal ini dilakukan untuk menghindari kerumitan yang timbul dari variasi fitur yang sangat spesifik dan memudahkan standarisasi dalam model pencarian.
	      % \item Sistem Smart Contract Discovery dirancang untuk menghasilkan sejumlah hasil yang relevan tanpa perlu simulasi besar atau pengujian dengan berbagai variasi fungsionalitas. 
	\item \textbf{Fokus Tugas Akhir berada pada Pengembangan Sistem Pencarian (BM4)} \newline
	      Untuk memastikan lingkup yang jelas, maka fokus dari Tugas Akhir ini adalah pada mekanisme pencarian data Smart Contracts, bukan pada pengembangan antarmuka pengguna, pengujian keamanan Smart Contracts, atau aspek lain yang tidak terkait langsung dengan pencarian. Mekanisme pencarian yang dimaksud adalah bagaimana sistem dapat menemukan Smart Contracts yang relevan dengan kebutuhan.
	      % \item \textbf{Pengujian Kinerja Sistem Dilakukan dalam Lingkungan Terkontrol} \newline
	      % Alasan dari batasan ini adalah untuk menghindari dampak variabilitas eksternal dan memastikan data yang diperoleh valid dan akurat.
\end{enumerate}