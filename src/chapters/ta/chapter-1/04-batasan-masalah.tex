\section{Batasan Masalah}
\label{sec:batasan-masalah}

% Tuliskan batasan-batasan yang diambil dalam pelaksanaan tugas akhir. Batasan ini dapat dihindari (tidak perlu ada) jika topik/judul tugas akhir dibuat cukup spesifik.

Batasan masalah yang diambil adalah sebagai berikut:

\begin{enumerate}
  \item \textbf{Pencarian Hanya pada Smart Contract dengan \textit{Source Code} yang Tersedia dan Terverifikasi} \newline
  Alasan dari batasan ini adalah untuk berfokus kepada \textit{proof-of-concept} dari sistem Smart Contract Discovery, bukan kepada aspek teknis dari \textit{decompilation} Bytecode. Setelah konsep dari sistem pencarian ini terbuktif efektif, sistem dapat diperluas untuk mencakup Smart Contract yang tidak memiliki \textit{source code}.
  \item \textbf{Analisis dan Implementasi Difokuskan pada Blockchain Ethereum Mainnet} \newline
  Alasan dari batasan ini adalah karena Ethereum mainnet merupakan ekosistem yang paling banyak digunakan dan menyediakan data yang cukup. Pembatasan ini bertujuan untuk menyederhanakan analisis dan pengembangan.
  \item \textbf{Pemodelan Fungsionalitas Smart Contract Hanya pada Fungsionalitas Umum} \newline
  Alasan dari batasan ini adalah untuk mengembangkan model pencarian yang lebih efisien dan relevan untuk berbagai jenis Smart Contract. Hal ini dilakukan untuk menghindari kerumitan yang timbul dari variasi fitur yang sangat spesifik dan memudahkan standarisasi dalam model pencarian.
  % \item Sistem Smart Contract Discovery dirancang untuk menghasilkan sejumlah hasil yang relevan tanpa perlu simulasi besar atau pengujian dengan berbagai variasi fungsionalitas. 
  \item \textbf{Fokus Tugas Akhir pada Pengembangan Mekanisme Pencarian yang Efektif dan Relevan} \newline
  Alasan dari batasan ini adalah untuk memastikan tugas akhir ini tetap memprioritaskan aspek pencarian, terutama dalam menghubungkan kebutuhan pengguna dengan fungsionalitass Smart Contract melalui pendekatan semantik. Aspek lainnya seperti evaluasi keamanan, integrasi mendalam dengan Blockchain, atau pengembangan fitur tambahan dianggap sebagai pengembangan lanjutan di masa depan.
  % \item \textbf{Pengujian Kinerja Sistem Dilakukan dalam Lingkungan Terkontrol} \newline
  % Alasan dari batasan ini adalah untuk menghindari dampak variabilitas eksternal dan memastikan data yang diperoleh valid dan akurat.
\end{enumerate}