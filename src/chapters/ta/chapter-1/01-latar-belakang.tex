\section{Latar Belakang}
\label{sec:latarbelakang}

% Latar Belakang berisi dasar pemikiran, kebutuhan atau alasan yang menjadi ide dari topik tugas akhir. Tujuan utamanya adalah untuk memberikan informasi secukupnya kepada pembaca agar memahami topik yang akan dibahas.  Saat menuliskan bagian ini, posisikan anda sebagai pembaca – apakah anda tertarik untuk terus membaca?
% ide dan kondisi aktual dari landscapenya sekarang
Konsep dasar blockchain berakar pada karya \cite{haber1991time} yang memperkenalkan sebuah metode untuk mencatat dokumen dengan cap waktu yang tidak dapat diubah. Konsep dasar tersebut kemudian dikembangkan oleh \cite{nakamoto2008bitcoin}, dalam karyanya, Bitcoin Whitepaper, sebuah dokumen yang memperkenalkan konsep Bitcoin kepada dunia yang menggabungkan konsep \textit{digital timestamping} yang diperkenalkan oleh \cite{haber1991time} dengan konsep kriptografi \parencite{hellman1976new} \parencite{standard1995secure}, Merkle tree \parencite{merkle1987digital}, Konsensus \textit{proof-of-work} \parencite{dwork1992pricing}, dan Smart Contracts \parencite{szabo1997formalizing}. Setelah perilisan Bitcoin pada tahun 2009, terjadi peningkatan eksponensial dalam adopsi dan penelitian terkait blockchain dan \textit{cryptocurrency}. Secara pasar finansial, dunia blockchain dan \textit{cryptocurrency} sudah mencapai USD 26,91 miliar pada tahun 2024 \parencite{rosencrance2024top}, dan diprediksi untuk bertumbuh sampai USD 825,93 miliar pada tahun 2032. Pertumbuhan tersebut menandakan investasi yang tinggi pada teknologi blockchain, diiringi dengan perkembangan teknologi dan juga peluang yang bertumbuh pada teknologi blockchain.

Dengan diperkenalkannya Ethereum oleh \cite{buterin2013ethereum}, teknologi blockchain memungkinkan realisasi Smart Contracts yang pertama kali diusulkan oleh \cite{szabo1997formalizing}. Realisasi Smart Contracts ini membuka peluang besar dengan mereduksi risiko, menurunkan biaya administrasi, meningkatkan efisiensi proses bisnis, serta mendukung pengembangan aplikasi dalam berbagai spektrum \parencite{zheng2020overview}.

Smart Contracts bukan hanya dimanfaatkan untuk melakukan transaksi di dalam blockchain dengan lebih efisien, terpercaya, dan \textit{trustless}, dan juga menjadi dasar bagi pengembangan dApps (\textit{Decentralized Applications}), sebuah aplikasi terdesentralisasi yang memanfaatkan Smart Contracts untuk menjalankan logika bisnis dan pemrosesannya. Smart Contracts menjadi konsep yang penting di dalam blockchain. Sejak pengenalannya, terjadi pertumbuhan drastis dalam jumlah kontrak yang dibuat. Pada kuartal 1 tahun 2022 saja, tercatat 1,45 juta Smart Contracts baru yang di-\textit{deploy}, meningkat sebesar 24,7\% dari kuartal 4 tahun 2021 yang mencatat 1,16 juta Smart Contracts \parencite{alchemy_ethereum_statistics}.

Namun, pertumbuhan ini juga menghadirkan berbagai tantangan, seperti:
\begin{itemize}
	\item \textbf{Kesulitan dalam Menemukan Smart Contract yang Sesuai} \newline
	      Terdapat banyak Smart Contract publik, sehingga sulit bagi pengguna atau pengembang menemukan dan menggunakan Smart Contract yang sesuai dengan kegunaan yang diinginkan.
	\item \textbf{Hasil Pencarian yang Terlalu Spesifik atau Tidak Relevan} \newline
	      Metode pencarian berbasis kata kunci sering menghasilkan hasil yang tidak tepat, baik terlalu sempit maupun tidak relevan, sehingga menghambat proses pencarian Smart Contract yang sesuai.
	\item \textbf{Latensi Pencarian yang Meningkat} \newline
	      Pertumbuhan data di blockchain menyebabkan latensi pencarian yang semakin tinggi, sehingga proses pencarian Smart Contracts menjadi lebih lambat.
	\item \textbf{Keterbatasan Informasi Karena Tidak Selalu Ada \textit{Source Code}} \newline
	      Tidak semua Smart Contract yang telah di-\textit{deploy} disertai dengan \textit{source code}, sehingga sulit untuk mendapatkan informasi semantik yang mendalam baik bagi pengguna maupun sistem.
	\item \textbf{Kesulitan dalam Memahami Konteks Semantik dari \textit{Source Code}} \newline
		  Pemahaman konteks semantik yang benar sulit dan memakan waktu yang lama bagi pengguna, terutama jika mereka tidak memiliki latar belakang teknis yang kuat.
	\item \textbf{Kurangnya Dokumentasi dan Informasi yang Jelas} \newline
	      Banyak Smart Contract yang kekurangan dokumentasi dan informasi penjelasan, sehingga menyulitkan pemahaman dan evaluasi Smart Contract tersebut.
	\item \textbf{Kurangnya Standarisasi dan Pengartian Semantik} \newline
	      Tidak adanya standarisasi serta definisi semantik yang jelas mengenai aspek dan fungsionalitas Smart Contract menyulitkan adopsi dan pemanfaatan yang optimal.
	\item \textbf{Redundansi Penulisan Smart Contract} \newline
	      Karena Smart Contract yang serupa sulit ditemukan, pengembang cenderung menulis ulang kontrak yang sebenarnya sudah ada, sehingga terjadi redundansi dalam blockchain.
	\item \textbf{Kurangnya Standar Keamanan dan Mekanisme \textit{Reuse}} \newline
	      Tidak terdapat mekanisme standar yang memudahkan penggunaan ulang Smart Contract, sehingga standar keamanan dalam penulisan Smart Contract pun belum optimal.
	\item \textbf{Hambatan Integrasi dan Kurangnya Interoperabilitas} \newline
	      Desain Smart Contract yang tidak mendukung interoperabilitas menghambat integrasi dan kolaborasi antar sistem atau Smart Contract lain.
	\item \textbf{Kesulitan Menilai Keamanan, Kredibilitas, dan Kualitas} \newline
	      Tidak adanya mekanisme evaluasi yang standar membuat penilaian terhadap keamanan, kredibilitas, dan kualitas Smart Contract menjadi sulit.
	\item \textbf{Sulit Mengidentifikasi Versi Terbaru dan Efisien} \newline
	      Pengguna sering mengalami kesulitan dalam menemukan versi Smart Contract yang terbaru dan lebih efisien, sehingga kontrak yang sudah usang atau tidak optimal tetap digunakan.
\end{itemize}

Tantangan-tantangan ini jika dibiarkan dengan laju pertumbuhan Smart Contracts yang terus naik, akan membuat ekosistem pengembangan dan juga penggunaan Smart Contracts tidak efisien. Hal ini juga didukung dengan data yang didapatkan oleh \cite{aimar2023extraction}, yang mendapatkan bahwa hanya 5,78\% dari Smart Contracts yang pernah setidaknya menerima sebuah transaksi dan mengemisi sebuah log. Di tengah banyaknya Smart Contracts ini, kesulitan pencarian akan membuat lebih banyak pengguna membuat kontrak baru yang lebih tidak terstandarisasi dan \textit{error-prone} dibandingkan menggunakan yang sudah ada, walaupun Smart Contracts yang sudah ada memiliki fungsionalitas yang sama persis dengan apa yang dibutuhkan pengguna dan lebih terjamin secara standar implementasi. Sehingga menyebabkan inefisiensi secara usaha, waktu, dan juga besar data yang disimpan di dalam blockchain. Hal ini juga berarti akan sulit mengembangkan sistem yang lebih kompleks menggunakan Smart Contracts.

Terdapat beberapa penelitian yang dapat membantu menyelesaikan tantangan-tantangan tersebut, seperti yang dilakukan oleh \cite{third2017linked}, di mana dilakukan \textit{indexing} terhadap data di dalam blockchain, termasuk juga data Smart Contract, dan melakukan \textit{mapping} kedua data tersebut ke sebuah ontologi untuk memudahkan pencarian. Penelitian lainnya yang dilakukan oleh \cite{aimar2023extraction}, membuat sebuah perangkat lunak, eth2dgraph, yang dapat melakukan \textit{mapping} antara data di dalam Blockchain Ethereum ke dalam basis data Dgraph, dan juga beberapa penelitian lainnya seperti yang dilakukan oleh \cite{baqa2019semantic} dan \cite{cano2021toward} juga mencoba menyelesaikan tantangan-tantangan tersebut. Penelitian-penelitian ini, walau inovatif, belum menyelesaikan tantangan yang ada secara nyata, karena kurangnya implementasi pada sistem blockchain dan koneksi dengan penggunanya.

Sejak tahun 2017, yaitu setelah diperkenalkannya arsitektur Transformer oleh Google yang menjadi tulang punggung dari \textit{Large Language Model} (LLM) modern, terdapat perkembangan pesat dalam dunia \textit{Artificial Intelligence} (AI), terutama dalam \textit{Natural Language Processing} (NLP). Perkembangan yang pesat ini membawa banyak inovasi baru yang dapat diintegrasikan dengan berbagai teknologi lainnya untuk menghasilkan solusi yang inovatif.

Pendekatan yang belum tereksplorasi untuk menjawab tantangan-tantangan pada ekosistem blockchain adalah menggunakan pendekatan \textit{semantic understanding} pada data Smart Contracts, seperti yang digunakan oleh penelitian seperti yang dilakukan oleh \cite{third2017linked}, \cite{shi2021semantic}, \cite{stan}, dan \cite{sopek2018graphchain}, tetapi dengan penggunaan bantuan LLM untuk melakukan \textit{semantic enrichment}, \textit{semantic understanding}, dan \textit{data retrieval}. Secara teori, dengan memanfaatkan LLM dan \textit{vector embeddings}, sistem dapat memahami semantik dari kebutuhan pengguna dan isi serta konteks semantik dari Smart Contracts, sehingga memungkinkan pemetaan yang tepat secara semantik antara permintaan dan Smart Contracts yang tersedia.

% Penelitian-penelitian dan kapabilitas dari \textit{Artificial Intelligence} sekarang menjadi fondasi untuk solusi yang ditawarkan pada tugas akhir ini untuk menjawab tantangan-tantangan dalam ekosistem Blockhain, yaitu sebuah sistem Smart Contract Discovery yang dapat memudahkan pencarian Smart Contracts berdasarkan semantik untuk membantu pengguna untuk menggunakan dan mengembangkan Smart Contract di dalam ekosistem blockchain dengan memanfaatkan LLM dan RAG.

Pada tugas akhir ini, akan dieksplorasi dan dibangun sebuah prototipe sistem dengan pendekatan yang menggunakan LLM untuk melakukan \textit{semantic understanding}, terutama pada konteks Smart Contracts yang akan membantu \textit{data retrieval} berbasis semantik.

% Meskipun Smart Contracts memiliki potensi yang sangat besar untuk mengubah cara kerja proses bisnis konvensional dan mengefisiensikan sistem, masih terdapat banyak tantangan yang perlu dijawab, seperti masalah terkait \textit{privacy}, \textit{security}, \textit{interoperability}, dan lainnya.

% kayanya perlu ditambahin terkait Smart Contract itu banyak banget, dan susah discover yang sesuai, dan bisa interoperable, intinya introduce kebutuhan discoverability disini. kaya sesuai fungsionalitas. dan gimana inituh bisa sangat membantu, misal dengan Smart Contract yang discoverable, bakal lebih efisien atau interoperable, introduce juga usecase yang developernya disini

% Tugas Akhir ini berfokus untuk menjawab tantangan di dalam Smart Contract Discovery, yang akan berefek pada berbagai faktor seperti interoperabilitas, integritas, aksesibilitas, dan \textit{compliance}. Tujuan dari sistem tersebut adalah untuk memudahkan pencarian Smart Contracts berdasarkan semantik, sehingga hasil dari pencarian relevan dengan kebutuhan. 

% Sebagai hasil dari Tugas Akhir ini, akan dibangun sebuah sistem yang memanfaatkan indeks \textit{linked-data} di dalam blockchain, dan menggunakan sebuah ekstensi dari Semantic Smart Contract Language untuk mempermudah pencarian Smart Contract yang sesuai dengan fungsionalitas tertentu. Perangkat lunak ini diharapkan dapat memperbolehkan pencarian Smart Contract berdasarkan semantik, sebagai langkah pertama untuk mencapai interoperabilitas dan integrasi dari sistem pengembangan Smart Contract.

% ada indexingnya, gmn cara nyarinya itu query 

% nah yang gua mau bikin itu discoverynya, gimana dapetin sc yang tepat untuk kebutuhan tertentu

% dengan pembuatan sc discovery juga bisa dimanfaatkan untuk bikin package manager buat Smart Contract itu, tapi itu udah different topic, tapi harus dipikirin format yang jadi si sistem inituh bisa dipakai oleh banyak aplikasi dengan mudah, either dia json or something

% (abis ini ngomongin, banyak juga riset paper yang rilis, sekitar X jumlahnya, yang menandakan bahwa teknologinya terus berkembang) Dengan diperkenalkannya Ethereum oleh  (nah ethereum memperkenalkan Smart Contract, dan disitu jadi muncul banyak isu buat optimasi seperti skalabilitas, security, privacy, dan salah satunya adalah interoperabilitas, nah si discovery ini bukan cuma interoperabilitas, tetapi juga (nah pikirin benefit apa yang muncul dengan si Smart Contract discovery inituh??))

% setelah ini ngomongin terkait tren nya sekarang dan kenapa makin lama makin shift ke Smart Contract, kaya kenapa Smart Contract dipakai dan kenapa butuh, terus jadi ke gimana Smart Contract discovery itu akan dibutuhkan