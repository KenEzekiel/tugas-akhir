\chapter{Analisis Masalah dan Rancangan Solusi}

Bab ini akan berisikan deskripsi detail terkait persoalan, rencana penyelesaian persoalan, dan solusi yang akan dibangun dalam tugas akhir. Bab ini diharapkan dapat memberikan gambaran jelas pada persoalan yang dibahas dan solusi yang akan dibangun.

% Tujuan utama penulisan bab ini adalah untuk menguraikan rencana penyelesaian masalah tugas akhir I.. Bab ini mencakup antara lain: 
% 1.	Deskripsi dan analisis persoalan yang terkait dengan Rumusan Masalah, misalnya menjelaskan secara detail latar belakang dan masalah yang menjadi dasar munculnya topik, menunjukkan gap/celah antara kondisi saat ini dengan kondisi yang diharapkan, dan kaitan antara sistem/aplikasi yang dikembangkan dengan sistem/aplikasi lain yang terkait.
% 2.	Analisis solusi yang terdiri dari pilihan alternatif solusi yang dapat digunakan untuk setiap permasalahan berdasarkan hasil studi literatur atau survei, pemilihan solusi beserta justifikasinya.
% 3.	Deskripsi umum solusi yang dipilih, mencakup:
% a.	Modul/subsistem/komponen yang akan dikembangkan untuk menyelesaikan masalah, berikut penjelasannya.
% b.	Alur umum algoritma atau langkah-langkah pengembangan sistem dan penjelasannya.
% c.	Penggunaan kakas yang diperlukan
% Dianjurkan untuk menggunakan diagram sebagai pendukung penjelasan bagian ini.

\section{Analisis}

\subsection{Analisis Masalah}
\label{subsec:analisis-masalah}

Dalam pengembangan dApps, Smart Contracts berperan sebagai \textit{building blocks} yang saling terhubung dan berinteraksi sesuai dengan fungsionalitasnya. Namun, dengan jumlah Smart Contracts yang terus bertambah di Blockchain dan ketiadaan mekanisme klasifikasi yang jelas, pemilihan Smart Contracts yang tepat menjadi tantangan yang kompleks. Kesulitan ini terutama muncul dalam pengklasifikasian dan pelabelan Smart Contracts secara semantik, yang berdampak pada sulitnya menemukan Smart Contracts yang sesuai dengan kebutuhan pengguna.

Kesulitan dalam menemukan Smart Contracts yang sesuai sering kali mendorong pengembang dApps untuk membuat Smart Contracts khusus untuk aplikasi yang dikembangkan, meskipun fungsionalitasnya serupa dengan yang sudah ada. Akibatnya, jumlah Smart Contracts di Blockchain tumbuh secara eksponensial, memperbesar ukuran Blockchain secara signifikan. Pertumbuhan yang tidak terkendali ini menyebabkan \textit{bloating}, yang berdampak pada inefisiensi penyimpanan dan masalah skalabilitas, sehingga memengaruhi kinerja sistem konsensus. Selain itu, biaya pengembangan dApps pun meningkat karena pengembang harus membuat dan mendeploy Smart Contracts baru untuk setiap aplikasi, meskipun solusi serupa sebenarnya sudah tersedia. Kurangnya dukungan terhadap mekanisme \textit{reusability} ini juga berdampak pada kualitas dari Smart Contracts secara umum karena pengembang cenderung menulis kode yang tidak terjamin secara kualitas.
% terutama dari segi keamanan

\subsubsection{Dekomposisi Masalah}
\label{subsubsec:dekomposisi-masalah}
% masalah utama: sulit untuk menemukan Smart Contracts yang sesuai dengan kebutuhan pengguna di tengah banyaknya Smart Contracts yang ada pada Blockchain. Kesulitan ini terutama muncul dalam pengklasifikasian dan pelabelan Smart Contracts secara semantik, yang berdampak pada sulitnya menemukan Smart Contracts yang sesuai dengan kebutuhan pengguna.
% dampak masalah: pertumbuhan jumlah Smart Contracts yang tidak terkendali dan peningkatan ukuran Blockchain (\textit{bloating}), peningkatan biaya dan waktu serta penurunan kualitas karena kurangnya mekanisme \textit{reusability} Smart Contracts.

% dekomposisi permasalahan
% 1. bagaimana mengekstraksi data smart contracts dari Blockchain
% 2. bagaimana menyimpan dan mengelola data smart contracts yang sudah diekstrak
% 3. (ini nanti jawabannya dengen semantik enrichment) bagaimana melabelkan dan mengklasifikasikan Smart Contracts yang akan mempermudah pencarian berdasarkan kebutuhan pengguna -> untuk identifikasi kebutuhan yang dijawab oleh Smart Contract
% 4. bagaimana melakukan manajemen terhadap data smart contracts yang sudah dilabelkan dan diklasifikasikan
% 5. (ini nanti jawabannya pake RAG) bagaimana menghubungkan permintaan kebutuhan pengguna dengan data yang sudah dilabelkan dan diklasifikasikan

Secara singkat, masalah utama yang diidentifikasi adalah kesulitan untuk menemukan Smart Contracts yang sesuai dengan kebutuhan pengguna di tengah pertumbuhan jumlah Smart Contracts yang ada pada Blockchain. Masalah ini terutama muncul karena ketiadaan mekanisme klasifikasi yang jelas untuk mengidentifikasi Smart Contracts, sehingga pengguna kesulitan dalam menemukan Smart Contracts yang sesuai. 

Permasalahan ini dapat dijawab dengan sebuah mekanisme pencarian Smart Contracts yang lebih akurat dan efisien, yang memungkinkan pengguna untuk menemukan Smart Contracts yang sesuai dengan kebutuhan mereka dengan memanfaatkan pelabelan atau pengklasifikasian dari Smart Contracts. Mekanisme ini juga diinspirasi oleh gabungan dari Semantic Web dan Blockchain seperti pada GraphChain \parencite{sopek2018graphchain}, yang memungkinkan untuk melakukan pencarian Smart Contracts berdasarkan semantik yang lebih mendalam. Dengan demikian, pengguna dapat menemukan Smart Contracts yang sesuai dengan kebutuhan mereka dengan lebih mudah dan efisien.
% berdasarkan kebutuhan yang dijawabnya

% ini masih umum kaya yaudah ini idenya dan emang cara solve nya ini, tapi nanti di bab berikutnya itu fokusnya gimana implementasinya dan ide utama yang menawarkan kebaruannya tuh apa

Untuk mengembangkan mekanisme tersebut, digunakan referensi dengan kasus serupa, yaitu Data Retrieval Pipeline yang digunakan dalam RAG \parencite{CrateDB_RAG_Pipelines}. Referensi ini dipilih karena pada dasarnya mekanisme yang akan dikembangkan merupakan Data Retrieval Pipeline. Berdasarkan referensi tersebut, terdapat beberapa poin penting yang perlu dijawab, dan poin-poin ini akan menjadi dasar pemilihan teknologi serta metode yang digunakan dalam sistem yang akan dibangun. Berikut adalah poin-poin dekomposisi masalah yang menjadi dasar Tugas Akhir ini:

\begin{enumerate}
  \item Bagaimana melakukan ekstraksi data Smart Contracts dari Blockchain? (DK1)
  \item Bagaimana menyimpan dan mengelola data Smart Contracts yang sudah diekstrak? (DK2)
  % ngomongin tentang eth2dgraph dan format schema
  \item Bagaimana melabelkan atau mengklasifikasikan Smart Contracts yang akan mempermudah pencarian berdasarkan kebutuhan? (DK3)
  \item Bagaimana melakukan manajemen terhadap data Smart Contracts yang sudah dilabelkan dan diklasifikasikan? (DK4)
  % manajemen ini dilakuinnya dengan ada mekanisme update (maybe), atau akan dilakuin dengan ada mekanisme upsert, atau yang dimaksud ADALAH MEKANISME UPDATE DATA YANG UDA DILABELIN KE STORAGE
  \item Bagaimana memahami kebutuhan dari pengguna? (DK5)
  \item Bagaimana menghubungkan permintaan kebutuhan pengguna dengan data yang sudah dilabelkan dan diklasifikasikan? (DK6)
  \item Bagaimana melakukan pencarian dan penggunaan ulang tanpa mempengaruhi kinerja sistem? (DK7)
  \item Bagaimana mempermudah integrasi Smart Contracts dalam pengembangan dApps? (DK8)
  % \item Bagaimana melakukan \textit{semantic enrichment} terhadap data Smart Contracts? (DK3)
  % ini salah, semantic enrichment itu belum muncul dari sini harusnya
  % \item Bagaimana melakukan query terhadap data Smart Contracts yang sudah dilakukan \textit{semantic enrichment}? 
  % salah, terlalu spesifik, sama (harusnya RAG ini muncul di bawah)
  % \item Bagaimana menghubungkan permintaan kebutuhan pengguna dengan query terhadap data yang sesuai? 
\end{enumerate}

% bisa dipakai untuk memperbaiki rumusan masalah (kalau masih kurang jelas)

\subsubsection{Dampak Masalah}
\label{subsubsec:dampak-masalah}

Secara lebih luas, dampak dari permasalahan utama yang telah diidentifikasi adalah:

\begin{enumerate}
  \item \textbf{\textit{Codebase} yang Kompleks} \newline
  Redundansi pengembangan Smart Contracts menghasilkan \textit{codebase} yang tumbuh besar dan semakin sulit untuk dipelihara. Kompleksitas ini dapat dihindari jika pengembang mampu menggunakan kembali Smart Contracts yang sudah ada, sehingga efisiensi pengelolaan kode meningkat.
  \item \textbf{Keterbatasan Pengembangan Sistem} \newline
  Lingkungan pengembangan Smart Contracts yang saat ini sudah mendukung arsitektur berbasis Service-Oriented Architecture (SOA) atau Object-Oriented Approach (OOA) menjadi kurang optimal, karena pengembang lebih banyak berfokus pada pembuatan kontrak baru daripada memanfaatkan kembali kontrak yang sudah ada. Hal ini membatasi potensi inovasi pengembangan internal dApps.
  \item \textbf{Biaya dan Waktu Pengembangan yang Tinggi} \newline 
  Membuat dan mendeploy Smart Contracts baru membutuhkan biaya signifikan, termasuk biaya \textit{deployment} ke jaringan blockchain, yang semakin besar dengan frekuensi pengembangan yang tinggi. Selain itu, waktu yang dihabiskan untuk membuat kontrak baru juga meningkat, memperlambat proses pengembangan aplikasi secara keseluruhan.
  \item \textbf{Inefisiensi Blockchain} \newline
  Dengan sifat Blockchain sebagai ledger terdistribusi, setiap Smart Contract yang di-\textit{deploy} akan tetap tersimpan secara permanen. Redundansi kontrak ini memperbesar ukuran Blockchain, memperlambat akses data, dan menciptakan \textit{bottleneck} yang mengurangi efisiensi sistem secara keseluruhan. Masalah ini tidak hanya berdampak pada penyimpanan, tetapi juga memengaruhi kemampuan Blockchain untuk mendukung aplikasi berskala besar.
  \item \textbf{Kerentanan Keamanan} \newline
  Semakin banyak baris kode yang ditulis, akan semakin besar kemungkinan terdapat kerentanan. Dengan kurangnya prinsip \textit{reusability}, banyak kode yang ditulis kembali tetapi tidak terjamin secara kualitas, sehingga meningkatkan risiko kesalahan dan kerentanan keamanan yang mungkin terjadi.
\end{enumerate}

\subsubsection{Kebutuhan Utama}
\label{subsubsec:kebutuhan-utama}

Berdasarkan permasalahan utama dan dekomposisi masalah yang sudah diidentifikasi, muncul beberapa kebutuhan utama untuk menyelesaikannya, terutama yang berkaitan dengan pengelolaan dan penggunaan Smart Contracts. Berikut adalah beberapa kebutuhan yang perlu dijawab:

\begin{enumerate}
	\item \textbf{Kebutuhan Ekstraksi Data Smart Contracts (K1)} \newline
	      Sistem harus mampu mengekstraksi data Smart Contracts yang ada di blockchain untuk dapat diproses dan digunakan lebih lanjut.
	      % DK1

	\item \textbf{Kebutuhan Penyimpanan dan Pengelolaan Data Smart Contracts (K2)} \newline
	      Sistem harus dapat menyimpan dan mengelola data Smart Contracts yang telah diekstrak dan di-\textit{enrich}, sehingga data dapat diakses dan digunakan untuk pencarian lebih lanjut.
	      % DK2, DK4

	\item \textbf{Kebutuhan Klasifikasi Fungsional dan Semantik (K3)} \newline
	      Sistem harus dapat mengklasifikasikan Smart Contracts berdasarkan fungsionalitas dan semantik, dengan memperkaya data semantik (\textit{semantic enrichment}) untuk memudahkan pencarian kontrak yang relevan.
	      % DK3, DK5

	\item \textbf{Kebutuhan Solusi \textit{Off-Chain} (K4)} \newline
	      Sistem harus berjalan secara \textit{off-chain} untuk mengurangi beban pada jaringan blockchain dan meningkatkan skalabilitas.
	      % DK7

	\item \textbf{Kebutuhan Translasi Kebutuhan Pengguna ke Smart Contracts (K5)} \newline
	      Sistem harus dapat mentranslasikan kebutuhan fungsionalitas yang diajukan oleh pengguna menjadi kontrak yang relevan.
	      % DK5, DK6 

	\item \textbf{Kebutuhan Kemampuan Query Data (K6)} \newline
	      Data yang disimpan dalam sistem harus dapat di-query dengan mudah untuk memungkinkan pengambilan informasi yang relevan.
	      % DK6

	\item \textbf{Kebutuhan Peningkatan \textit{Reusability} (K7)} \newline
	      Sistem harus memudahkan penggunaan ulang Smart Contracts yang telah ada, dengan memastikan kemudahan dalam melakukan \textit{import} Smart Contract.
	      % DK8
\end{enumerate}

Kebutuhan-kebutuhan ini akan menjadi dasar dalam merancang solusi yang tepat untuk mengatasi permasalahan utama yang diidentifikasi sebelumnya. Dengan memenuhi kebutuhan-kebutuhan tersebut, diharapkan solusi yang diusulkan dapat memberikan manfaat yang maksimal bagi pengguna dalam mengelola dan menggunakan Smart Contracts di blockchain.

% Bisa dikasih visualisasi juga 

% \subsubsection{Ajuan Solusi}
\label{subsubsec:ajuan-solusi}

Berdasarkan kebutuhan-kebutuhan utama yang telah diidentifikasi, diperlukan sebuah solusi yang mampu menjawab berbagai tantangan dalam pengelolaan dan penggunaan Smart Contracts di blockchain. Solusi ini harus memungkinkan pencarian Smart Contracts berdasarkan fungsionalitasnya, sehingga pengembang tidak perlu membuat kontrak baru setiap kali mengembangkan dApps, terutama ketika kontrak dengan fungsionalitas serupa sudah tersedia. Dengan meningkatkan \textit{reusability} dari Smart Contracts yang ada, sistem ini dapat membantu menekan redundansi dan mengoptimalkan penggunaan sumber daya yang ada.

Untuk memenuhi kebutuhan ini, diajukan sebuah sistem pencarian Smart Contracts berbasis semantik yang dirancang untuk:

% rancangan solusi
% 

\begin{enumerate}
	\item Mengidentifikasi Smart Contracts dengan fungsionalitas yang sesuai berdasarkan kebutuhan pengembang. (AS1)
	\item Merekomendasikan Smart Contracts yang sudah ada dengan mempertimbangkan hubungan semantik antar kontrak, sehingga hasil pencarian lebih relevan dan dapat langsung digunakan. (AS2)
	\item Mendukung efisiensi biaya, waktu, dan pengelolaan blockchain, dengan memastikan bahwa proses pencarian dan pengelolaan dilakukan secara terstruktur dan tidak membebani jaringan blockchain. (AS3)
	      % tidak membebani dengan cara off-chain -> melakukan ekstraksi di luar blockchain, mapping di luar blockchain dst
	\item Merekomendasikan Smart Contracts yang aman dan terpercaya dengan mempertimbangkan verifikasi dari Smart Contracts. (AS4)
\end{enumerate}

Solusi ini dirancang untuk memenuhi kebutuhan \textit{reusability}, manajemen Smart Contracts, dan efisiensi sistem, sekaligus mendukung off-chain \textit{processing} agar ekosistem blockchain tetap skalabel dan efisien. Dengan pendekatan berbasis semantik ini, pengembang dApps dapat lebih mudah menemukan dan menggunakan Smart Contracts yang sesuai, tanpa harus menghabiskan waktu dan biaya untuk menciptakan kontrak baru.


\subsection{Analisis Alternatif Solusi}
\label{subsec:analisis-alternatif-solusi}

\subsubsection{Pendekatan Solusi}
Berdasarkan analisis masalah pada bagian \ref{subsec:analisis-masalah}, terdapat beberapa kebutuhan yang perlu dipenuhi dalam pengembangan solusi. Pemenuhan kebutuhan-kebutuhan ini akan menjadi konsiderasi untuk memilih alternatif solusi yang tepat. Altenatif-alternatif solusi akan dibagi menjadi beberapa kategori berdasarkan kebutuhan yang dipenuhinya, yaitu:

\begin{enumerate}
	\item Ekstraksi data Smart Contracts dari Blockchain Ethereum
	\item Pemodelan, penyimpanan, dan \textit{indexing} data Smart Contracts
	\item Klasifikasi fungsional dan semantik Smart Contracts
	\item Pencarian dan rekomendasi Smart Contracts berdasarkan kebutuhan pengguna dan pengembang
\end{enumerate}

Selain keempat alternatif tersebut, akan digunakan GUI dan API bagi pengguna untuk mengakses sistem.

Bagian berikut akan menguraikan alternatif yang memenuhi tiap kebutuhan. Alternatif yang disajikan berasal dari peninjauan sejumlah riset relevan, berfungsi sebagai referensi sekaligus fondasi pengembangan solusi. Peninjauan ini mempertimbangkan aspek seperti aksesibilitas hasil riset, kompleksitas teknis, skalabilitas, dan dukungan fungsional.

\subsubsection{Ekstraksi Data Smart Contracts dari Blockchain Ethereum}

\begin{figure}[ht]
	\centering
	\includegraphics[width=0.7\textwidth]{resources/chapter-3/ekstraksi-1.png}
	\caption{Perbandingan alternatif ekstraksi data Smart Contracts dari Blockchain Ethereum}
	\label{image:perbandingan-ekstraksi-1}
\end{figure}

\begin{figure}[ht]
	\centering
	\includegraphics[width=0.7\textwidth]{resources/chapter-3/ekstraksi-2.png}
	\caption{Perbandingan alternatif ekstraksi data Smart Contracts dari Blockchain Ethereum}
	\label{image:perbandingan-ekstraksi-2}
\end{figure}

\begin{figure}[ht]
	\centering
	\includegraphics[width=0.7\textwidth]{resources/chapter-3/ekstraksi-3.png}
	\caption{Perbandingan alternatif ekstraksi data Smart Contracts dari Blockchain Ethereum}
	\label{image:perbandingan-ekstraksi-3}
\end{figure}

Seluruh data Blockchain Ethereum dapat diakses melalui Ethereum node yang terhubung ke jaringan Ethereum. Node ini dapat berupa node lokal yang di-\textit{hosting} sendiri atau layanan node publik seperti Infura, Alchemy, atau QuickNode. Dukungan fungsional yang dikonsiderasi pada bagian ini bukan hanya dukungan untuk ekstraksi, tetapi kapabilitas untuk mendapatkan Source Code, dan memetakan Source Code dengan Deployments, yang dibutuhkan untuk melakukan analisis tanpa dekompilasi (sesuai batasan masalah) dan mendapatkan hasil pencarian yang dapat digunakan (pemetaan ke Deployments). Urutan prioritas untuk dukungan fungsional adalah kemampuan mengekstrak ke format yang dapat digunakan, kapabilitas mendapatkan source code, dan kapabilitas menyambungkan source code dengan deployments. Terdapat beberapa riset yang dapat digunakan sebagai referensi atau dasar untuk melakukan ekstraksi data Smart Contracts dari Blockchain Ethereum, di antaranya adalah:

\begin{enumerate}
	\item \textbf{eth2dgraph} \parencite{aimar2023extraction} (Bagian \ref{subsec:extraction-indexing-analysis-ethereum-sc}): Riset ini berfokus pada ekstraksi, \textit{indexing}, dan penyimpanan data Ethereum berbasis Distributed Graph. Keunggulannya adalah penggunaan ekstraksi ABI, bytecode, dan metadata yang dapat diubah menjadi format berbasis graf, serta implementasinya yang \textit{open source} dan \textit{public}. Untuk melakukan ekstraksi, dibutuhkan node Ethereum, baik yang dijalankan secara lokal maupun sebagai \textit{service}. Menggunakan Rust untuk kinerja tinggi dan Dgraph untuk skalabilitas, riset ini dapat melakukan query pada hubungan Smart Contracts di Ethereum. Kompleksitasnya moderat karena memerlukan pengetahuan dasar tentang Rust dan Dgraph, namun dapat diperluas untuk menambahkan aspek semantik. Skalabilitas dari hasil riset ini tinggi berkat kinerja Dgraph yang \textit{scalable}. Secara dukungan fungsional, selain cocok untuk melakukan ekstraksi dengan kecepatan tinggi, eth2dgraph juga memiliki kapabilitas untuk menyambungkan Smart Contracts Deployment dengan Verified Source Code.

	\item \textbf{Ethereum ETL} \parencite{ethereum_etl} (Bagian \ref{subsec:ethereum-etl}): Merupakan toolkit \textit{open source} berbasis Python untuk mengekstrak, mengonversi, dan memuat data blockchain Ethereum ke format yang mudah diolah (CSV, Parquet, BigQuery). Keunggulannya adalah kemudahan penggunaan dan dokumentasi yang baik, serta dukungan untuk berbagai jenis data seperti transaksi, blok, dan Smart Contracts. Secara kompleksitas, Ethereum ETL tergolong rendah karena menyediakan abstraksi layer ETL. Meskipun dukungan fungsional dari Ethereum ETL baik, terdapat beberapa kelemahan dari Ethereum ETL, yaitu tidak memiliki ekstraksi data ABI, membutuhkan waktu yang lama untuk melakukan ekstraksi data, diperlukannya beberapa operasi untuk mendapatkan data Smart Contracts, dan tidak memiliki kapabilitas untuk mendapatkan Source Code tanpa dekompilasi. Secara skalabilitas, Ethereum ETL dapat menangani volume data besar dengan baik, namun mengeluarkan format data yang hanya cocok untuk basis data relasional, sehingga tidak cocok untuk sistem big data.

	\item \textbf{Etherscan API} \parencite{etherscan2024} (Bagian \ref{subsec:etherscan}): Etherscan adalah salah satu penjelajah blok Ethereum yang paling populer. Menyediakan API yang memungkinkan pengguna untuk mengakses data Smart Contracts, transaksi, dan informasi lainnya dari Blockchain Ethereum. API ini dapat digunakan untuk mengekstrak data Smart Contracts beserta Verified Source Code-nya secara langsung dari Etherscan, yang merupakan sumber data yang terpercaya dan terupdate. Secara aksesibilitas, Etherscan API mudah digunakan. Namun, data yang diekstrak dan di-\textit{indexing} berasal sepenuhnya dari Etherscan, sehingga pengguna perlu mempercayai Etherscan sebagai sumber data, dengan kata lain, sumber data Etherscan tersentralisasi. Selain itu, Etherscan API juga memiliki batasan pada jumlah permintaan dan memiliki biaya penggunaan yang dapat menjadi kendala untuk ekstraksi data dengan skala besar. Secara kompleksitas, Etherscan API tergolong rendah karena menyediakan antarmuka yang sederhana untuk mengakses data. Namun, pengguna perlu memahami cara menggunakan API dan mengelola batasan permintaan.

	\item \textbf{XBlock-ETH} \parencite{zheng2020xblock} (Bagian \ref{subsec:xblock-eth}): XBlock-ETH adalah alat yang dirancang untuk mengekstrak data dari Blockchain Ethereum kepada format CSV. Selain itu, XBlock-ETH juga merilis dataset yang sudah diekstrak pada website XBlock (\url{https://xblock.pro/xblock-eth.html}). XBlock-ETH memiliki keunggulan untuk mendapatkan data tanpa menggunakan Ethereum node, tetapi data yang disimpan dalam bentuk CSV perlu dilakukan \textit{parsing} dan tidak mudah dilakukan query atau \textit{indexing}. Selain itu, XBlock-ETH juga tidak mudah diakses karena kode untuk melakukan ekstraksi data tidak bersifat \textit{open source}, sehingga tidak dapat melakukan replikasi ekstraksi data. Secara kompleksitas, XBlock-ETH tergolong rendah karena tidak memerlukan pengetahuan teknis yang mendalam untuk menggunakannya. Namun, pengguna perlu memahami cara melakukan \textit{parsing} data CSV yang dihasilkan. Dalam hal skalabilitas, XBlock-ETH dapat menangani volume data besar dengan baik, tetapi tidak cocok untuk sistem big data karena format CSV yang dihasilkan tidak efisien untuk penyimpanan dan pemrosesan data besar.

	\item \textbf{The Graph Protocol} \parencite{TheGraphDocs} (Bagian \ref{subsec:the-graph-protocol}): The Graph adalah protokol dan Network untuk membangun dan mengelola indeks data terdesentralisasi dari Blockchain Ethereum. Protokol ini memungkinkan pengguna untuk membuat indeks data yang dapat di-query dengan efisien menggunakan GraphQL. Keunggulannya adalah kemudahan penggunaan dengan kemampuan untuk melakukan query dengan cepat. Secara aksesibilitas, The Graph dapat dengan mudah digunakan dan terintegrasi dengan baik, tetapi perlu pembayaran sesuai dengan penggunaan (\textit{pay-as-you-go}), yang membuat ekstraksi data lebih mahal. Secara kompleksitas, The Graph tergolong rendah karena dokumentasi ekstensif dan juga infrastruktur yang baik. Secara dukungan fungsional, The Graph menghasilkan data dalam bentuk json, yang perlu dilakukan impor ulang ke format yang dipilih, yang tidak mudah dan membutuhkan waktu lama, tetapi mudah di-\textit{extend}.

	\item \textbf{Sourcify} \parencite{sourcify_website} (Bagian \ref{subsec:sourcify}): Sourcify adalah layanan \textit{open source} yang memungkinkan pengguna untuk memverifikasi dan menyimpan Source Code Smart Contracts di Ethereum. Sourcify menyediakan API yang memungkinkan pengguna untuk mengakses data Smart Contracts dan Source Code-nya yang telah diverifikasi. Secara aksesibilitas, Sourcify mudah digunakan karena sudah disediakan abstraksi untuk mengunduh data. Secara data, Sourcify sudah menyediakan pemetaan data antar Source Code dan Deployments. Namun, Sourcify tidak memiliki kemampuan ekstraksi data yang lebih kompleks seperti ABI dan bytecode. Data Sourcify diupdate setiap harinya, tetapi dibutuhkan proses mengunduh yang terus berulang untuk memperbaharui data, yang perlu dijadikan lebih efisien. Secara kompleksitas, Sourcify tergolong moderat karena walaupun sudah tersedia abstraksi untuk mengunduh data, tetap diperlukan proses untuk menjadikan prosesnya lebih efisien.
\end{enumerate}

Secara singkat, keenam riset ini dapat disimpulkan dengan diagram-diagram pada gambar \ref{image:perbandingan-ekstraksi-1}, gambar \ref{image:perbandingan-ekstraksi-2}, dan gambar \ref{image:perbandingan-ekstraksi-3}.

\newpage

% Setelah melakukan analisis dari alternatif yang ada, diputuskan untuk menggunakan riset oleh \cite{aimar2023extraction}, karena memiliki implementasi yang \textit{open source}, yang mempermudah ekstraksi dan \textit{indexing} data menjadi Distributed Graph Database, yang memiliki skalabilitas yang baik untuk data yang banyak pada Blockchain Ethereum. eth2dgraph juga memiliki kemampuan ekstensibilitas yang baik dalam \textit{domain} yang lebih umum, sehingga lebih mudah diimplementasikan sebagai fondasi dari sistem keseluruhan.

\subsubsection{Pemodelan, Penyimpanan, dan \textit{Indexing} Data Smart Contracts}

% jadi bahas alternatif dulu, misal yang terpilih eth2dgraph
% lalu bahas schema yang dipakainya gimana, yang base nya apa aja secara singkat, dan yang mau ditambahinnya apa, berdasarkan apa
% Jadiin dua subheading

\begin{figure}[ht]
	\centering
	\includegraphics[width=0.9\textwidth]{resources/chapter-3/pemodelan-1.png}
	\caption{Perbandingan alternatif pemodelan, penyimpanan, dan \textit{indexing} data Smart Contracts}
	\label{image:pemodelan-1}
\end{figure}

\begin{figure}[ht]
	\centering
	\includegraphics[width=0.9\textwidth]{resources/chapter-3/pemodelan-2.png}
	\caption{Perbandingan alternatif pemodelan, penyimpanan, dan \textit{indexing} data Smart Contracts}
	\label{image:pemodelan-2}
\end{figure}

Data yang diekstrak dari Blockchain Ethereum perlu dimodelkan, disimpan, dan dilakukan \textit{indexing} agar dapat diakses dengan efisien. Secara fungsional, untuk pemodelan, dibutuhkan ekstensibilitas yang baik untuk model, sehingga dapat menampung data yang lebih banyak dan lebih kompleks. Untuk penyimpanan, dibutuhkan skalabilitas yang baik untuk menyimpan data yang besar, serta dukungan untuk melakukan query dengan efisien. Untuk \textit{indexing}, dibutuhkan kemampuan untuk melakukan query dengan cepat dan efisien.

Gambar \ref{image:pemodelan-1} dan \ref{image:pemodelan-2} menunjukkan rangkuman perbandingan berbagai alternatif pemodelan, penyimpanan, dan \textit{indexing} data Smart Contracts. Secara rinci, berikut adalah analisis dari masing-masing alternatif:

\begin{enumerate}
	\item \textbf{eth2dgraph (Dgraph)} \parencite{aimar2023extraction} (Bagian \ref{subsec:extraction-indexing-analysis-ethereum-sc}): Riset ini tidak hanya mengekstrak data, tetapi juga menyediakan solusi terpadu untuk pemodelan, indexing, dan penyimpanan dengan memanfaatkan Dgraph. Keunggulannya terletak pada skalabilitas tinggi dan efisiensi query, serta kemudahan akses dan eksekusi secara lokal. Meskipun memerlukan pemahaman dasar mengenai \textit{node} dan pemodelan graf, dokumentasi yang lengkap dan komunitas aktif membuat teknologi ini lebih mudah diaplikasikan. Format Dgraph juga mudah di-\textit{extend} untuk menambahkan aspek semantik yang mendukung pencarian Smart Contracts.

	\item \textbf{Linked Data Indexing of Distributed Ledgers (Linked Data)} \parencite{third2017linked} (Bagian \ref{subsec:linked-data-indexing-distributed-ledgers}): Pendekatan riset ini bersifat publik namun tidak menyediakan implementasi open source. Metodenya membutuhkan pemetaan ontologi secara ekstensif dan generasi RDF triple yang kompleks, dengan dukungan tools atau framework yang minim. Selain itu, skalabilitasnya terbatas karena bergantung pada Linked Data berbasis RDF, meskipun format tersebut dapat dengan mudah di-\textit{extend} untuk menambahkan aspek semantik guna mendukung pencarian Smart Contracts.

	\item \textbf{JSON}: JSON menawarkan kemudahan penggunaan dan dukungan luas di berbagai bahasa pemrograman. Namun, meskipun sederhana dan mudah dikembangkan, JSON tidak mendukung indexing secara efisien untuk data besar, sehingga kurang ideal untuk sistem yang perlu menangani volume data tinggi.

	\item \textbf{Relational Database (SQL)}: Sistem basis data relasional dengan SQL unggul dalam indexing dan query yang efisien untuk data besar. Walaupun demikian, model relasional ini memiliki keterbatasan dalam fleksibilitas dan skalabilitas untuk data yang tidak terstruktur atau semi-terstruktur. Dengan pengetahuan dasar mengenai basis data relasional dan SQL, sistem ini dapat di-\textit{extend} melalui penambahan aspek semantik untuk memudahkan pencarian Smart Contracts.
\end{enumerate}

\subsubsection{Klasifikasi Fungsional dan Semantik Smart Contracts}

\begin{figure}[ht]
	\centering
	\includegraphics[width=0.7\textwidth]{resources/chapter-3/klasifikasi - 1.png}
	\caption{Perbandingan alternatif klasifikasi fungsional dan semantik Smart Contracts}
    \label{image:klasifikasi-1}
\end{figure}

\begin{figure}[ht]
	\centering
	\includegraphics[width=0.7\textwidth]{resources/chapter-3/klasifikasi - 2.png}
	\caption{Perbandingan alternatif klasifikasi fungsional dan semantik Smart Contracts}
    \label{image:klasifikasi-2}
\end{figure}

\begin{figure}[ht]
	\centering
	\includegraphics[width=0.7\textwidth]{resources/chapter-3/klasifikasi - 3.png}
	\caption{Perbandingan alternatif klasifikasi fungsional dan semantik Smart Contracts}
    \label{image:klasifikasi-3}
\end{figure}

\begin{figure}[ht]
	\centering
	\includegraphics[width=0.7\textwidth]{resources/chapter-3/klasifikasi - 4.png}
	\caption{Perbandingan alternatif klasifikasi fungsional dan semantik Smart Contracts}
    \label{image:klasifikasi-4}
\end{figure}

Data yang sudah dimodelkan dan disimpan ke dalam sistem perlu dilakukan klasifikasi untuk memudahkan pencarian Smart Contracts berdasarkan fungsionalitas dan semantik. Terdapat berbagai alternatif untuk melakukan klasifikasi fungsional dan semantik Smart Contracts dengan berbagai pendekatan, baik dengan pemodelan ontologi, deskripsi fungsional, maupun format lainnya. Dukungan fungsional yang baik untuk klasifikasi adalah riset yang dapat melakukan klasifikasi pada data Smart Contracts dengan baik dan memberikan deskripsi atau hasil klasifikasi yang baik. Pada bagian ini, aspek skalabilitas digantikan dengan aspek \textit{feasibility}. Berikut adalah beberapa alternatif yang dapat digunakan untuk klasifikasi fungsional dan semantik Smart Contracts:

\begin{enumerate}
    \item \textbf{Semantic Smart Contracts for Blockchain-based Services in the Internet of Things} \parencite{baqa2019semantic} (Bagian \ref{subsec:semantic-smart-contract-iot}): Riset ini menggunakan ekstensi pada OWL-S Service Ontology untuk melakukan klasifikasi Smart Contracts berdasarkan semantik dan fungsionalitas. Keunggulannya adalah penggunaan ontologi yang dapat di-\textit{extend} untuk menambahkan terminologi yang \textit{domain specific}. Secara aksesibilitas, riset ini bersifat \textit{public}, namun tidak memiliki implementasi \textit{open source}. Selain itu, riset ini terbatas pada \textit{scope} Internet-of-Things. Secara kompleksitas, riset ini tergolong tinggi karena memerlukan pemahaman yang mendalam tentang ontologi dan pemodelan semantik. Dalam hal dukungan fungsional, riset ini dapat memberikan deskripsi yang baik, tetapi tidak terdapat mekanisme untuk mengklasifikasikan data dengan baik.
    
    \item \textbf{Ontological Modeling of Smart Contracts in Solidity} \parencite{cano2021toward} (Bagian \ref{subsec:solidity-ontology}): Riset ini menerapkan ontologi pada bahasa pemrograman Solidity. Ontologi ini digunakan untuk mendeskripsikan elemen-elemen dalam Smart Contracts, seperti fungsi, variabel, dan struktur data. Secara aksesibilitas, riset ini bersifat publik, dengan ontologi yang dihasilkan dapat diterapkan. Namun, implementasi ontologi pada riset ini terbatas pada sintaks dan semantik dari kode bahasa pemrograman Solidity sendiri dibandingkan Smart Contracts secara keseluruhan. Secara kompleksitas, riset ini tergolong tinggi karena memerlukan pemahaman yang mendalam tentang ontologi dan pemodelan semantik, dan memerlukan proses klasifikasi yang ekstensif untuk memetakan data dengan ontologi yang dihasilkan. Dalam hal dukungan fungsional, riset ini tidak sesuai dengan kebutuhan sistem, yaitu memodelkan fungsionalitas dari Smart Contract, bukan aspek sintaks bahasa pemrograman Soliditynya, dan tidak terdapat mekanisme untuk mengklasifikasikan data dengan baik.
    
    \item \textbf{STAN} \parencite{stan} (Bagian \ref{subsec:stan}): STAN adalah sebuah sistem untuk memberikan deskripsi terhadap bytecodes dari Smart Contracts. Secara aksesibilitas, STAN tidak bersifat \textit{open source}, sehingga tidak tersedia implementasinya untuk melakukan replikasi. Secara kompleksitas, STAN tergolong rendah karena abstraksi yang sudah diberikan untuk menghasilkan deskripsi. Riset STAN ini juga belum menginkorporasikan teknologi seperti Artificial Intelligence (AI) untuk melakukan klasifikasi. Dalam hal dukungan fungsional, STAN dapat membantu mengklasifikasikan Smart Contracts berdasarkan deskripsi yang dihasilkan.
    
    \item \textbf{Agents} (Bagian \ref{sec:agents}): Alternatif untuk melakukan klasifikasi Smart Contracts adalah menggunakan AI Agents yang dapat melakukan dekomposisi tasks yang kompleks menjadi sub-tasks yang lebih sederhana. Dengan menggunakan AI Agents, klasifikasi Smart Contracts dapat dilakukan dengan lebih menyeluruh dan efisien karena dapat memperhitungkan berbagai aspek yang ada pada Smart Contracts. Secara aksesibilitas, sudah banyak \textit{framework} dan \textit{tools} yang tersedia untuk membangun sebuah sistem berbasis AI Agents (Agentic AI). Namun, perlu dilakukan pembangunan secara independen karena tidak ada sistem yang secara langsung memberikan fungsionalitas yang sesuai. Secara kompleksitas, penggunaan AI Agents tergolong moderat karena memerlukan pengetahuan terkait AI dan membuat sistem berbasis AI Agents. Secara \textit{feasibility}, pembangunan sistem berbasis agents cukup \textit{feasible} dengan penggunaan \textit{framework} dan \textit{tools} yang baik. Dukungan fungsional yang diberikan oleh sistem dengan AI Agents tergolong baik karena dapat disesuaikan dan melakukan pekerjaan kompleks secara otonom, tetapi perlu menggunakan format yang disesuaikan untuk mendeskripsikan data.
    % menggunakan agents yang dimasukkin source code, dengan break down step by step

    \item \textbf{Uniform Description Language for Smart Contracts} \parencite{udlsc} (Bagian \ref{subsec:uniform-description-language}): Riset ini mengusulkan sebuah bahasa deskripsi ekstensi dari USDL untuk Smart Contracts. Secara aksesibilitas, hasil dari riset ini bersifat \textit{public}, namun perlu melakukan replikasi untuk mendapatkan hasil yang didapatkan dari riset. Secara kompleksitas, riset ini tergolong moderat karena walaupun tidak memerlukan pengetahuan yang mendalam terkait ontologi, perlu memahami terkait USDL dan implementasi ekstensi dari USDL. Secara skalabilitas, riset ini tergolong baik karena dapat digunakan sebagai ekstensi data tanpa masalah. Dalam hal dukungan fungsional, riset ini kurang baik karena tidak terdapat mekanisme untuk mengklasifikasikan data dengan baik, tetapi dapat digunakan untuk mendeskripsikan data Smart Contracts dengan baik.
    
    \item \textbf{Service Oriented Format Descriptor} \parencite{guida2019supporting} (Bagian \ref{subsec:supporting-reuse-smart-contracts}): Riset ini mengusulkan sebuah format deskripsi untuk Smart Contracts dengan pendekatan Service. Riset ini juga mengusulkan sebuah Service Registry dan Contract Editor berbasis visual yang mengkomplemen format deskripsi yang diusulkan. Format deskripsi ini dapat digunakan dan digabungkan dengan sistem lain, sedangkan Service Registry dan Contract Editor kurang fleksibel untuk diintegrasikan dengan sistem lain. Secara aksesibilitas, hasil dari riset ini bersifat \textit{public} dan \textit{open source}, sehingga dapat digunakan untuk melakukan replikasi. Secara kompleksitas, format yang dihasilkan oleh riset ini tergolong rendah karena tidak memerlukan pengetahuan yang mendalam dan dapat langsung dijadikan ekstensi ke format lain. Dalam hal dukungan fungsional, riset ini dapat mengakomodasi deskripsi Smart Contracts dengan baik, tetapi tidak ada mekanisme untuk melakukan klasifikasi data dengan baik.
    
    \item \textbf{Smart Contract Summarizer} \parencite{zhang2021smart} (Bagian \ref{subsec:smart-contract-solidity-summary}): Riset ini mengusulkan sebuah sistem untuk menghasilkan ringkasan dan anotasi dari Smart Contracts, terutama dalam bahasa Solidity. Sistem ini menggunakan teknik NLG (Natural Language Generation) dengan pendekatan berbasis \textit{transformer} untuk menghasilkan ringkasan yang lebih baik dibandingkan dengan metode berbasis template sebelumnya. Secara aksesibilitas, konsep dari riset ini bersifat \textit{public}, namun tidak memiliki implementasi \textit{open source}. Secara kompleksitas, jika perlu mengimplementasikan dari awal, riset ini tergolong tinggi karena memerlukan pengetahuan yang mendalam terkait NLP dan pemodelan semantik. Secara skalabilitas, sistem ini tidak diketahui untuk kinerja menangani data yang banyak. Dalam hal dukungan fungsional, riset ini dapat membantu mengklasifikasikan Smart Contracts berdasarkan ringkasan yang dihasilkan dengan mekanisme yang digunakan.

\end{enumerate}

Secara singkat, ketujuh alternatif ini dapat disimpulkan dengan diagram-diagram pada gambar \ref{image:klasifikasi-1}, gambar \ref{image:klasifikasi-2}, gambar \ref{image:klasifikasi-3}, dan gambar \ref{image:klasifikasi-4}.


\subsubsection{Pencarian dan Rekomendasi Smart Contracts}

\begin{figure}[ht]
	\centering
	\includegraphics[width=0.7\textwidth]{resources/chapter-3/pencarian-1.png}
	\caption{Perbandingan alternatif pencarian dan rekomendasi Smart Contracts}
	\label{image:pencarian-1}
\end{figure}

\begin{figure}[ht]
	\centering
	\includegraphics[width=0.7\textwidth]{resources/chapter-3/pencarian-2.png}
	\caption{Perbandingan alternatif pencarian dan rekomendasi Smart Contracts}
	\label{image:pencarian-2}
\end{figure}

Setelah data diklasifikasikan, data akan disimpan dengan format yang sesuai untuk memudahkan pencarian dan rekomendasi. Aspek skalabilitas dalam bagian ini akan digantikan dengan aspek \textit{feasibility}. Terdapat beberapa metode yang dapat digunakan untuk melakukan pencarian dan rekomendasi Smart Contracts, antara lain:

\begin{enumerate}
	\item \textbf{Retrieval-Augmented Generation (RAG)} (Bagian \ref{sec:rag}): RAG adalah sebuah metode pipeline AI untuk menambahkan kemampuan pencarian dengan menggunakan sebuah \textit{retriever} untuk mendapatkan data dari sumber data eksternal. Secara aksesibilitas, terdapat \textit{framework} dan \textit{tools} yang dapat digunakan untuk membangun sistem berbasis RAG. Secara kompleksitas, RAG tergolong moderat karena walau membutuhkan pengetahuan terkait RAG, implementasi dari RAG dapat dimudahkan dengan infrastruktur yang sudah ada. Secara \textit{feasibility}, sistem ini dapat diimplementasikan sesuai skala yang diinginkan. Secara dukungan fungsional, RAG dapat digunakan untuk melakukan pencarian dan rekomendasi Smart Contracts lalu melakukan generasi jawaban dengan baik, walau terdapat redundansi layer generasi jika sistem hanya digunakan untuk pencarian.

	\item \textbf{Multimodal Smart Contract Search} \parencite{shi2021semantic} (Bagian \ref{subsec:semantic-code-search}): Riset ini mengusulkan sebuah sistem pencarian Smart Contracts berbasis multimodal, yang tidak hanya menggunakan teks, tetapi juga menggunakan alur kontrol (Contract Elements Dependency Graph). Meskipun hasil dari riset ini baik, sistem yang dihasilkan tidak bersifat \textit{open source} sehingga tidak dapat digunakan untuk membangun sistem di atasnya. Secara kompleksitas, untuk mereplikasi sistem ini membutuhkan kemampuan teknis yang tinggi karena dibutuhkan pemahaman dan keterampilan untuk membangun komponen-komponen di dalam \textit{framework} MM-SCS seperti modul GAT (Graph Attention Network) yang menganalisa CEDG. Secara dukungan fungsional, sistem tidak dapat di-\textit{extend}, walaupun sudah mengakomodasi untuk pendekatan semantik berbasis eksekusi kode.

	\item \textbf{Vector Embedding Search}: Vector Embedding Search adalah metode pencarian yang menggunakan representasi vektor sebagai embedding dari data yang akan dicari. Cara kerjanya adalah dengan membuat embedding dari query dan mencari kemiripan antar vektor dari query dengan data. Secara aksesibilitas, terdapat Vector Database yang dapat mengakomodasi pencarian ini. Secara kompleksitas, Vector Embedding Search tergolong rendah ke moderat karena hanya membutuhkan pengetahuan dasar tentang Vector Embeddings dan Vector Database. Secara \textit{feasibility}, tergolong tinggi karena hanya perlu mengonversi data ke dalam bentuk embeddings dan menyimpannya di dalam Vector Database. Secara dukungan fungsional, Vector Embedding Search dapat digunakan untuk melakukan pencarian dan rekomendasi Smart Contracts dengan baik dengan implementasi dan fungsionalitas sederhana.
\end{enumerate}

Secara singkat, ketiga alternatif ini dapat disimpulkan dengan diagram-diagram pada gambar \ref{image:pencarian-1} dan gambar \ref{image:pencarian-2}.

% \subsubsection{Interaksi Pengguna dengan Sistem}

% Setelah mekanisme pencarian dan rekomendasi Smart Contracts selesai, diperlukan cara untuk pengguna mengakses dan berinteraksi dengan sistem. Terdapat dua alternatif utama untuk interaksi pengguna dengan sistem, yaitu melalui \textit{Graphical User Interface} dan \textit{Application Programming Interface}.

\subsubsection{Kesimpulan Pemilihan Alternatif}

\begin{figure}[ht]
	\centering
	\includegraphics[width=0.7\textwidth]{resources/chapter-3/hasil-pemilihan.png}
    \caption{Kesimpulan pemilihan alternatif}
    \label{image:layer-architecture}
\end{figure}

Berdasarkan analisis dari alternatif-alternatif yang dibahas pada bagian \ref{subsec:analisis-alternatif-solusi}, dapat disimpulkan bahwa alternatif yang dipilih untuk membangun sistem adalah sebagai berikut:

\begin{enumerate}
    \item \textbf{Ekstraksi data Smart Contracts dari Blockchain Ethereum}: Alternatif yang dipilih adalah \textit{eth2dgraph} \parencite{aimar2023extraction}, karena memiliki aksesibilitas yang baik, bersifat \textit{open source}, memiliki kecepatan ekstraksi yang tinggi, dan juga menyediakan infrastruktur lengkap sampai pada penyimpanan data dalam format Distributed Graph Database.
    \item \textbf{Pemodelan, penyimpanan, dan \textit{indexing} data Smart Contracts}: Alternatif yang dipilih adalah \textit{eth2dgraph} \parencite{aimar2023extraction}, karena memiliki kemampuan skalabilitas tinggi dan dapat melakukan query dengan efisien. Selain itu, Dgraph juga memiliki kemampuan untuk melakukan \textit{indexing} data dengan baik.
    \item \textbf{Klasifikasi fungsional dan semantik Smart Contracts}: Alternatif yang dipilih untuk mekanisme klasifikasi adalah Agents, karena memiliki kemampuan kustomisasi yang baik untuk klasifikasi, sedangkan model semantik yang akan digunakan adalah gabungan dari model-model yang dituliskan seperti UDL-SC, Service Oriented Format, dan ontologi (ekstensi OWL-S dan Solidity Ontology). Pendekatan yang akan dilakukan adalah pendekatan \textit{incremental}, dimana model akan digunakan model paling sederhana di awal, lalu ditambahkan atribut-atribut lainnya untuk melakukan pengujian dan evaluasi.
    \item \textbf{Pencarian dan rekomendasi Smart Contracts}: Alternatif yang dipilih untuk mekanisme pencarian awal adalah alternatif Vector Embedding Search, karena simplisitas yang ditawarkan dan tidak ada redundansi \textit{layer}. Alternatif yang dapat dikonsiderasikan untuk pengembangan berikutnya, terutama jika dikembangkan sebuah fitur untuk berinteraksi dengan sistem yang lebih kompleks adalah alternatif \textit{Retrieval-Augmented Generation (RAG)}, karena dapat mengakomodasi interaksi yang lebih kompleks. Untuk menambahkan kapabilitas dari Vector Embedding Search, dapat digunakan aspek \textit{enrichment} yang dilakukan oleh AI Agents terhadap query.
    \item \textbf{Interaksi pengguna dengan sistem}: Seluruh alternatif yang diajukan, yaitu API dan GUI akan diimplementasikan untuk interaksi pengguna dengan sistem karena dapat memberikan fleksibilitas dan kemudahan bagi pengguna. Sehingga, pengguna dapat melakukan pencarian Smart Contracts dengan cara yang sesuai dengan kebutuhan mereka.
\end{enumerate}

% Untuk mengatasi permasalahan pemilihan Smart Contracts yang tepat dan mengurangi redundansi Smart Contracts di Blockchain, solusi yang diusulkan adalah sebuah sistem pencarian Smart Contracts yang dapat memberikan hasil berdasarkan fungsionalitas Smart Contracts. Sistem akan dibangun dengan memanfaatkan berbagai teknologi dan riset yang sudah ada, yang melakukan \textit{indexing} maupun modeling yang menjadikan Smart Contracts \textit{discoverable} untuk mengefisiensikan pengembangan.

% Beberapa riset yang dilakukan peninjauan untuk digunakan sebagai basis adalah riset oleh \cite{third2017linked}, \cite{aimar2023extraction}, \cite{baqa2019semantic}, \cite{cano2021toward}. Peninjauan didasari dengan beberapa aspek yaitu aksesibilitas dari hasil riset, kompleksitas teknis, skalabilitas, dan dukungan fungsional untuk mencapai tujuan utama.

% % Masukin diagram yang dibuat di ppt

% % preliminary analysis

% % gambaran solusi

% % menjelaskan secara lebih detail latar belakang dan masalah yang menjadi dasar munculnya topik TA ini, intinya kita coba lihat & analisis gapnya 
% % gap analysis
% % kaitan antara sistem yang dikembangkan dengan yang terkait -> apa kelebihannya? atau apa kekurangan dari aplikasi lain? emang belum terpenuhi? apa yang belum terpenuhi?
% % posisi sistem yang dikembangkan terhadap sistem yang lebih besar

% % PLACEHOLDER
% \subsubsection{Semantic Indexing with Linked Data \parencite{third2017linked}}

% Riset ini menerapkan indeks semantik pada data Blockchain menggunakan Linked Data dengan keunggulan penggunaan ontology BLONDiE dan MSM untuk mendeskripsikan semantik Smart Contracts dan fokus pada aspek \textit{discoverability}. Secara aksesibilitas, konsep riset ini \textit{public}, namun tanpa implementasi \textit{open source}. Implementasinya kompleks karena memerlukan pemetaan ontology ekstensif dan RDF triple generation, tanpa dukungan \textit{tools} atau \textit{framework}. Skalabilitas riset ini terbatas karena bergantung pada RDF-based Linked Data, yang kurang cocok untuk data Blockchain besar.

% \subsubsection{eth2dgraph \parencite{aimar2023extraction}}

% Riset ini berfokus pada ekstraksi, \textit{indexing}, dan penyimpanan data Ethereum berbasis Distributed Graph. Keunggulannya adalah penggunaan ekstraksi ABI, bytecode, dan metadata yang dapat diubah menjadi format berbasis graf, serta implementasinya yang \textit{open source} dan \textit{public}. Menggunakan Rust untuk performa tinggi dan Dgraph untuk skalabilitas, riset ini dapat melakukan query pada hubungan Smart Contracts di Ethereum. Kompleksitasnya moderat karena memerlukan pengetahuan dasar tentang Rust dan Dgraph, namun dapat diperluas untuk menambahkan aspek semantik. Skalabilitasnya tinggi berkat kinerja Dgraph.

% \subsubsection{Alternatif Lainnya}

% Kedua riset alternatif lainnya oleh \cite{baqa2019semantic} dan \cite{cano2021toward} tidak dapat dipilih karena \textit{domain} yang terlalu spesifik, ditambah dengan implementasi yang tidak bersifat \textit{open source} dan \textit{public}.

% \subsubsection{Hasil Analisis}

% Setelah melakukan analisis dari alternatif yang ada, diputuskan untuk menggunakan riset oleh \cite{aimar2023extraction}, karena memiliki implementasi yang \textit{open source}, yang mempermudah ekstraksi dan \textit{indexing} data menjadi Graph Database, sehingga tidak perlu membuat RDF Triples ada model ontology dari awal. Distributed Graph Database juga memiliki skalabilitas yang baik untuk data yang banyak pada Blockchain Ethereum. eth2dgraph juga memiliki kemampuan ekstensibilitas yang baik dalam \textit{domain} yang lebih umum, sehingga lebih mudah diimplementasikan sebagai fondasi dari sistem keseluruhan. 

% \subsubsection{Rancangan Solusi}

% Dengan penggunaan eth2dgraph sebagai fondasi dari sistem pencarian Smart Contract, berikut merupakan ajuan rancangan dari sistem:

% \begin{enumerate}
%   \item Layer 1: Blockchain Data Extraction (eth2dgraph) \newline Ekstraksi data Blockchain Ethereum menjadi Dgraph
%   \item Layer 2: Semantic Indexing and Enrichment \newline \textit{Mapping} data hasil ekstraksi kepada sebuah ontology seperti BLONDiE atau EthOn, pelabelan fungsional Smart Contracts, dan Version Control
%   \item Layer 3: Query and Discovery System \newline Sebuah Search Engine menggunakan GraphQL Queries diatas Dgraph Database yang memperkenalkan pencarian berbasis semantik
%   \item Layer 4: User Interaction Layer \newline Sebuah \textit{dashboard} atau API untuk pengembang melakukan pencarian Smart Contracts berdasarkan fungsionalitas, metadata, atau relasi, membandingkan Smart Contracts yang serupa, dan melakukan \textit{export} atau \textit{reuse} dari Smart Contract 
% \end{enumerate}


\subsection{Analisis Kebutuhan Sistem}

% bingung nulis apa lagi disini
Pada bagian \ref{subsec:analisis-alternatif-solusi}, telah disimpulkan pilihan alternatif yang akan digunakan dalam membangun sistem. Alternatif-alternatif solusi yang digunakan akan menjadi komponen yang saling berinteraksi dalam sistem Smart Contract Discovery untuk menyediakan fungsionalitas yang terpadu. Pada bagian ini akan diuraikan lebih lanjut mengenai sistem yang akan dibangun sehingga dapat memberikan panduan dalam fase pengembangan sistem. Penjelasan ini mencakup deskripsi sistem, karakteristik pengguna, kebutuhan fungsional dan non-fungsional, serta model use case yang akan digunakan dalam sistem.

\subsubsection{Deskripsi Sistem}
% penjelasan gambaran umum sistem, komponen utamanya apa aja
% buat diagram UML gambaran umum
% Jelaskan tujuan utama sistem (misalnya: "Membangun sistem pencarian smart contract berbasis semantik untuk meningkatkan efisiensi pengembangan dApps").

Solusi yang akan dikembangkan adalah sebuah sistem Smart Contract Discovery yang bertujuan untuk menyediakan \textit{platform} pencarian Smart Contract dalam Blockchain Ethereum berbasis semantik memanfaatkan LLM dan RAG. Sistem akan dibagi menjadi beberapa komponen utama yang saling berinteraksi untuk menyediakan fungsionalitas yang terpadu. Komponen utama sistem adalah sebagai berikut:

\begin{enumerate}
  \item Komponen Ekstraksi Data
  \item Komponen Penyimpanan Data
  \item Komponen \textit{Semantic Enrichment}
  \item Komponen Pencarian
  \item Komponen Antarmuka Pengguna
\end{enumerate}

\begin{figure}[ht]
	\centering
	\includegraphics[width=1\textwidth]{resources/chapter-3/komponen-utama-new.png}
	\caption{Gambaran Umum Interaksi Komponen Utama Sistem}
	\label{image:komponen-sistem}
\end{figure}

\begin{figure}[ht]
	\centering
	\includegraphics[width=0.7\textwidth]{resources/chapter-3/layer-arsitektur-new.png}
	\caption{Gambaran Arsitektur Layer Sistem}
	\label{image:layer-arsitektur}
\end{figure}

% Ethereum Archive Node → eth2dgraph (ekstraksi) → Dgraph (penyimpanan) → LLM (semantic enrichment) → Dgraph (update) → RAG (query).  

Gambar \ref{image:komponen-sistem} dan gambar \ref{image:layer-arsitektur} menunjukkan gambaran umum alur kerja sistem. Sistem ini akan melakukan ekstraksi data dari Ethereum Archive Node menggunakan \textit{eth2dgraph} dan menyimpannya dalam Dgraph. Setelah itu, sistem akan melakukan \textit{semantic enrichment} menggunakan LLM untuk memperkaya deskripsi dan metadata Smart Contract, lalu memperbarui data di Dgraph. Sebelum data dapat dilakukan query menggunakan RAG, akan dilakukan indexing data menjadi bentuk vectorstore. Terakhir, sistem akan menggunakan RAG untuk melakukan pencarian berdasarkan kebutuhan pengguna. 



\subsubsection{Karakteristik Pengguna}

Sistem ini hanya akan digunakan oleh satu jenis pengguna, yaitu User yang ingin mencari Smart Contract. Belum ada mekanisme yang membutuhkan campur tangan pengguna lain, seperti pengembang atau administrator. Pengguna dapat melakukan pencarian Smart Contract berdasarkan kebutuhan fungsionalitas yang diinginkan. Pengguna tidak perlu memiliki pengetahuan teknis yang mendalam tentang Smart Contract atau Blockchain untuk menggunakan sistem ini. Antarmuka pengguna dirancang agar mudah digunakan dan intuitif, sehingga pengguna dapat dengan mudah menemukan Smart Contract yang sesuai dengan kebutuhan mereka, dan menggunakannya, baik digunakan secara langsung, atau menjadi komponen di dalam aplikasi yang lebih besar.

\subsubsection{Kebutuhan Fungsional}

Sistem ini memiliki beberapa kebutuhan fungsional yang harus dipenuhi agar dapat berfungsi dengan baik. Kebutuhan fungsional ini mencakup semua fitur dan fungsi yang harus ada dalam sistem untuk memenuhi tujuan utama sistem. Berikut adalah daftar kebutuhan fungsional yang diidentifikasi:

% tabel
% \begin{table}[h]
% 	\caption{Kebutuhan Fungsional Sistem}
% 	\vspace{0.25cm}
% 	\begin{center}
% 		\begin{tabular}{|c|l|}
% 			\hline
% 			\textbf{ID} & \textbf{Penjelasan} \\ \hline
% 			F01 & User dapat melakukan input kebutuhan dalam bentuk \textit{Natural Language} \\ \hline
% 			F02 & User dapat mencari Smart Contract  \\ \hline
% 			F03 & Pengembangan Prototipe \\ \hline
% 			F04 & Pengujian Prototipe \\ \hline
% 			F05 & Evaluasi dan Perbaikan \\ \hline
% 		\end{tabular}
% 	\end{center}
% \end{table}

\subsubsection{Kebutuhan Non-Fungsional}

\subsubsection{Model Use Case}

Kebutuhan fungsional dan karakteristik pengguna yang dirumuskan dapat dimodelkan menjadi use case. Seperti pada gambar \ref{image:usecase}, Use case kemudian dapat dipetakan menjadi sebuah Use Case Diagram yang menghubungkan relasi antara aktor dengan use case yang berkolerasi. Diagram ini menggambarkan interaksi antara pengguna dan sistem, serta fungsi-fungsi yang tersedia dalam sistem. Diagram use case ini akan membantu dalam memahami bagaimana pengguna akan berinteraksi dengan sistem dan fitur-fitur apa saja yang harus ada dalam sistem. Use case dapat dilihat secara detail pada lampiran XX.

\begin{figure}[ht]
	\centering
	\includegraphics[width=0.7\textwidth]{resources/chapter-3/use-case.png}
	\caption{Use Case Diagram}
	\label{image:usecase}
\end{figure}

\subsubsection{Risiko dan Mitigasi}


\break

\section{Rancangan}

Bagian ini akan menjelaskan terkait rancangan sistem yang akan dibangun. Rancangan akan mencakup rancangan struktural, rancangan behavioral, dan rancangan detail dari setiap komponen. Sistem Smart Contract Discovery ini, seperti pada gambar \ref{image:komponen-sistem}, terdiri dari 5 komponen utama yang saling berinteraksi satu sama lain. Komponen-komponen tersebut adalah:

\begin{enumerate}
	\item Ethereum Archive Node: Node Ethereum yang digunakan untuk melakukan ekstraksi data dari blockchain Ethereum. Berada pada eksternal sistem.
	\item eth2dgraph: Komponen yang digunakan untuk melakukan ekstraksi data dari Ethereum Archive Node dan menyimpannya ke dalam DgraphDB. Berada pada eksternal sistem.
	\item DgraphDB: Komponen yang digunakan untuk menyimpan data Smart Contracts yang sudah diekstraksi dari Ethereum Archive Node sekaligus data vector embeddings dari Smart Contracts yang telah diperkaya. Diperlukan modul data access untuk mengakses data dari DgraphDB dan mentransformasi data ke dalam bentuk vector embeddings.
	\item Smart Contract Enricher: Komponen yang digunakan untuk melakukan klasifikasi dan \textit{enrichment} data Smart Contracts.
	% \item Vector Database: Komponen yang digunakan untuk menyimpan data Smart Contracts yang sudah diklasifikasikan dan di-\textit{enrich} oleh Smart Contract Enricher. Diperlukan modul data access untuk mengakses data dari Vector Database dan mentransformasi data ke dalam bentuk vector embedding.
	% \item Query Enricher dan Data Retriever: Komponen yang digunakan untuk melakukan \textit{enrichment} terhadap query yang diberikan oleh pengguna dan melakukan data retrieval dari Vector Database.
	\item User Interface: Komponen yang digunakan untuk berinteraksi dengan pengguna. Terdapat dua jenis antarmuka yang akan dibangun, yaitu antarmuka berbasis \textit{API} dan antarmuka berbasis \textit{GUI}.
\end{enumerate}

\subsection{Rancangan Struktural}
\label{subsec:rancangan-struktural}

Arsitektur sistem yang akan dibangun akan digambarkan dengan Package Diagram. Sistem akan dibagi menjadi 3 kelompok komponen utama, yaitu eth2dgraph sebagai komponen luar sistem yang akan mengakses Ethereum Archive Node, komponen sistem utama yang terdiri dari DgraphDB dengan modul data access yang menyertakan modul transformasi data menjadi \textit{embeddings} dan pencarian data menggunakan \textit{embeddings}, Smart Contract Enricher, serta API yang akan mengekspos fungsi-fungsi yang ada pada sistem, dan komponen GUI. Ilustrasi dari rancangan struktural sistem dapat dilihat pada gambar \ref{image:rancangan-struktural}.

\begin{figure}[ht]
	\centering
	\includegraphics[width=1\textwidth]{resources/chapter-3/struktural.png}
	\caption{Rancangan struktural sistem}
	\label{image:rancangan-struktural}
\end{figure}

\subsection{Rancangan Behavioral}

Berdasarkan Use Case yang telah dirumuskan, terdapat 2 skenario utama yang akan dilakukan oleh pengguna, yaitu pencarian Smart Contracts dan pengambilan data address dari Smart Contracts. Selain itu, terdapat juga skenario tambahan yang terjadi di dalam sistem tanpa campur tangan pengguna, yaitu ekstraksi data Smart Contracts dari Blockchain, transformasi data Smart Contracts ke dalam Dgraph Database, Semantic Enrichment pada data Smart Contracts, transformasi data Smart Contracts menjadi vector embeddings, dan penyimpanan vector embeddings ke dalam Vector Database.

\subsubsection{Alur Mencari Smart Contract}

Pengguna akan melakukan pencarian dengan menggunakan API atau GUI. Pengguna akan memasukkan bahasa alami yang mendeskripsikan kebutuhan Smart Contract yang dicari. Data masukan yang diterima akan diteruskan ke API lalu ke komponen Retriever, yang akan melakukan query enrichment dan mengirimkan query ke dalam Vector Database dalam bentuk embedding. Hasil pencarian yang didapatkan akan dikembalikan ke API dan ditampilkan kepada pengguna dalam bentuk tabel yang berisi informasi Smart Contract yang relevan dengan query yang dimasukkan oleh pengguna. Ilustrasi alur pencarian Smart Contract dapat dilihat pada Gambar \ref{image:alur-pencarian-smart-contract}.

\begin{figure}[ht]
	\centering
	\includegraphics[width=0.7\textwidth]{resources/chapter-3/sequence-1.png}
	\caption{Alur Pencarian Smart Contract}
	\label{image:alur-pencarian-smart-contract}
\end{figure}

\subsubsection{Alur Mendapatkan Address Smart Contract}

Pengguna akan melakukan pencarian dengan menggunakan API atau GUI. Pengguna akan memasukkan dgraph ID yang didapatkan dari hasil pencarian Smart Contract. Data ID yang diterima akan diteruskan ke API lalu ke dgraph Client, yang akan melakukan query ke dalam Dgraph Database. Hasil pencarian yang didapatkan akan dikembalikan ke API dan ditampilkan kepada pengguna dalam bentuk tabel yang berisi informasi Smart Contract dengan dgraph ID yang dimasukkan oleh pengguna. Ilustrasi alur mendapatkan address Smart Contract dapat dilihat pada Gambar \ref{image:alur-mendapatkan-address-smart-contract}.

\begin{figure}[ht]
	\centering
	\includegraphics[width=0.7\textwidth]{resources/chapter-3/sequence-2.png}
	\caption{Alur Mendapatkan Address Smart Contract}
	\label{image:alur-mendapatkan-address-smart-contract}
\end{figure}

% \subsubsection{Alur Ekstraksi Data}

% \subsubsection{Alur Transformasi Data Menjadi Dgraph}

% \subsubsection{Alur Semantic Enrichment}

% \subsubsection{Alur Transformasi Data Menjadi Vector Embeddings}

% \subsubsection{Alur Penyimpanan Vector Embeddings}

\subsection{Rancangan Detail Komponen}

\subsubsection{Rancangan Detail Komponen GUI}

Komponen GUI akan digunakan sebagai salah satu antarmuka pengguna untuk berinteraksi dengan sistem. Komponen ini akan menyediakan tampilan sederhana yang memungkinkan pengguna untuk memasukkan query dalam bahasa alami, melihat hasil pencarian Smart Contracts, dan mendapatkan informasi lebih lanjut tentang Smart Contracts yang relevan. Komponen GUI akan berkomunikasi dengan API untuk mengirimkan query dan menerima hasil pencarian.

Terdapat 3 tampilan utama pada komponen GUI, yaitu:
\begin{itemize}
    \item Tampilan Pencarian Smart Contract: Pengguna dapat memasukkan query dalam bahasa alami untuk mencari Smart Contracts yang relevan dan mendapatkan daftar hasil pencarian Smart Contracts dengan informasi umum terkait Smart Contracts yang didapatkan.
    \item Tampilan Detail Smart Contract: Menampilkan informasi lebih lanjut tentang Smart Contract yang dipilih, termasuk address, deskripsi, dan metadata lainnya.
    \item Tampilan Import Smart Contract: Menampilkan cara melakukan import Smart Contract yang khusus untuk Smart Contract yang dipilih, sehingga pengguna dapat dengan mudah mengimport Smart Contract tersebut.
\end{itemize}

\subsubsection{Rancangan Detail Komponen API}

Komponen API akan menjadi antarmuka utama bagi pengguna untuk berinteraksi dengan fungsinoalitas sistem. Komponen ini akan menerima permintaan dari pengguna, memproses permintaan tersebut, dan mengembalikan respons yang sesuai. Komponen API akan berkomunikasi dengan komponen Retriever dan Dgraph Client untuk melakukan pencarian Smart Contracts dan mendapatkan informasi lebih lanjut tentang Smart Contracts yang relevan.

Komponen API akan menyediakan satu endpoint utama untuk menerima query dalam bahasa alami dari pengguna. Endpoint ini akan menerima permintaan pencarian, meneruskan query ke komponen Retriever, dan mengembalikan hasil pencarian kepada pengguna. Sebelum permintaan pencarian diteruskan ke komponen Retriever, API akan melakukan query expansion untuk meningkatkan relevansi hasil pencarian. Selain itu, sebelum hasil pencarian dikembalikan kepada pengguna, komponen Retriever akan melakukan query ke Dgraph Client untuk mendapatkan informasi detail dari Smart Contracts yang dihasilkan.

\subsubsection{Rancangan Detail Komponen Retriever}

Komponen Retriever akan bertanggung jawab untuk melakukan pencarian Smart Contracts berdasarkan query yang diberikan oleh pengguna. Komponen ini akan menerima query dari komponen API, melakukan query expansion untuk meningkatkan relevansi hasil pencarian, dan kemudian melakukan pencarian menggunakan interface yang disediakan oleh VectorDB Client. Hasil pencarian akan dikembalikan ke komponen API untuk ditampilkan kepada pengguna.

\subsubsection{Rancangan Detail Komponen Enricher}
% Bahas terkait schema disini

Komponen Enricher akan bertanggung jawab untuk memperkaya data Smart Contracts yang akan disimpan dalam VectorDB. \textit{Enrichment} ini dilakukan untuk membantu meningkatkan kualitas dan relevansi data yang disimpan, sehingga memudahkan proses pencarian dan pengambilan informasi.

Komponen Enricher terbagi menjadi 2 subkomponen, yaitu:
\begin{enumerate}
    \item \textbf{Semantic Enricher}: Subkomponen ini akan melakukan proses pengayaan data Smart Contracts secara paralel. Proses ini akan memperkaya data dengan informasi tambahan yang relevan, seperti metadata, deskripsi, dan informasi lainnya yang dapat membantu dalam pencarian dan pengambilan informasi.
    \item \textbf{Data Schema}:
    Subkomponen skema akan diterapkan kepada data Smart Contracts yang akan disimpan dalam DgraphDB. Skema ini akan memastikan bahwa data yang disimpan memiliki struktur yang konsisten dan dapat mendeskripsikan semantik Smart Contracts. Skema ini akan mencakup atribut-atribut penting dari Smart Contracts, seperti address, deskripsi, fungsionalitas, metadata, dan informasi lainnya yang relevan.
\end{enumerate}

Proses yang dilakukan oleh komponen Enricher adalah sebagai berikut:
\begin{enumerate}
    \item Komponen Enricher akan mengambil data Smart Contracts yang disimpan pada DgraphDB.
    \item Data Source Code dari Smart Contracts akan diambil dan dilakukan preprocessing sehingga dapat diproses dengan lebih efisien.
    \item Data Smart Contracts akan diproses oleh subkomponen Semantic Enricher untuk melakukan data enrichment. Proses ini akan menghasilkan data yang lebih kaya dan relevan.
    \item Setelah proses pengayaan selesai, data yang telah diperkaya akan disimpan kembali ke dalam DgraphDB. Proses penyimpanan ini akan menggunakan komponen Dgraph Client untuk memastikan bahwa data yang disimpan sesuai dengan skema yang telah ditentukan.
\end{enumerate}


\subsubsection{Rancangan Detail Komponen Parallel Enricher}

Komponen Parallel Enricher merupakan sebuah komponen \textit{wrapper} untuk komponen Semantic Enricher. Komponen ini akan membagi \textit{task} untuk enrichment dan menjalankan proses enrichment secara paralel. Hal ini dilakukan untuk meningkatkan efisiensi dan kecepatan proses pengayaan data Smart Contracts.

\subsubsection{Rancangan Detail Komponen VectorDB Client}

Komponen VectorDB Client akan bertanggung jawab untuk berkomunikasi dengan VectorDB yang digunakan untuk menyimpan dan mengambil data embeddings dari Smart Contracts yang sudah dilakukan enrichment. Komponen ini akan menyediakan fungsi untuk mengakses fungsionalitas pencarian semantik dari VectorDB. Terdapat sebuah skrip untuk melakukan \textit{pull} data dari DgraphDB, mengonversi menjadi embeddings, dan menyimpannya ke dalam VectorDB. Skrip ini akan dijalankan secara berkala untuk memastikan bahwa data yang disimpan di VectorDB selalu diperbarui dengan data terbaru dari DgraphDB.

\subsubsection{Rancangan Detail Komponen Dgraph Client}

Komponen Dgraph Client akan bertanggung jawab untuk berkomunikasi dengan DgraphDB yang digunakan untuk menyimpan data Smart Contracts. Komponen ini akan menyediakan fungsi untuk melakukan query, penyimpanan, mutasi skema, dan pengambilan data dari DgraphDB. Komponen ini akan menjadi antarmuka utama bagi komponen lain untuk berinteraksi dengan DgraphDB.


% package diagram
% sesuai domain, ada class diagramnya

