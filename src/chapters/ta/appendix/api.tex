\chapter{API}
\label{appendix:api}

\begin{enumerate}
	\item \texttt{FastAPI Application}: Aplikasi utama FastAPI dengan CORS middleware dan custom OpenAPI spec yang dimuat dari `openapi.yaml`.
	\item \texttt{Pydantic Models}: Model request dan response yang digunakan untuk validasi dan serialisasi, khususnya:
	      \begin{itemize}
		      \item \texttt{VectorSearchRequest}: Mendefinisikan `query`, `limit`, dan `threshold` untuk pencarian similarity.
		      \item \texttt{ContractResult}: Hasil pencarian yang berisi field seperti `id`, `name`, `description`, `created`, `verified`, `tags`, `storage\textunderscore protocol`, `storage\textunderscore address`, `experimental`, `solc\textunderscore version`, `verified\textunderscore source`, `verified\textunderscore source\textunderscore code`, `functionality`, `domain`, `security\textunderscore risks`, dan `similarity\textunderscore score`.
	      \end{itemize}
	\item \texttt{DgraphClient}: Klien Dgraph tunggal yang digunakan untuk melakukan operasi vector similarity search melalui method `vector\textunderscore search`.
	\item \texttt{POST /search}: Endpoint yang menerima `VectorSearchRequest`, memanggil `DgraphClient.vector\textunderscore search`, memetakan setiap hasil ke model `ContractResult`, dan mengembalikan JSONResponse dengan daftar hasil.
	\item \texttt{GET /}: Root endpoint yang mengembalikan halaman HTML statis sebagai landing page dokumentasi dan link ke Swagger UI, ReDoc, dan spesifikasi OpenAPI.
\end{enumerate}

% Sistem Query Refinement menggunakan prompt sistem yang dirancang khusus untuk meningkatkan kualitas pencarian Smart Contract. Prompt ini akan:
% \begin{enumerate}
%     \item Mengidentifikasi terminologi teknis yang relevan dengan blockchain dan Smart Contract.
%     \item Menambahkan deskripsi fungsionalitas yang spesifik.
%     \item Menyertakan pola dan standar umum seperti ERC20, ERC721, protokol DeFi.
%     \item Mempertimbangkan aspek keamanan yang relevan.
%     \item Menambahkan detail use case dan logika bisnis.
% \end{enumerate}

% API juga menyediakan fallback enhancement berbasis aturan yang akan digunakan jika client OpenAI tidak tersedia, dengan pattern matching untuk kategori-kategori seperti token, NFT, DeFi, governance, dan security.