\subsubsection{Kebutuhan Utama}
\label{subsubsec:kebutuhan-utama}

Berdasarkan permasalahan utama dan dekomposisi masalah yang sudah diidentifikasi, muncul beberapa kebutuhan utama untuk menyelesaikannya, terutama yang berkaitan dengan pengelolaan dan penggunaan Smart Contracts. Berikut adalah beberapa kebutuhan yang perlu dijawab:

\begin{enumerate}
	\item \textbf{Kebutuhan Ekstraksi Data Smart Contracts (K1)} \newline
	      Sistem harus mampu mengekstraksi data Smart Contracts yang ada di blockchain untuk dapat diproses dan digunakan lebih lanjut.
	      % DK1

	\item \textbf{Kebutuhan Penyimpanan dan Pengelolaan Data Smart Contracts (K2)} \newline
	      Sistem harus dapat menyimpan dan mengelola data Smart Contracts yang telah diekstrak dan di-\textit{enrich}, sehingga data dapat diakses dan digunakan untuk pencarian lebih lanjut.
	      % DK2, DK4

	\item \textbf{Kebutuhan Klasifikasi Fungsional dan Semantik (K3)} \newline
	      Sistem harus dapat mengklasifikasikan Smart Contracts berdasarkan fungsionalitas dan semantik, dengan memperkaya data semantik (\textit{semantic enrichment}) untuk memudahkan pencarian kontrak yang relevan.
	      % DK3, DK5

	\item \textbf{Kebutuhan Solusi \textit{Off-Chain} (K4)} \newline
	      Sistem harus berjalan secara \textit{off-chain} untuk mengurangi beban pada jaringan blockchain dan meningkatkan skalabilitas.
	      % DK7

	\item \textbf{Kebutuhan Translasi Kebutuhan Pengguna ke Smart Contracts (K5)} \newline
	      Sistem harus dapat mentranslasikan kebutuhan fungsionalitas yang diajukan oleh pengguna menjadi kontrak yang relevan.
	      % DK5, DK6 

	\item \textbf{Kebutuhan Kemampuan Query Data (K6)} \newline
	      Data yang disimpan dalam sistem harus dapat di-query dengan mudah untuk memungkinkan pengambilan informasi yang relevan.
	      % DK6

	\item \textbf{Kebutuhan Peningkatan \textit{Reusability} (K7)} \newline
	      Sistem harus memudahkan penggunaan ulang Smart Contracts yang telah ada, dengan memastikan kemudahan dalam melakukan \textit{import} Smart Contract.
	      % DK8
\end{enumerate}

Kebutuhan-kebutuhan ini akan menjadi dasar dalam merancang solusi yang tepat untuk mengatasi permasalahan utama yang diidentifikasi sebelumnya. Dengan memenuhi kebutuhan-kebutuhan tersebut, diharapkan solusi yang diusulkan dapat memberikan manfaat yang maksimal bagi pengguna dalam mengelola dan menggunakan Smart Contracts di blockchain.

% Bisa dikasih visualisasi juga 