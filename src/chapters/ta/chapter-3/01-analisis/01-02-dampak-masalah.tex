\subsubsection{Dampak Masalah}
\label{subsubsec:dampak-masalah}

Secara lebih luas, dampak dari permasalahan utama yang telah diidentifikasi adalah:

\begin{enumerate}
  \item \textbf{\textit{Codebase} yang Kompleks} \newline
  Redundansi pengembangan Smart Contracts menghasilkan \textit{codebase} yang tumbuh besar dan semakin sulit untuk dipelihara. Kompleksitas ini dapat dihindari jika pengembang mampu menggunakan kembali Smart Contracts yang sudah ada, sehingga efisiensi pengelolaan kode meningkat.
  \item \textbf{Keterbatasan Pengembangan Sistem} \newline
  Lingkungan pengembangan Smart Contracts yang saat ini sudah mendukung arsitektur berbasis Service-Oriented Architecture (SOA) atau Object-Oriented Approach (OOA) menjadi kurang optimal, karena pengembang lebih banyak berfokus pada pembuatan kontrak baru daripada memanfaatkan kembali kontrak yang sudah ada. Hal ini membatasi potensi inovasi pengembangan internal dApps.
  \item \textbf{Biaya dan Waktu Pengembangan yang Tinggi} \newline 
  Membuat dan mendeploy Smart Contracts baru membutuhkan biaya signifikan, termasuk biaya \textit{deployment} ke jaringan blockchain, yang semakin besar dengan frekuensi pengembangan yang tinggi. Selain itu, waktu yang dihabiskan untuk membuat kontrak baru juga meningkat, memperlambat proses pengembangan aplikasi secara keseluruhan.
  \item \textbf{Inefisiensi Blockchain} \newline
  Dengan sifat Blockchain sebagai ledger terdistribusi, setiap Smart Contract yang di-\textit{deploy} akan tetap tersimpan secara permanen. Redundansi kontrak ini memperbesar ukuran Blockchain, memperlambat akses data, dan menciptakan \textit{bottleneck} yang mengurangi efisiensi sistem secara keseluruhan. Masalah ini tidak hanya berdampak pada penyimpanan, tetapi juga memengaruhi kemampuan Blockchain untuk mendukung aplikasi berskala besar.
  \item \textbf{Kerentanan Keamanan} \newline
  Semakin banyak baris kode yang ditulis, akan semakin besar kemungkinan terdapat kerentanan. Dengan kurangnya prinsip \textit{reusability}, banyak kode yang ditulis kembali tetapi tidak terjamin secara kualitas, sehingga meningkatkan risiko kesalahan dan kerentanan keamanan yang mungkin terjadi.
\end{enumerate}