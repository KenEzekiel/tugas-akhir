\subsection{Analisis Masalah}
\label{subsec:analisis-masalah}

% Dalam pengembangan dApps, Smart Contracts berperan sebagai \textit{building blocks} yang saling terhubung dan berinteraksi sesuai dengan fungsionalitasnya. Namun, dengan jumlah Smart Contracts yang terus bertambah di blockchain dan ketiadaan mekanisme klasifikasi yang jelas, pemilihan Smart Contracts yang tepat menjadi tantangan yang kompleks. Kesulitan ini terutama muncul dalam pengklasifikasian dan pelabelan Smart Contracts secara semantik, yang berdampak pada sulitnya menemukan Smart Contracts yang sesuai dengan kebutuhan pengguna.

% Kesulitan dalam menemukan Smart Contracts yang sesuai sering kali mendorong pengembang dApps untuk membuat Smart Contracts khusus untuk aplikasi yang dikembangkan, meskipun fungsionalitasnya serupa dengan yang sudah ada. Akibatnya, jumlah Smart Contracts di blockchain tumbuh secara eksponensial, memperbesar ukuran blockchain secara signifikan. Pertumbuhan yang tidak terkendali ini menyebabkan \textit{bloating}, yang berdampak pada inefisiensi penyimpanan dan masalah skalabilitas, sehingga memengaruhi kinerja sistem konsensus. Selain itu, biaya pengembangan dApps pun meningkat karena pengembang harus membuat dan mendeploy Smart Contracts baru untuk setiap aplikasi, meskipun solusi serupa sebenarnya sudah tersedia. Kurangnya dukungan terhadap mekanisme \textit{reusability} ini juga berdampak pada kualitas dari Smart Contracts secara umum karena pengembang cenderung menulis kode yang tidak terjamin secara kualitas.
% terutama dari segi keamanan


% Seringkali butuh untuk berinteraksi dengan kontrak yang udah ada untuk mengakses fungsionalitasnya, ataupun menggunakan sebuah infrastruktur publik
% tapi dengan jumlah SC yang bertambah dan belum ada mekanisme pencarian yang baik, masih sulit untuk mendapatkan SC yang relevan
% Terutama dalam klasifikasinya secara semantik, karena sistem sekarang belum ada, dan kalaupun ada, itu diinput oleh orang yang melakukan verifikasinya saja, yang belum tentu mendeskripsikan dengan sesuai. selain itu, walaupun ada klasifikasi pun, belum ada mekanisme buat menemukan SC yang sesuainya.
Dalam pengembangan Smart Contracts, seringkali dibutuhkan interaksi dengan kontak yang terdapat di blockchain untuk mengakses fungsionalitas kontrak tersebut atau menggunakan sebuah infrastruktur publik. Namun, dengan jumlah Smart Contracts yang terus bertambah dan ketiadaan mekanisma pencarian yang baik, masih sulit untuk mendapatkan Smart Contract yang relevan dengan kebutuhan. Terutama dalam pengklasifikasian Smart Contracts secara semantik yang belum disediakan secara luas oleh sistem-sistem. Sistem yang memiliki mekanisme tersebut seringkali menggunakan mekanisme dimana pengguna yang melakukan pendaftaran Smart Contracts itu sendiri yang memasukkan deskripsi dan kategori dari Smart Contracts, yang belum tentu sesuai dengan kode yang dimiliki kontrak. Selain itu, belum terdapat juga sebuah mekanisme untuk menemukan Smart Contract yang sesuai sehingga sistem klasifikasi yang dimiliki masih belum dapat dimanfaatkan dengan baik.

% Sistem saat ini masih menggunakan sebuah repositori utama untuk mendaftarkan sebuah Smart Contract, sehingga data dari blockchain harus di push ke repositori. Model ini bikin bottleneck karena akan ada banyak instance repositori yang berbeda-beda (contohnya skrg ada Etherscan, Sourcify, untuk mendaftarkan Smart Contract). Di repositori-repositori ini juga belum ada mekanisme pencarian yang baik, karena sekarang masih bersifat lexical, bertumpu pada ketersesuaian kata kunci atau sintaks. Padahal sintaks tidak terlalu merepresentasikan semantik dari sebuah SC.
Sistem-sistem saat ini masih menggunakan pendekatan sebuah repositori utama untuk mendaftarkan Smart Contract, sehingga data dari blockchain harus di-\textit{push} ke repositori oleh masing-masing pengguna. Model ini mengakibatkan \textit{bottleneck} karena akan terdapat banyak \textit{instance} repositori yang berbeda-beda kontennya dengan mekanisme pendaftaran atau verifikasi yang berbeda-beda. Hal ini tercermin dari Etherscan dan Sourcify yang memiliki perbedaan mekanisme untuk mendaftarkan Smart Contracts. Repositori-repositori ini menggunakan mekanisme pencarian berbasis teks ataupun \textit{address} Smart Contracts, yang bersifat leksikal dan bertumpu pada ketersesuaian kata kunci, sintaks, atau \textit{address}. Padahal, sintaks dan \textit{address} tidak dapat merepresentasikan semantik dari Smart Contract dengan baik.

% Karena kesulitan itu, pengembangan sistem SC sering terhambat, seperti hambatan dalam mencari kontrak untuk mengintegrasikan dengan USDC, sebuah stablecoin yang umum, di tengah banyaknya kontrak fraud. atau contoh lainnya adalah menggunakan router decentralized exchange seperti uniswap atau sushiswap.
% Selain menggunkana instance, ada juga kontrak yang emang di deploy sebagai infrastructure publik seperti Uniswap V3 Factory, atau ENS registry, oracles, bridges, etc
Kesulitan-kesulitan itu menghambat pengembangan sistem Smart Contracts, seperti hambatan dalam pencarian kontrak untuk mengintegrasikan USDC, sebuah \textit{stablecoin} yang umum, di tengah banyaknya kontrak \textit{fraud}; atau hambatan dalam mencari \textit{router} untuk \textit{Decentralized Exchange} seperti UniSwap atau SushiSwap. \textit{Use case} lainnya untuk penggunaan ulang ini adalah kontrak yang berperan sebagai infrastruktur publik, seperti UniSwap V3 Factory, ENS Registry, Oracles, Bridges, dan lainnya.

% Banyak penelitian yang udah mencoba menjawab permasalahan ini, untuk bridge the gap antar kebutuhan dengan semantik dari apa yang dikerjakan oleh kontrak secara nyata, dan salah satu approach nya adalah dengan menggunakan AI untuk mendeskripsikan semantik yaitu fungsi dan maksud dari sebuah Smart Contracts (insert citations), tapi masalahnya adalah riset ini tidak open source, tidak ada data pipeline untuk mengintegrasikan menjadi sistem yang dapat digunakan di industri, dan masih menggunakan model lama.
Terdapat banyak penelitian yang sudah mencoba menjawab permasalahan ini, yaitu untuk menjembatani kesenjangan antar kebutuhan pengguna dengan semantik dari apa yang sebenarnya terjadi di dalam sebuah kontrak. Salah satu pendekatan untuk menyelesaikan permasalahan ini adalah menggunakan AI untuk mendeskripsikan semantik yaitu fungsi dan maksud dari sebuah Smart Contracts \parencite{zhang2021smart} \parencite{stan} \parencite{shi2021semantic} \parencite{shi2021semantic}. Permasalahan dari riset-riset ini adalah kesenjangan antar konsep yang ditawarkan dengan \textit{use case} dan posisi dengan industri karena sistem yang ditawarkan tidak bersifat terbuka dan tidak terdapat sebuah \textit{pipeline} yang dapat digunakan di industri yang mengintegrasikan data blockchains eara langsung. Selain itu, riset-riset ini juga belum mengeksplorasi kapabilitas model-model LLM modern untuk mengerti dan menganalisis semantik dari sebuah kode.


\subsubsection{Dekomposisi Masalah}
\label{subsubsec:dekomposisi-masalah}
% masalah utama: sulit untuk menemukan Smart Contracts yang sesuai dengan kebutuhan pengguna di tengah banyaknya Smart Contracts yang ada pada Blockchain. Kesulitan ini terutama muncul dalam pengklasifikasian dan pelabelan Smart Contracts secara semantik, yang berdampak pada sulitnya menemukan Smart Contracts yang sesuai dengan kebutuhan pengguna.
% dampak masalah: pertumbuhan jumlah Smart Contracts yang tidak terkendali dan peningkatan ukuran Blockchain (\textit{bloating}), peningkatan biaya dan waktu serta penurunan kualitas karena kurangnya mekanisme \textit{reusability} Smart Contracts.

% dekomposisi permasalahan
% 1. bagaimana mengekstraksi data smart contracts dari Blockchain
% 2. bagaimana menyimpan dan mengelola data smart contracts yang sudah diekstrak
% 3. (ini nanti jawabannya dengen semantik enrichment) bagaimana melabelkan dan mengklasifikasikan Smart Contracts yang akan mempermudah pencarian berdasarkan kebutuhan pengguna -> untuk identifikasi kebutuhan yang dijawab oleh Smart Contract
% 4. bagaimana melakukan manajemen terhadap data smart contracts yang sudah dilabelkan dan diklasifikasikan
% 5. (ini nanti jawabannya pake RAG) bagaimana menghubungkan permintaan kebutuhan pengguna dengan data yang sudah dilabelkan dan diklasifikasikan

Secara singkat, masalah utama yang diidentifikasi adalah kesulitan untuk menemukan Smart Contracts yang sesuai dengan kebutuhan pengguna. Masalah ini terutama muncul karena ketiadaan mekanisme klasifikasi yang jelas untuk mengidentifikasi Smart Contracts dan mekanisme pencarian Smart Contracts yang dapat mengakomodasi klasifikasi tersebut.

Permasalahan ini dapat dijawab dengan sebuah mekanisme pencarian Smart Contracts yang lebih akurat dan efisien, yang memungkinkan pengguna untuk menemukan Smart Contracts yang sesuai dengan kebutuhan mereka dengan memanfaatkan pelabelan atau pengklasifikasian dari Smart Contracts. Mekanisme ini juga diinspirasi oleh gabungan dari Semantic Web dan blockchain seperti pada GraphChain \parencite{sopek2018graphchain}, yang memungkinkan untuk melakukan pencarian Smart Contracts berdasarkan semantik yang lebih mendalam. Dengan demikian, pengguna dapat menemukan Smart Contracts yang sesuai dengan kebutuhan mereka dengan lebih mudah dan efisien.
% berdasarkan kebutuhan yang dijawabnya

% ini masih umum kaya yaudah ini idenya dan emang cara solve nya ini, tapi nanti di bab berikutnya itu fokusnya gimana implementasinya dan ide utama yang menawarkan kebaruannya tuh apa

Untuk mengembangkan mekanisme tersebut, digunakan referensi dengan kasus serupa, yaitu Data Retrieval Pipeline yang digunakan dalam RAG \parencite{CrateDB_RAG_Pipelines}. Referensi ini dipilih karena pada dasarnya mekanisme yang akan dikembangkan merupakan Data Retrieval Pipeline. Berdasarkan referensi tersebut, terdapat beberapa poin penting yang perlu dijawab, dan poin-poin ini akan menjadi dasar pemilihan teknologi serta metode yang digunakan dalam sistem yang akan dibangun. Berikut adalah poin-poin dekomposisi masalah yang menjadi dasar tugas akhir ini:

\begin{enumerate}
	\item Bagaimana melakukan ekstraksi data Smart Contracts dari blockchain?

	\item Bagaimana menyimpan dan mengelola data Smart Contracts yang sudah diekstrak?

	\item Bagaimana melabelkan atau mengklasifikasikan Smart Contracts yang akan mempermudah pencarian berdasarkan kebutuhan?

	\item Bagaimana melakukan manajemen terhadap data Smart Contracts yang sudah dilabelkan dan diklasifikasikan?

	\item Bagaimana memahami kebutuhan dari pengguna?

	\item Bagaimana menghubungkan permintaan kebutuhan pengguna dengan data yang sudah dilabelkan dan diklasifikasikan?

	\item Bagaimana melakukan pencarian dan memfasilitasi penggunaan ulang tanpa mempengaruhi kinerja sistem?

	\item Bagaimana mempermudah integrasi Smart Contracts dalam pengembangan dApps?
\end{enumerate}

% \subsubsection{Dampak Masalah}
\label{subsubsec:dampak-masalah}

Secara lebih luas, dampak dari permasalahan utama yang telah diidentifikasi adalah:

\begin{enumerate}
	\item \textbf{\textit{Codebase} yang Kompleks} \newline
	      Redundansi pengembangan Smart Contracts menghasilkan \textit{codebase} yang tumbuh besar dan semakin sulit untuk dipelihara. Kompleksitas ini dapat dihindari jika pengembang mampu menggunakan kembali Smart Contracts yang sudah ada, sehingga efisiensi pengelolaan kode meningkat.
	\item \textbf{Keterbatasan Pengembangan Sistem} \newline
	      Lingkungan pengembangan Smart Contracts yang saat ini sudah mendukung arsitektur berbasis Service-Oriented Architecture (SOA) atau Object-Oriented Approach (OOA) menjadi kurang optimal, karena pengembang lebih banyak berfokus pada pembuatan kontrak baru daripada memanfaatkan kembali kontrak yang sudah ada. Hal ini membatasi potensi inovasi pengembangan internal dApps.
	\item \textbf{Biaya dan Waktu Pengembangan yang Tinggi} \newline
	      Membuat dan mendeploy Smart Contracts baru membutuhkan biaya signifikan, termasuk biaya \textit{deployment} ke jaringan blockchain, yang semakin besar dengan frekuensi pengembangan yang tinggi. Selain itu, waktu yang dihabiskan untuk membuat kontrak baru juga meningkat, memperlambat proses pengembangan aplikasi secara keseluruhan.
	\item \textbf{Inefisiensi Blockchain} \newline
	      Dengan sifat blockchain sebagai ledger terdistribusi, setiap Smart Contract yang di-\textit{deploy} akan tetap tersimpan secara permanen. Redundansi kontrak ini memperbesar ukuran blockchain, memperlambat akses data, dan menciptakan \textit{bottleneck} yang mengurangi efisiensi sistem secara keseluruhan. Masalah ini tidak hanya berdampak pada penyimpanan, tetapi juga memengaruhi kemampuan blockchain untuk mendukung aplikasi berskala besar.
	\item \textbf{Kerentanan Keamanan} \newline
	      Semakin banyak baris kode yang ditulis, akan semakin besar kemungkinan terdapat kerentanan. Dengan kurangnya prinsip \textit{reusability}, banyak kode yang ditulis kembali tetapi tidak terjamin secara kualitas, sehingga meningkatkan risiko kesalahan dan kerentanan keamanan yang mungkin terjadi.
\end{enumerate}

\subsubsection{Kebutuhan Utama}
\label{subsubsec:kebutuhan-utama}

Berdasarkan permasalahan utama dan dekomposisi masalah yang sudah diidentifikasi, muncul beberapa kebutuhan utama untuk menyelesaikannya, terutama yang berkaitan dengan pengelolaan dan penggunaan Smart Contracts. Berikut adalah beberapa kebutuhan yang perlu dijawab:

% \begin{enumerate}
%   \item \textbf{Peningkatan \textit{Reusability} Smart Contracts} \newline
%   Dibutuhkan sebuah sistem yang dapat meningkatkan \textit{reusability} dari Smart Contracts yang sudah ada. Hal ini penting untuk mengurangi redundansi dalam pengembangan Smart Contracts baru yang sebenarnya memiliki fungsionalitas serupa. Dengan adanya mekanisme pencarian berdasarkan fungsionalitas dan semantik, pengembang dapat memanfaatkan kontrak yang tersedia tanpa perlu menulis dari awal, sehingga menekan biaya serta waktu \textit{deployment}.
%   \item \textbf{Manajemen \textit{Smart Contracts} yang Efisien} \newline
%   Dibutuhkan sistem yang mendukung manajemen Smart Contracts secara efisien. Sistem ini tidak hanya membantu dalam melakukan pelabelan dan pengklasifikasian Smart Contracts berdasarkan fungsinya, tetapi juga memungkinkan pengembang untuk memahami hubungan antar kontrak. Dengan pendekatan manajemen yang baik, proses pengelolaan Smart Contracts menjadi lebih terstruktur dan sistematis, yang pada akhirnya memudahkan integrasi dalam pengembangan dApps.
%   \item \textbf{Solusi \textit{Off-Chain} untuk Mengurangi Beban Blockchain} \newline
%   Untuk mengurangi beban pada Blockchain, diperlukan solusi yang memungkinkan sebagian proses dilakukan di luar jaringan Blockchain (off-chain). Dengan cara ini, pencarian dan pengelolaan Smart Contracts dapat dilakukan tanpa membebani kapasitas penyimpanan atau mengganggu efisiensi sistem konsensus di Blockchain. Solusi semacam ini dapat menjadi pendukung utama untuk menciptakan ekosistem Blockchain yang lebih skalabel dan efisien.
%   \item \textbf{Pengurangan Biaya dan Waktu Pengembangan} \newline
%   Selain meminimalkan redundansi, sistem yang dikembangkan harus membantu menekan biaya dan waktu pengembangan. Dengan menyediakan mekanisme pencarian berbasis semantik yang tepat sasaran, pengembang dapat langsung menemukan dan menggunakan Smart Contracts yang sudah ada, menghemat biaya \textit{deployment} baru dan waktu pengembangan.
%   \item \textbf{Skalabilitas dan Efisiensi Penyimpanan Blockchain} \newline
%   Solusi yang diusulkan juga harus mendukung peningkatan skalabilitas dan efisiensi penyimpanan Blockchain. Dengan mengurangi jumlah Smart Contracts yang \textit{redundant}, sistem ini dapat membantu mencegah \textit{bloating} Blockchain dan memastikan penyimpanan tetap efisien, sehingga jaringan dapat terus mendukung pengembangan aplikasi berskala besar.
% \end{enumerate}

% kebutuhan yang muncul
% 1. sistem harus bisa klasifikasi fungsional dan semantik dari sc (semantic enrichment)
% 2. import dan reuse sc harus dipermudah oleh sistem || sistem harus bisa memudahkan reusability -> lebih mudah untuk reuse sc
% 3. sistem harus bisa manajemen sc -> grouping sc berdasarkan label semantik
% 4. sistem harus berjalan secara offchain -> tidak membebani blockchain

% 5. sistem harus bisa ekstraksi data sc dari blockchain
% 6. sistem harus bisa menyimpan dan mengelola data sc yang sudah diekstrak
% 7. sistem harus bisa mentranslasikan kebutuhan pengguna ke query yang sesuai
% 8. data yang disimpan di dalam sistem harus dapat dilakukan query 

\begin{enumerate}
	\item \textbf{Kebutuhan Ekstraksi Data Smart Contracts (K1)} \newline
	      Sistem harus mampu mengekstraksi data Smart Contracts yang ada di Blockchain untuk dapat diproses dan digunakan lebih lanjut.
	      % DK1

	\item \textbf{Kebutuhan Penyimpanan dan Pengelolaan Data Smart Contracts (K2)} \newline
	      Sistem harus dapat menyimpan dan mengelola data Smart Contracts yang telah diekstrak dan di-\textit{enrich}, sehingga data dapat diakses dan digunakan untuk pencarian lebih lanjut.
	      % DK2, DK4

	\item \textbf{Kebutuhan Klasifikasi Fungsional dan Semantik (K3)} \newline
	      Sistem harus dapat mengklasifikasikan Smart Contracts berdasarkan fungsionalitas dan semantik, dengan memperkaya data semantik (\textit{semantic enrichment}) untuk memudahkan pencarian kontrak yang relevan.
	      % DK3, DK5

	\item \textbf{Kebutuhan Solusi \textit{Off-Chain} (K4)} \newline
	      Sistem harus berjalan secara \textit{off-chain} untuk mengurangi beban pada jaringan Blockchain dan meningkatkan skalabilitas.
	      % DK7

	\item \textbf{Kebutuhan Translasi Kebutuhan Pengguna ke Smart Contracts (K5)} \newline
	      Sistem harus dapat mentranslasikan kebutuhan fungsionalitas yang diajukan oleh pengguna menjadi kontrak yang relevan.
	      % DK5, DK6 

	\item \textbf{Kebutuhan Kemampuan Query Data (K6)} \newline
	      Data yang disimpan dalam sistem harus dapat di-query dengan mudah untuk memungkinkan pengambilan informasi yang relevan.
	      % DK6

	\item \textbf{Kebutuhan Peningkatan \textit{Reusability} (K7)} \newline
	      Sistem harus memudahkan penggunaan ulang Smart Contracts yang telah ada, dengan memastikan kemudahan dalam melakukan \textit{import} Smart Contract.
	      % DK8
\end{enumerate}

Kebutuhan-kebutuhan ini akan menjadi dasar dalam merancang solusi yang tepat untuk mengatasi permasalahan utama yang diidentifikasi sebelumnya. Dengan memenuhi kebutuhan-kebutuhan tersebut, diharapkan solusi yang diusulkan dapat memberikan manfaat yang maksimal bagi pengguna dalam mengelola dan menggunakan Smart Contracts di Blockchain.

% Bisa dikasih visualisasi juga 

% \subsubsection{Ajuan Solusi}
\label{subsubsec:ajuan-solusi}

Berdasarkan kebutuhan-kebutuhan utama yang telah diidentifikasi, diperlukan sebuah solusi yang mampu menjawab berbagai tantangan dalam pengelolaan dan penggunaan Smart Contracts di blockchain. Solusi ini harus memungkinkan pencarian Smart Contracts berdasarkan fungsionalitasnya, sehingga pengembang tidak perlu membuat kontrak baru setiap kali mengembangkan dApps, terutama ketika kontrak dengan fungsionalitas serupa sudah tersedia. Dengan meningkatkan \textit{reusability} dari Smart Contracts yang ada, sistem ini dapat membantu menekan redundansi dan mengoptimalkan penggunaan sumber daya yang ada.

Untuk memenuhi kebutuhan ini, diajukan sebuah sistem pencarian Smart Contracts berbasis semantik yang dirancang untuk:

% rancangan solusi
% 

\begin{enumerate}
	\item Mengidentifikasi Smart Contracts dengan fungsionalitas yang sesuai berdasarkan kebutuhan pengembang. (AS1)
	\item Merekomendasikan Smart Contracts yang sudah ada dengan mempertimbangkan hubungan semantik antar kontrak, sehingga hasil pencarian lebih relevan dan dapat langsung digunakan. (AS2)
	\item Mendukung efisiensi biaya, waktu, dan pengelolaan blockchain, dengan memastikan bahwa proses pencarian dan pengelolaan dilakukan secara terstruktur dan tidak membebani jaringan blockchain. (AS3)
	      % tidak membebani dengan cara off-chain -> melakukan ekstraksi di luar blockchain, mapping di luar blockchain dst
	\item Merekomendasikan Smart Contracts yang aman dan terpercaya dengan mempertimbangkan verifikasi dari Smart Contracts. (AS4)
\end{enumerate}

Solusi ini dirancang untuk memenuhi kebutuhan \textit{reusability}, manajemen Smart Contracts, dan efisiensi sistem, sekaligus mendukung off-chain \textit{processing} agar ekosistem blockchain tetap skalabel dan efisien. Dengan pendekatan berbasis semantik ini, pengembang dApps dapat lebih mudah menemukan dan menggunakan Smart Contracts yang sesuai, tanpa harus menghabiskan waktu dan biaya untuk menciptakan kontrak baru.
