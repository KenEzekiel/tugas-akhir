\subsubsection{Ajuan Solusi}
\label{subsubsec:ajuan-solusi}

Berdasarkan kebutuhan-kebutuhan utama yang telah diidentifikasi, diperlukan sebuah solusi yang mampu menjawab berbagai tantangan dalam pengelolaan dan penggunaan Smart Contracts di blockchain. Solusi ini harus memungkinkan pencarian Smart Contracts berdasarkan fungsionalitasnya, sehingga pengembang tidak perlu membuat kontrak baru setiap kali mengembangkan dApps, terutama ketika kontrak dengan fungsionalitas serupa sudah tersedia. Dengan meningkatkan \textit{reusability} dari Smart Contracts yang ada, sistem ini dapat membantu menekan redundansi dan mengoptimalkan penggunaan sumber daya yang ada.

Untuk memenuhi kebutuhan ini, diajukan sebuah sistem pencarian Smart Contracts berbasis semantik yang dirancang untuk:

% rancangan solusi
% 

\begin{enumerate}
	\item Mengidentifikasi Smart Contracts dengan fungsionalitas yang sesuai berdasarkan kebutuhan pengembang. (AS1)
	\item Merekomendasikan Smart Contracts yang sudah ada dengan mempertimbangkan hubungan semantik antar kontrak, sehingga hasil pencarian lebih relevan dan dapat langsung digunakan. (AS2)
	\item Mendukung efisiensi biaya, waktu, dan pengelolaan blockchain, dengan memastikan bahwa proses pencarian dan pengelolaan dilakukan secara terstruktur dan tidak membebani jaringan blockchain. (AS3)
	      % tidak membebani dengan cara off-chain -> melakukan ekstraksi di luar blockchain, mapping di luar blockchain dst
	\item Merekomendasikan Smart Contracts yang aman dan terpercaya dengan mempertimbangkan verifikasi dari Smart Contracts. (AS4)
\end{enumerate}

Solusi ini dirancang untuk memenuhi kebutuhan \textit{reusability}, manajemen Smart Contracts, dan efisiensi sistem, sekaligus mendukung off-chain \textit{processing} agar ekosistem blockchain tetap skalabel dan efisien. Dengan pendekatan berbasis semantik ini, pengembang dApps dapat lebih mudah menemukan dan menggunakan Smart Contracts yang sesuai, tanpa harus menghabiskan waktu dan biaya untuk menciptakan kontrak baru.