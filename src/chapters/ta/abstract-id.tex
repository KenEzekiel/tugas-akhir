\clearpage
\chapter*{ABSTRAK}
\addcontentsline{toc}{chapter}{ABSTRAK}
\begin{center}
	\center
	\begin{singlespace}
		\large\bfseries\MakeUppercase{\thetitle}

		\normalfont\normalsize
		Oleh:

		\bfseries \theauthor
	\end{singlespace}
\end{center}

\begin{singlespace}
	\small

    % Di tengah pertumbuhan data Smart Contracts dalam blockchain, kebutuhan untuk sebuah sistem pencarian yang mengakomodasi penggunaan ulang semakin meningkat.
	% Pertumbuhan adopsi blockchain dan peningkatan jumlah Smart Contracts di dalam ekosistem blockchain menciptakan tantangan-tantangan baru. Salah satu tantangan utamanya adalah untuk menggunakan kembali Smart Contracts yang telah ada yang sesuai dengan kebutuhan. Sistem pencarian Smart Contracts saat ini masih mengandalkan pencarian berbasis kata kunci atau teks yang kurang mempertimbangkan konteks semantik dari kontrak tersebut. Sehingga, pengguna masih sulit untuk menemukan Smart Contracts yang sesuai dengan kebutuhan mereka. Pada tugas akhir ini dikembangkan sebuah Smart Contract Discovery System berbasis konteks semantik yang memanfaatkan kemampuan LLM untuk mendeskripsikan \textit{source code} dari sebuah kontrak secara semantik dan melakukan pengambilan data menggunakan kemiripan semantik. Sistem ini memanfaatkan Distributed Graph Database (Dgraph) untuk menyimpan data kontrak dan embeddings yang merepresentasikan kontrak tersebut. Sistem ini mengintegrasikan modul ekstraksi data, \textit{enrichment} data, pemodelan data, dan \textit{retrieval} data untuk meningkatkan relevansi hasil pencarian. Hasil evaluasi menunjukkan bahwa sistem ini mampu meningkatkan presisi dalam menemukan kontrak yang relevan dibandingkan dengan pendekatan tradisional. Dengan demikian, rancangan ini membuktikan efektivitas penggunaan konteks semantik dalam penemuan Smart Contracts pada ekosistem blockchain.

	Smart Contracts sudah diadopsi secara luas dan terdapat kebutuhan untuk berinteraksi atau menggunaka ulang kontrak yang sudah ada. Sistem saat ini menggunakan mekanisme pencarian yang bersifat leksikal, yang bertumpu pada ketersesuaian sintaks, yang tidak selalu merepresentasikan fungsi dan maksud dari kontrak. Terdapat beberapa sistem yang sudah mengeksplorasi pendekatan untuk mendeskripsikan fungsi dan maksud dari sebuah kontrak, tetapi sistem-sistem tersebut tidak terintegrasi secara \textit{end-to-end} dari ekstraksi sampai pencarian dan belum memanfaatkan LLM modern dengan kapabilitasnya untuk menganalisis dan mengerti fungsi dari sebuah kode. 
	
	Pada tugas akhir ini dikembangkan sebuah Smart Contract Discovery System berbasis konteks semantik, yang merupakan sebuah \textit{pipeline} dari ekstraksi data Smart Contracts pada Blockchain Ethereum, ekstraksi deskripsi semantik dari Smart Contracts tersebut menggunakan LLM modern dengan struktur yang modular sehingga dapat mengganti model dengan mudah, lalu memasukkan data ke sebuah repositori yang dapat dilakukan pencarian secara semantik menggunakan \textit{cosine similarity} dari \textit{embeddings} antar kueri dan data. Hasil evaluasi menunjukkan bahwa sistem ini mampu menemukan Smart Contracts dengan kategori yang dimaksud dari kueri pengguna dari hasil ekstraksi Smart Contracts yang terdapat di blockchain dan terdapat peningkatan presisi saat dibandingkan dengan sistem pencarian berbasis teks. Dengan demikian, rancangan ini membuktikan efektivitas sistem pencarian berbasis semantik dalam penemuan Smart Contracts pada ekosistem blockchain.

	\textbf{\textit{Kata kunci: Smart Contracts, Sistem Pencarian, Semantik, LLM, Enrichment, Embeddings, Retrieval}}

\end{singlespace}
\clearpage