\clearpage
\chapter*{ABSTRAK}
\addcontentsline{toc}{chapter}{ABSTRAK}
\begin{center}
	\center
	\begin{singlespace}
		\large\bfseries\MakeUppercase{\thetitle}

		\normalfont\normalsize
		Oleh:

		\bfseries \theauthor
	\end{singlespace}
\end{center}

\begin{singlespace}
	\small

    Di tengah pertumbuhan data Smart Contracts dalam blockchain, kebutuhan untuk sebuah sistem pencarian yang mengakomodasi penggunaan ulang semakin meningkat.
	Pertumbuhan adopsi blockchain dan peningkatan jumlah Smart Contracts di dalam ekosistem blockchain menciptakan tantangan-tantangan baru. Salah satu tantangan utamanya adalah untuk menggunakan kembali Smart Contracts yang telah ada yang sesuai dengan kebutuhan. Sistem pencarian Smart Contracts saat ini masih mengandalkan pencarian berbasis kata kunci yang kurang mempertimbangkan konteks semantik dari kontrak tersebut. Sehingga, pengguna masih sulit untuk menemukan Smart Contracts yang sesuai dengan kebutuhan mereka. Pada tugas akhir ini dikembangkan sebuah Smart Contract Discovery System berbasis konteks semantik yang memanfaatkan kemampuan LLM untuk mendeskripsikan Source Code dari sebuah kontrak secara semantik dan melakukan pengambilan data menggunakan kemiripan semantik. Sistem ini memanfaatkan Distributed Graph Database (Dgraph) untuk menyimpan data kontrak dan embeddings yang merepresentasikan kontrak tersebut. Sistem ini mengintegrasikan modul ekstraksi data, \textit{enrichment} data, pemodelan data, dan \textit{retrieval} data untuk meningkatkan relevansi hasil pencarian. Hasil evaluasi menunjukkan bahwa sistem ini mampu meningkatkan presisi dalam menemukan kontrak yang relevan dibandingkan dengan pendekatan tradisional. Dengan demikian, rancangan ini membuktikan efektivitas penggunaan konteks semantik dalam penemuan Smart Contracts pada ekosistem blockchain.

	\textbf{\textit{Kata kunci: Smart Contracts, Sistem Pencarian, Semantik, LLM, Enrichment, Embeddings, Retrieval}}

\end{singlespace}
\clearpage