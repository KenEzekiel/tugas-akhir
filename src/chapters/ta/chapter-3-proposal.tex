\chapter{Deskripsi Rancangan Penyelesaian Masalah}

Tujuan utama penulisan bab ini adalah untuk menguraikan rencana penyelesaian masalah implementasi dari judul TA

% Tujuan utama penulisan bab ini adalah untuk menguraikan rencana penyelesaian masalah tugas akhir I.. Bab ini mencakup antara lain: 
% 1.	Deskripsi dan analisis persoalan yang terkait dengan Rumusan Masalah, misalnya menjelaskan secara detail latar belakang dan masalah yang menjadi dasar munculnya topik, menunjukkan gap/celah antara kondisi saat ini dengan kondisi yang diharapkan, dan kaitan antara sistem/aplikasi yang dikembangkan dengan sistem/aplikasi lain yang terkait.
% 2.	Analisis solusi yang terdiri dari pilihan alternatif solusi yang dapat digunakan untuk setiap permasalahan berdasarkan hasil studi literatur atau survei, pemilihan solusi beserta justifikasinya.
% 3.	Deskripsi umum solusi yang dipilih, mencakup:
% a.	Modul/subsistem/komponen yang akan dikembangkan untuk menyelesaikan masalah, berikut penjelasannya.
% b.	Alur umum algoritma atau langkah-langkah pengembangan sistem dan penjelasannya.
% c.	Penggunaan kakas yang diperlukan
% Dianjurkan untuk menggunakan diagram sebagai pendukung penjelasan bagian ini.


\section{Analisis}
\blindtext

\section{Rancangan}
\blindtext

preliminary analysis

gambaran solusi
% kalau ada sejumlah pilihan, kajiannya baru berdasarkan survey dan studi literatur, alternatif solusinya diberikan gambaran masing-masing

% menjelaskan secara lebih detail latar belakang dan masalah yang menjadi dasar munculnya topik TA ini, intinya kita coba lihat & analisis gapnya 
% gap analysis
% kaitan antara sistem yang dikembangkan dengan yang terkait -> apa kelebihannya? atau apa kekurangan dari aplikasi lain? emang belum terpenuhi? apa yang belum terpenuhi?
% posisi sistem yang dikembangkan terhadap sistem yang lebih besar
% alur umum dari algoritma spesifik yang ada di dalam solusi