\subsection{Ontological Modeling of Smart Contracts in Solidity}
\label{subsec:solidity-ontology}

Formalisasi menggunakan ontologi berbasis domain baru dilakukan pada blockchain dan Smart Contract, tetapi belum ada yang dilakukan terhadap bahasa pemrograman Smart Contract itu sendiri. Pada penelitiannya, \cite{cano2021toward} mengusulkan sebuah representasi dari bahasa pemrograman Smart Contract yang terkemuka, yaitu Solidity, dengan mendefinisikan semua entitas yang dibutuhkan untuk mencakup seluruh bahasa dan menyelaraskan ke ontologi terstandarisasi lainnya seperti EthOn.

% Beberapa spesifikasi yang dituliskan di dalam penelitian ini:

% \begin{enumerate}
% 	\item Implementasi Solidity Library
% 	\item Implementasi Solidity Contract
% 	\item Interface dan Abstract Contract Solidity
% 	\item Spesifikasi Attributes dan representasinya di dalam Ontology
% 	\item Spesifikasi Types dan representasinya di dalam Ontology
% 	\item Spesifikasi Constructor dan representasinya di dalam Ontology
% 	\item Spesifikasi Function dan representasinya di dalam Ontology
% 	\item Spesifikasi Modifier dan representasinya di dalam Ontology
% 	\item Spesifikasi Receive dan representasinya di dalam Ontology
% 	\item Spesifikasi Fallback dan representasinya di dalam Ontology
% 	\item Spesifikasi Event dan representasinya di dalam Ontology
% \end{enumerate}