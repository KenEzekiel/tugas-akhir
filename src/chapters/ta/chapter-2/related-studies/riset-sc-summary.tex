\subsection{Smart Contracts and Solidity Code Summarization}
\label{subsec:smart-contract-solidity-summary}

Penelitian yang dilakukan oleh \cite{zhang2021smart} membahas tantangan dalam pemrograman dan pemahaman kode Smart Contracts, khususnya pada platform Ethereum, yang sebagian besar menggunakan bahasa pemrograman Solidity. Masalah utama yang dihadapi adalah kurangnya dokumentasi pada kode Smart Contracts, yang menghambat pemahaman bagi pengembang, terutama yang masih pemula dalam pengembangan Smart Contracts. Dalam penelitian ini, penulis mengusulkan sistem yang dinamakan SoliditySummarizer untuk menghasilkan komentar otomatis untuk kode Solidity, guna membantu pengembang, baik yang berpengalaman maupun pemula, dalam memahami dan mendokumentasikan kode Smart Contracts secara lebih mudah.

Pada bagian pertama dari tesis ini, penulis menjelaskan tentang Blockchain dan Ethereum, dengan fokus pada kemampuan Ethereum untuk menjalankan Smart Contracts yang otomatis, terdesentralisasi, dan dapat dieksekusi tanpa pengawasan pihak ketiga. Penelitian ini juga menggali masalah terkait \textit{readability} dan dokumentasi yang sering kali tidak ada pada kode Smart Contracts, serta bagaimana masalah ini dapat diselesaikan dengan menggunakan SoliditySummarizer.

SoliditySummarizer didesain untuk menghasilkan ringkasan dari kode sumber Smart Contracts, dengan menggunakan pendekatan \textit{transformer models} dalam teknik \textit{Natural Language Generation} (NLG). Salah satu elemen penting dalam penelitian ini adalah pembuatan dataset yang digunakan untuk melatih model, yang dikumpulkan dan dibersihkan dari kontrak-kontrak yang terdapat di Etherscan dan GitHub. Selanjutnya, sistem ini menggunakan model berbasis \textit{Transformer} untuk menghasilkan deskripsi yang dapat membantu pengembang memahami fungsi dan struktur kode Solidity.

Dalam bab kedua, penulis mengulas \textit{state of the art} dalam teknologi \textit{code summarization} untuk Smart Contracts, serta menyoroti tantangan yang dihadapi saat menerapkan \textit{machine learning} dalam \textit{summarization} kode, khususnya dalam bahasa Solidity. SMTranslator adalah salah satu alat yang dikaji sebagai perbandingan, namun alat ini memiliki keterbatasan dalam hal akurasi dan fleksibilitas dalam menghasilkan deskripsi yang sesuai dengan kompleksitas Smart Contracts.

SoliditySummarizer menggunakan teknik yang lebih canggih, termasuk \textit{multi-head self-attention networks} dan penggunaan \textit{pre-trained models} untuk mengatasi kekurangan data pelatihan yang terbatas. Hasil evaluasi menunjukkan bahwa SoliditySummarizer lebih unggul dibandingkan alat sebelumnya, terutama dalam hal \textit{accuracy} dan \textit{conciseness}, meskipun masih ada tantangan dalam \textit{readability} dari komentar yang dihasilkan.

Penelitian ini menekankan bahwa meskipun SoliditySummarizer memberikan hasil yang lebih baik dibandingkan dengan metode berbasis template sebelumnya, tantangan terbesar terletak pada \textit{readability} komentar yang dihasilkan, yang cenderung terlalu teknis dan sulit dipahami bagi \textit{non-programmer}. SoliditySummarizer tetap memberikan kontribusi besar dalam meningkatkan pemahaman dan penggunaan Smart Contracts di platform Ethereum, serta membuka kemungkinan penggunaan \textit{machine learning} untuk meningkatkan dokumentasi otomatis di masa depan.