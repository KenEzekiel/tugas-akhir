\subsection{Standard Smart Contract}
\label{subsec:standard-sc}

Smart Contract Standards adalah kumpulan pedoman dan spesifikasi yang perlu diikuti oleh sebuah Smart Contracts agar dapat beroperasi secara konsisten dan kompatibel di seluruh jaringan. Di dalam Ethereum, standar tersebut didefinisikan dalam bentuk Ethereum Request for Comments (ERC), yang merupakan seperangkat standar teknis yang digunakan untuk mengatur cara pengembangan Smart Contract. ERC diperkenalkan oleh Fabian Vogelsteller dan Vitalik Buterin pada tahun 2015, dan ERC-20 adalah salah satu standar yang paling terkenal dan banyak digunakan di dalam Ethereum \parencite{vogelsteller2015erc20}.

ERC memastikan bahwa token, Smart Contract, dan dApps dapat berinteraksi secara konsisten dan \textit{interoperable} di seluruh ekosistem Ethereum. Contohnya, ERC-20 adalah standar yang mendefinisikan antarmuka untuk token Ethereum, yang memungkinkan pengguna untuk mengirim dan menerima token, memeriksa saldo token, dan melihat riwayat transaksi token. ERC-20 juga memastikan bahwa token yang dibuat oleh pengembang dapat beroperasi dengan aman dan dapat diintegrasikan dengan berbagai aplikasi dan layanan lainnya di dalam ekosistem Ethereum.

Terdapat kakas-kakas seperti OpenZeppelin Contracts yang menyediakan implementasi siap pakai untuk standar-standar seperti ERC-20, ERC-721, dan lainnya yang dapat diimpor langsung ke dalam kode Solidity \parencite{openzeppelin_contracts_5x}. Saat menggunakan kakas, \textit{source code} dari \textit{file} Solidity yang diimpor dari kakas akan disalin ke dalam proyek pengguna saat proses kompilasi. Proses ini mengambil \textit{source code} yang telah disediakan oleh kakas melalui repositori atau \textit{package manager}. Secara lebih lengkap, terdapat dua mode penggunaan kakas, yaitu \textit{internal} dan \textit{external}. Mode \textit{internal} digunakan ketika fungsi di dalam kakas dideklarasikan dengan visibilitas `internal`, dimana fungsi akan disalin secara \textit{inline} ke Smart Contract pemanggil. Mode \textit{external} digunakan ketika fungsi di dalam kakas dideklarasikan dengan visibilitas `public` atau `external`, dimana kakas harus di-\textit{deploy} sebagai kontrak terpisah dan akan dilakukan \textit{linking} ke alamat kakas yang sudah di-\textit{deploy} \parencite{solidity_lowlevelcalls}.

Pemanggilan Smart Contract lainnya dalam suatu Smart Contract menyerupai pemanggilan kakas dengan mode \textit{external}, karena Smart Contract lainnya di-\textit{deploy} di dalam Blockchain. Perbedaan utamanya terletak pada cara pemanggilannya, dimana pemanggilan kakas \textit{external} menggunakan fungsi \texttt{delegatecall}, yang berarti kode kakas dijalankan dalam konteks \textit{state} dari Smart Contract pemanggil, sehingga segala perubahan \textit{state} yang terjadi di dalam fungsi kakas akan langsung mempengaruhi \textit{state} dari Smart Contract pemanggil, bukan \textit{state} dari kakas itu sendiri \parencite{rareskills_delegatecall}. Sedangkan pemanggilan Smart Contract lainnya menggunakan fungsi \texttt{call}, yang berarti eksekusi terjadi dalam konteks Smart Contract yang dipanggil, dan perubahan \textit{state} yang terjadi di dalam Smart Contract yang dipanggil tidak akan mempengaruhi \textit{state} dari Smart Contract pemanggil \parencite{rareskills_lowlevel}. Terdapat satu cara pemanggilan lainnya yaitu \texttt{staticcall}, yang memastikan bahwa fungsi yang dipanggil tidak melakukan modifikasi \textit{state}. Jika ada percobaan untuk mengubah \textit{state}, maka eksekusi akan gagal dan transaksi akan dibatalkan \parencite{rareskills_staticcall}.

Secara luas, penggunaan implementasi siap pakai untuk standar-standar maupun kode lain yang telah diuji dan diverifikasi dapat membantu pengembang untuk menghindari kesalahan dan kerentanan yang umumnya terjadi di dalam pengembangan Smart Contracts \parencite{consensys_duplication_reuse}.
% Penggunaan implementasi siap pakai untuk standar-standar maupun kode lain yang telah diuji dan diverifikasi dapat membantu pengembang untuk menghindari kesalahan dan kerentanan yang umumnya terjadi di dalam pengembangan Smart Contracts.