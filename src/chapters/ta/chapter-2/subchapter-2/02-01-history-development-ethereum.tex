\subsection{Sejarah dan Pengembangan Ethereum}
\label{subsec:sejarah-pengembangan-ethereum}

Ethereum adalah sebuah \textit{platform} berbasis blockchain yang berfokus pada pembangunan aplikasi terdesentralisasi dan Smart Contracts, dengan serangkaian \textit{tradeoffs} yang berbeda.
Dikembangkan oleh Vitalik Buterin pada tahun 2015, Ethereum memperluas konsep dasar blockchain yang sebelumnya hanya terbatas untuk transaksi. Ethereum melakukannya dengan membangun sebuah \textit{layer} fondasi dalam bentuk sebuah blockchain dengan \textit{built-in Turing-complete programming language}, yaitu Solidity, yang memungkinkan pengembang untuk membangun aplikasi terdesentralisasi yang kompleks dan Smart Contracts yang menjalankan logika dan sistem di dalam blockchain \parencite{buterin2013ethereum}.

Secara teknis, perbedaan utama Ethereum dengan Bitcoin adalah perbedaan dari jenis data yang disimpan di dalam blockchain. Bitcoin hanya menyimpan data kepemilikan mata uang, sedangkan Ethereum menyimpan \textit{state transitions} dari \textit{data store} yang lebih umum. Ethereum memiliki \textit{memory} yang dapat menyimpan data dan kode, dan blockchain Ethereum dapat digunakan untuk melacak bagaimana perubahan dari \textit{memory} dari waktu ke waktu. Seperti layaknya \textit{general-purpose computers}, Ethereum dapat menjalankan kode ke dalam \textit{state machine} yang dimilikinya, dan menyimpan hasilnya ke dalam blockchain. Perbedaan utama Ethereum dengan kebanyakan \textit{general-purpose computers} adalah mekanisme perubahan \textit{state} di dalam Ethereum yang diatur oleh rangkaian aturan di dalam konsensus, dan distribusi global dari \textit{state} \parencite{antonopoulos2018mastering}.

% Bahasa pemrograman bawaan Ethereum yang \textit{Turing-complete} adalah Solidity, yang digunakan untuk menulis Smart Contracts di Ethereum. 
% Solidity adalah bahasa dengan paradigma berorientasi objek, seluruh program Smart Contract akan dikompilasi menjadi sebuah \textit{Bytecode}, dalam kasus ini menjadi bentuk EVM Bytecode, dan spesifikasinya dituliskan pada sebuah Application Binary Interface (ABI). Dengan bahasa pemrograman yang \textit{Turing-complete} dan mekanisme eksekusi yang terjamin dengan EVM, Ethereum memberikan kemampuan bagi pengembang untuk mengembangkan sebuah aplikasi yang terdesentralisasi, yang disebut juga sebagai \textit{dApps}, di mana kode dan data disimpan di dalam Blockchain, sehingga mendapatkan karakteristik bawaan dari karakteristik Blockchain \parencite{buterin2013ethereum}.

% , dengan menambahkan sebuah bahasa pemrograman \textit{Turing-complete} untuk mengembangkan Smart Contracts yang menjalankan logika dan sistem di dalam Blockchain. Ethereum melakukannya dengan membangun sebuah fondasi 

