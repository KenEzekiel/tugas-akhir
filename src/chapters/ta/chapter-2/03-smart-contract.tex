\section{Smart Contracts}
\label{sec:smart-contract}

\subsection{Konsep Dasar}
\label{subsec:konsep-dasar}

Smart Contracts adalah sebuah protokol transaksi elektronik yang mengeksekusi kesepakatan dari sebuah kontrak. Klausa kesepakatan yang dimasukkan ke dalam sebuah Smart Contract akan diberlakukan secara otomatis saat kondisi yang sesuai sudah tercapai. Sehingga, suatu pihak yang melanggar kontrak akan dihukum secara otomatis. Smart Contract adalah sebuah cara untuk meminimalisir kepercayaan kepada perantara pihak ketiga sebagai \textit{enforcer} dari sebuah kontrak \parencite{szabo1997formalizing}.

Smart Contracts adalah salah satu teknologi yang dimungkinkan oleh teknologi blockchain. Seluruh klausa kontraktual dalam sebuah Smart Contract akan dikonversi menjadi sebuah bentuk \textit{executable computer programs}. Seluruh eksekusi dari setiap \textit{contract statement} direkam dan dimasukkan ke dalam transaksi yang \textit{immutable}, yang disimpan di dalam blockchain. Smart Contracts juga dapat menjamin \textit{access control} yang tepat dan \textit{contract enforcement} yang deterministik, karena dijamin dalam seluruh \textit{logic} yang terdapat di dalam Smart Contract tersebut \parencite{zheng2020overview}.



\subsection{Struktur Smart Contract}
\label{subsec:struktur-smart-contract}

Smart Contract di dalam Solidity serupa dengan \textit{class} di dalam pemrograman berbasis objek. Struktur dari Smart Contract tersusun dari \parencite{solidity_structure}:

\begin{enumerate}
	\item \textbf{State Variables} \newline
	      Variabel yang datanya disimpan secara permanen di \textit{contract storage} atau secara sementara di \textit{transient storage} yang dibersihkan di akhir setiap transaksi.
	\item \textbf{Functions} \newline
	      Unit kode berisi logika bisnis yang dapat dieksekusi, dapat didefinisikan di dalam atau di luar \textit{contract}.
	\item \textbf{Modifiers} \newline
	      Mekanisme untuk mengontrol akses atau mengimplementasikan logika tambahan sebelum atau sesudah eksekusi.
	\item \textbf{Events} \newline
	      Mekanisme \textit{logging} yang memungkinkan Smart Contract untuk mengeluarkan informasi ke log Blockchain, yang dapat dipantau oleh aplikasi eksternal.
	\item \textbf{Errors} \newline
	      Mekanisme untuk mendefinisikan nama dan data untuk situasi kegagalan, yang dapat digunakan dalam \textit{revert statement}.
	\item \textbf{Struct Types} \newline
	      Definisi tipe yang dapat digunakan secara \textit{custom} untuk menyimpan data yang kompleks.
	\item \textbf{Enum Types} \newline
	      Definisi tipe yang dapat digunakan untuk membuat tipe data dengan nilai yang terbatas.
	\item \textbf{Constructors} \newline
	      Fungsi khusus yang hanya dipanggil sekali saat kontrak di-\textit{deploy} untuk menginisialisasi \textit{state} awal dari kontrak.
	\item \textbf{Fallback and Receive Functions} \newline
	      Fungsi khusus yang menangani transaksi yang dikirim langsung ke kontrak tanpa data, atau ketika fungsi yang dipanggil tidak ditemukan.
	\item \textbf{Library and Inheritance} \newline
	      Mekanisme \textit{inheritance} dan \textit{reusability} kode yang memungkinkan kontrak untuk mewarisi fungsi dan variabel dari kontrak lain untuk mengorganisir kode den meminimalisir duplikasi.
\end{enumerate}


\subsection{Life Cycle Smart Contract}
\label{subsec:lifecycle}

\begin{figure}[ht]
	\centering
	\includegraphics[width=0.7\textwidth]{resources/chapter-2/sc-lifecycle.png}
	\caption{\textit{Life cycle} dari Smart Contract \parencite{zheng2020overview}}
	\label{image:sc-lifecycle}
\end{figure}

\textit{Life cycle} dari sebuah Smart Contract terdiri dari empat fase seperti pada ilustrasi di Gambar \ref{image:sc-lifecycle}:

\begin{enumerate}
	\item \textit{Creation}: negosiasi antar pihak untuk menyepakati ketentuan dari kontrak dalam \textit{natural language}, dan translasi menjadi Smart Contracts.
	\item \textit{Deployment}: kontrak yang sudah divalidasi dapat disimpan ke dalam Blockchain, menjadikannya tidak bisa dimodifikasi.
	\item \textit{Execution}: Setelah \textit{deployment}, klausa kontraktual akan dimonitor, dan saat kondisi yang sesuai dengan yang terdefinisi dalam Smart Contract, maka prosedur kontrak akan dieksekusi secara otomatis.
	\item \textit{Completion}: Setelah eksekusi, \textit{state} baru dari semua pihak akan diperbarui sesuai dengan hasil dari transaksi yang terjadi dan disimpan ke dalam Blockchain.
\end{enumerate}

\subsection{Smart Contract Sanctuary}
\label{subsec:sc-sanctuary}

Smart Contract Sanctuary adalah sebuah repositori GitHub yang dikembangkan oleh tintinweb dengan tujuan untuk menyediakan kumpulan Smart Contracts yang aman, telah diuji, dan diverifikasi secara menyeluruh untuk digunakan atau dijadikan referensi sehingga dapat mengurangi resiko kerentanan dan bug pada tahap pengembangan Smart Contract \parencite{smart_contract_sanctuary}.

% \subsection{Off-Chain Smart Contracts}
% \label{subsec:off-chain-smart-contracts}

% \textit{Off-chain Smart Contract} adalah Smart Contracts yang dieksekusi diluar Blockchain, \textit{signed} hanya oleh \textit{interested participants}, dan digunakan untuk mengenkapsulasi fungsi yang melibatkan komputasi \textit{high-cost} atau \textit{private information} terkait \textit{participants}. Terdapat banyak cara untuk tetap menjaga properti dan keuntungan penggunaan dari Blockchain, contohnya, hasil dari eksekusi sebuah \textit{off-chain Smart Contract} dapat dilakukan \textit{logging} pada Blockchain, sehingga jika terjadi \textit{dispute} dalam eksekusi \textit{off-chain Smart Contract}, sebuah \textit{on-chain Smart Contract} dapat digunakan untuk \textit{fork off-chain Smart Contract} dan mengeksekusinya di dalam Blockchain untuk menyelesaikan \textit{dispute} \parencite{zou2019smart}.

\subsection{Semantic Smart Contracts}
\label{subsec:semantic-smart-contracts}
Semantic Smart Contracts adalah sebuah cara untuk merepresentasikan \textit{semantics} dari Smart Contract menggunakan konsep \textit{EthOn contract extension} dan sebuah \textit{vocabulary} yang terkait dengan bisnis. Semantic Smart Contracts mengizinkan untuk membandingkan \textit{request} dengan beberapa \textit{request description} dengan mengonsiderasi semantik dari anotasi yang mereferensikan sebuah \textit{shared domain ontology} \parencite{baqa2019semantic}.
