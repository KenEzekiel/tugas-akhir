\subsection{Blockchain Properties}
\label{subsec:blockchain-properties}

Implementasi dari Blockchain memunculkan karakteristik dari Blockchain itu sendiri. Karakteristik ini dapat muncul secara inheren dari sistem dasar dimana Blockchain di bangun, ataupun muncul karena implementasi spesifik Blockchain. Beberapa karakteristik penting Blockchain \parencite{aimar2023extraction}:

\begin{itemize}
	\item \textit{Decentralized}: Tidak ada sebuah entitas terpusat yang mengontrol jaringan. Seluruh \textit{participants} mengikuti protokol yang berlaku dan memiliki kontrol yang sama.
	\item \textit{Distributed}: Komputasi dilakukan pada sejumlah \textit{node} atau komputer yang berbeda, yang tersebar dan saling berinteraksi melalui sebuah \textit{p2p network}. Kegagalan sebuah mesin seharusnya tidak mengganggu jalannya protokol.
	\item \textit{Immutable}: Tidak memungkinkan untuk mengubah \textit{history} apapun yang sudah tertulis di dalam Blockchain. Setelah sebuah blok divalidasi dan dimasukkan ke dalam Blockchain, tidak dapat dimodifikasi.
	\item \textit{Permissionless}: Semua orang dapat secara aktif berpartisipasi di dalam semua \textit{role} di dalam jaringan tanpa perlu meminta \textit{permission}.
	\item \textit{Permissioned}: Mewajibkan seluruh aktor di dalam jaringan mendapatkan \textit{authorization} secara eksplisit.
	\item \textit{Transparent}: Semua orang dapat secara independen melihat dan mengunduh data dari Blockchain.
	\item \textit{Pseudoanonymous}: \textit{Participants} dalam sebuah \textit{Blockchain network} tidak perlu membuktikan identitas asli mereka. Seluruh aktivitas di dalam jaringan akan disambungkan ke sebuah \textit{address}, bukan identitas asli seseorang.
	\item \textit{Account-based}: data disimpan berdasarkan akun, dan setiap akun memiliki \textit{balance} yang dapat digunakan. Kepemilikan sebuah akun dibuktikan dengan kepemilikan \textit{private key} untuk akun tersebut.
	\item \textit{UTXO-based}: Selain \textit{account-based}, model \textit{UTXO} hanya memiliki konsep dari transaksi. \textit{User} harus membuktikan bahwa mereka memiliki \textit{private key} untuk membuka kunci dari sebuah hasil transaksi untuk menggunakan \textit{balance} yang dimiliki. \textit{Balance} dari seorang \textit{user} adalah penjumlahan seluruh nilai dari hasil transaksi yang dapat dibuka dan digunakan.
\end{itemize}