\section{Blockchain}
\label{sec:blockchain}

Blockchain adalah sekuens dari blok yang menyimpan daftar transaksi lengkap seperti \textit{public ledger} konvensional, dimana setiap \textit{ledger} disimpan pada \textit{node} yang tersebar seperti pada \textit{Distributed Ledger Technology} \parencite{zheng2018blockchain}. Setiap transaksi yang masuk ke dalam sebuah blok akan divalidasi sesuai dengan metrik yang digunakan oleh sistem Blockchain, dan blok yang memiliki daftar transaksi yang lengkap, ditambah \textit{timestamp} pembuatan blok, nilai \textit{hash} dari blok sebelumnya ("\textit{parent}"), dan sebuah \textit{nonce}, yang adalah sebuah angka acak yang digunakan untuk mekanisme verifikasi \textit{hash}. Konsep ini memastikan integritas dari Blockchain dimulai dari blok pertama ("\textit{genesis block}") sampai ke blok terakhir, yang terus ditambahkan, karena setiap perubahan data akan membuat nilai \textit{hash} dari sebuah blok berubah, yang harus dipropagasikan ke setiap blok setelahnya. Sebelum ditambahkan ke dalam Blockchain, setiap blok dan transaksi di dalamnya harus divalidasi oleh mayoritas \textit{node}, menggunakan sebuah mekanisme \textit{consensus} \parencite{nofer2017blockchain}. Mekanisme \textit{consensus} adalah proses dimana mayoritas dari \textit{network validator} menyetujui atau menolak sebuah \textit{state} dari \textit{ledger}. Proses \textit{consensus} mengikuti sebuah kumpulan aturan dan prosedur untuk mempertahankan himpunan fakta yang koheren diantara beberapa \textit{participating nodes}. Terdapat banyak mekanisme \textit{consensus} yang berbeda yang digunakan dalam \textit{Blockchain network} yang berbeda. Dalam kasus \textit{Bitcoin}, \textit{ledger} yang dianggap \textit{ledger} yang \textit{valid} adalah \textit{ledger} dengan \textit{chain} terpanjang (\textit{longest chain}) \parencite{swanson2015consensus}.

\begin{figure}[ht]
	\centering
	\includegraphics[width=1\textwidth]{resources/chapter-2/struktur-blockchain.png}
	\caption{Struktur blok di dalam Blockchain \parencite{zheng2018blockchain}}
	\label{image:struktur-blockchain}
\end{figure}

\break

Struktur dari sebuah blok terdiri dari \textit{block header} dan \textit{block body} seperti pada Gambar \ref{image:struktur-blok}. Secara spesifik, \textit{block header} terdiri dari:

\begin{enumerate}
	\item \textit{Block Version}: mengindikasikan set dari aturan validasi yang diikuti.
	\item \textit{Parent Block Hash}: 256-bit \textit{hash} dari blok sebelumnya.
	\item \textit{Merkle Tree Root}: hasil \textit{hash} dari seluruh transaksi pada blok menggunakan mekanisme \textit{Merkle Tree}.
	\item \textit{Timestamp}: \textit{timestamp} saat ini dalam detik sejak 1970-01-01T00:00 UTC.
	\item \textit{nBits}: target \textit{hash} saat ini dalam format \textit{compact}.
	\item \textit{nonce}: \textit{number used only once}, sebuah angka yang digunakan untuk menambahkan tingkat keacakan dari nilai \textit{hash}.
\end{enumerate}

\begin{figure}[ht]
	\centering
	\includegraphics[width=0.7\textwidth]{resources/chapter-2/struktur-block.png}
	\caption{Struktur blok \parencite{zheng2018blockchain}}
	\label{image:struktur-blok}
\end{figure}

\input{chapters/ta/chapter-2/01-01-properties-blockchain}

\subsection{Ethereum}
\label{subsec:ethereum}

Ethereum adalah sebuah platform berbasis Blockchain untuk membangun aplikasi terdesentralisasi dan Smart Contracts, dengan serangkaian \textit{tradeoffs} yang berbeda. Dikembangkan oleh Vitalik Buterin pada tahun 2015, Ethereum mengembangkan konsep dasar Blockchain dengan menambahkan sebuah bahasa pemrograman \textit{Turing-complete} untuk mengembangkan Smart Contracts, yang dijalankan di dalam Ethereum Virtual Machine (EVM). Ethereum Virtual Machine (EVM) adalah sebuah lingkungan eksekusi untuk Smart Contracts di Ethereum, yang memungkinkan kode berjalan di seluruh jaringan Ethereum secara terdistribusi. Bahasa pemrograman bawaan Ethereum yang \textit{Turing-complete} adalah Solidity, yang digunakan untuk menulis Smart Contracts di Ethereum. Solidity adalah bahasa dengan paradigma berorientasi objek, seluruh program Smart Contract akan dikompilasi menjadi sebuah \textit{byte-code}, dalam kasus ini menjadi bentuk EVM Bytecode, dan spesifikasinya dituliskan pada sebuah Application Binary Interface (ABI). Dengan bahasa pemrograman yang \textit{Turing-complete} dan mekanisme eksekusi yang terjamin dengan EVM, Ethereum memberikan kemampuan bagi pengembang untuk mengembangkan sebuah aplikasi yang terdesentralisasi, yang disebut juga sebagai \textit{dApps}, dimana kode dan data disimpan di dalam Blockchain, sehingga mendapatkan karakteristik bawaan dari karakteristik Blockchain \parencite{buterin2013ethereum}.

\subsubsection{ABI}
\label{subsubsec:abi}

\begin{figure}[ht]
  \centering
  \includegraphics[width=0.7\textwidth]{resources/chapter-2/smart-contract-abi.jpg}
  \caption{Contoh ABI dari sebuah Smart Contract \parencite{third2017linked}}
  \label{image:abi-example}
\end{figure}

Application Binary Interface (ABI) adalah sebuah konvensi yang mendefinisikan aspek-aspek kode yang dihasilkan selama kompilasi, seperti representasi data, penggunaan register, dan konvensi pemanggilan fungsi \parencite{sciencedirect2024}. Dalam konteks Smart Contracts yang ditulis dengan bahasa Solidity dan dikompilasi, hasil dari kompilasinya akan berbentuk EVM Bytecode untuk dieksekusi di dalam EVM, disertai dengan ABI yang mendefinisikan Smart Contract tersebut, seperti pada gambar \ref{image:abi-example}.

\subsubsection{Decentralized Applications (dApps)}
\label{subsubsec:dapps}

Decentralized Applications (dApps) adalah sebuah aplikasi yang berjalan di atas infrastruktur Blockchain dengan memanfaatkan Smart Contracts untuk menyediakan fungsionalitasnya. Karena dApps berjalan di atas Blockchain, dApps mewarisi sifat-sifat yang inheren dari Blockchain, seperti terdesentralisasi, \textit{immutable}, transparan, dan sifat-sifat lainnya \parencite{investopedia2024}. dApps tidak berbeda dari aplikasi tradisional dari sisi pengguna, perbedaannya hanya terletak pada komputasi dari fungsionalitas yang diberikan oleh dApps dilakukan menggunakan Smart Contracts dan seluruh datanya terletak di dalam Blockchain \parencite{metcalfe2020ethereum}. 

\subsubsection{Etherscan}
\label{subsubsec:etherscan}

Etherscan adalah sebuah \textit{block explorer}, \textit{search engine} untuk pengguna agar dapat dengan mudah melihat, mengonfirmasi, dan memvalidasi transaksi untuk Blockchain Ethereum. Etherscan didirikan oleh Matthew Tan pada tahun 2015, dan berdiri secara independen dari Ethereum Foundation. Etherscan melakukan \textit{indexing} terhadap Blockchain untuk menampilkan informasinya, dimana informasi tersebut digunakan untuk menyediakan API untuk pengembang mengintegrasikan informasi Blockchain Ethereum ke dalam aplikasinya \parencite{etherscan2024}.

\subsection{Layer 2}
\label{subsec:layer-2}

Layer 2 adalah sebuah protokol sekunder yang dibangun di atas Blockchain yang sudah ada. Ide dasarnya adalah membangun sebuah \textit{framework} yang mengatasi transaksi \textit{off-chain} sehingga mereduksi beban dari Blockchain dan mendapatkan kecepatan transaksi yang lebih cepat \parencite{sguanci2021layer}. Terdapat beberapa jenis dari solusi Layer 2:

\begin{enumerate}
	\item \textit{Channels L2}: Membuat sebuah komunikasi \textit{off-chain} dengan \textit{node} lain, baik secara langsung maupun tidak langsung. Transaksi antar \textit{node} yang terhubung akan dikelola di Layer 2, dan hanya melaporkan pada \textit{main chain} untuk pembukaan dan penutupan \textit{channel}.
	\item \textit{Sidechains L2}: Sebuah Blockchain "anak" yang berjangkar pada \textit{main chain} dan berjalan secara paralel dengan \textit{main chain}. Serupa dengan Channels, tetapi dalam Sidechains, \textit{off-chain transaction} berjalan di atas Blockchain juga.
	\item \textit{Rollups L2}: Hanya mengeksekusi \textit{transaction} secara \textit{off-chain}, tetapi selalu melaporkan data tentang \textit{transaction} ke dalam \textit{main chain}. Dibandingkan kedua jenis lainnya, Rollups melaporkan data yang lebih kecil untuk setiap \textit{off-chain state update}.
\end{enumerate}

\newpage