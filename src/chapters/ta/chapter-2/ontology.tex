\section{Ontology}
\label{sec:ontology}

Pada tahun 1993, \cite{gruber1993translation} pertama mendefinisikan sebuah \textit{ontology} sebagai "\textit{explicit specification of a conceptualization}". Pada tahun 1997, \cite{borst1997construction} mendefinisikan \textit{ontology} sebagai "\textit{formal specification of a shared conceptualization}". Definisi ini menambahkan kebutuhan sebuah \textit{ontology} sebagai sebuah representasi konseptual yang \textit{shared} diantara beberapa pihak. Sehingga, konseptualisasi tersebut harus diekspresikan dengan sebuah format yang \textit{machine readable}. Sehingga pada tahun 1998, \cite{studer1998knowledge} menggabungkan kedua definisi tersebut sebagai "\textit{an ontology is a formal, explicit specification of a shared conceptualization}."

Dalam tulisannya, \cite{Guarino2009} merangkum ketiga definisi tersebut menjadi: \textit{Ontology} adalah sebuah \textit{framework} terstruktur untuk merepresentasikan pengetahuan di sebuah \textit{domain} tertentu, seperti mendefinisikan entitas, konsep, dan relasi di dalam \textit{domain} tersebut. \textit{Ontology} dapat membantu membuat sebuah model yang dapat dimengerti dan dapat dibagikan untuk digunakan dalam berbagai sistem dan aplikasi. Di dalam \textit{domain} sistem informasi, \textit{ontology} digunakan sebagai sebuah cara formal untuk mengorganisasikan dan merepresentasikan data untuk dibagikan, dilakukan pencarian, dan dilakukan penalaran \parencite{Guarino2009}. 

Beberapa komponen kunci dari sebuah \textit{ontology} adalah:

\begin{enumerate}
  \item \textit{Classes (Concepts)}: Tipe atau kategori fundamental di dalam sebuah \textit{domain} yang merepresentasikan konsep umum.
  \item \textit{Instances (Individuals)}: Contoh spesifik dari sebuah \textit{class}.
  \item \textit{Properties (Attributes)}: Mendeskripsikan karakteristik dari \textit{classes} atau \textit{instances}.
  \item \textit{Relationships}: Hubungan antar entitas, bisa \textit{hierarchical} atau \textit{associative}.
  \item \textit{Axioms}: Aturan atau \textit{constraint} yang mendefinisikan \textit{valid relationships} dan \textit{properties} di dalam \textit{ontology}, sehingga dapat dilakukan inferensi logis.
\end{enumerate}

\subsection{The Semantic Web}
\label{subsec:the-semantic-web}

\textit{The Semantic Web} adalah sebuah ekstensi dari \textit{web} dengan tambahan data dengan arti yang terdefinisi dengan baik, yang diturunkan menggunakan \textit{semantic theory} untuk menginterpretasikan simbol-simbol. \textit{Semantic theory} menyediakan catatan dari arti untuk sebuah istilah atau informasi, sehingga dapat membuat sebuah hubungan logis antar istilah \parencite{shadbolt2006semantic}.

\subsubsection{Universal Resource Identifiers}
\label{subsubsec:universal-resource-identifiers}

\textit{Universal Resource Identifiers (URIs)} adalah sebuah cara untuk mengidentifikasi sebuah \textit{resource} menggunakan sebuah konvensi penamaan global yang disepakati, sehingga dapat diinterpretasikan secara standar oleh seluruh mesin yang berada di \textit{web}. Saat sebuah \textit{resource} diasosiasikan dengan sebuah URI, maka itu berarti semua orang dapat melakukan \textit{link}, \textit{refer}, dan mengambil \textit{representasi} dari \textit{resource} tersebut menggunakan URI. Skema yang direkomendasikan pada tahun 2004 adalah \textit{Resource Definition Framework Schema} yang menggunakan spesifikasi dasar dari RDF dengan ekstensi untuk mendukung ekspresi dari \textit{structured vocabularies} \parencite{shadbolt2006semantic}.

\subsubsection{Web Ontology Langugage}
\label{subsubsec:web-ontology-language}

\textit{Web Ontology Language (OWL)} adalah keluarga bahasa yang dikembangkan oleh \textit{World Wide Web Consortium (W3C)}, digunakan untuk mengembangkan dan berbagi \textit{ontology} di dalam \textit{web}. OWL bertujuan untuk memberikan representasi yang efisien untuk \textit{ontology}, pengecekan \textit{logical consistency} dan klasifikasi \textit{concept}, menggunakan RDF untuk \textit{linking} sehingga \textit{ontology} dapat didistribusikan antar sistem \parencite{shadbolt2006semantic}.

\subsection{Ontology in Blockchain}
\label{subsec:ontology-in-blockchain}

\subsection{BLONDiE}
\label{subsec:blondie}

\subsection{Minimal Service Model}
\label{subsec:minimal-service-model}