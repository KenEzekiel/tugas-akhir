\section{Ontology}
\label{sec:ontology}

Pada tahun 1993, \cite{gruber1993translation} pertama mendefinisikan sebuah \textit{ontology} sebagai "\textit{explicit specification of a conceptualization}". Pada tahun 1997, \cite{borst1997construction} mendefinisikan \textit{ontology} sebagai "\textit{formal specification of a shared conceptualization}". Definisi ini menambahkan kebutuhan sebuah \textit{ontology} sebagai sebuah representasi konseptual yang \textit{shared} diantara beberapa pihak. Sehingga, konseptualisasi tersebut harus diekspresikan dengan sebuah format yang \textit{machine readable}. Sehingga pada tahun 1998, \cite{studer1998knowledge} menggabungkan kedua definisi tersebut sebagai "\textit{an ontology is a formal, explicit specification of a shared conceptualization}."

Dalam tulisannya, \cite{Guarino2009} merangkum ketiga definisi tersebut menjadi: \textit{Ontology} adalah sebuah \textit{framework} terstruktur untuk merepresentasikan pengetahuan di sebuah \textit{domain} tertentu, seperti mendefinisikan entitas, konsep, dan relasi di dalam \textit{domain} tersebut. \textit{Ontology} dapat membantu membuat sebuah model yang dapat dimengerti dan dapat dibagikan untuk digunakan dalam berbagai sistem dan aplikasi. Di dalam \textit{domain} sistem informasi, \textit{ontology} digunakan sebagai sebuah cara formal untuk mengorganisasikan dan merepresentasikan data untuk dibagikan, dilakukan pencarian, dan dilakukan penalaran \parencite{Guarino2009}. 
