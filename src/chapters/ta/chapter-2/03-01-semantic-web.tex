\subsection{The Semantic Web}
\label{subsec:the-semantic-web}

\textit{The Semantic Web} adalah sebuah ekstensi dari \textit{web} dengan tambahan data dengan arti yang terdefinisi dengan baik, yang diturunkan menggunakan \textit{semantic theory} untuk menginterpretasikan simbol-simbol. \textit{Semantic theory} menyediakan catatan dari arti untuk sebuah istilah atau informasi, sehingga dapat membuat sebuah hubungan logis antar istilah \parencite{shadbolt2006semantic}.

\subsubsection{Universal Resource Identifiers}
\label{subsubsec:universal-resource-identifiers}

\textit{Universal Resource Identifiers (URIs)} adalah sebuah cara untuk mengidentifikasi sebuah \textit{resource} menggunakan sebuah konvensi penamaan global yang disepakati, sehingga dapat diinterpretasikan secara standar oleh seluruh mesin yang berada di \textit{web}. Saat sebuah \textit{resource} diasosiasikan dengan sebuah URI, maka itu berarti semua orang dapat melakukan \textit{link}, \textit{refer}, dan mengambil \textit{representasi} dari \textit{resource} tersebut menggunakan URI. Skema yang direkomendasikan pada tahun 2004 adalah \textit{Resource Definition Framework Schema} yang menggunakan spesifikasi dasar dari RDF dengan ekstensi untuk mendukung ekspresi dari \textit{structured vocabularies} \parencite{shadbolt2006semantic}.

\subsubsection{Web Ontology Langugage}
\label{subsubsec:web-ontology-language}

\textit{Web Ontology Language (OWL)} adalah keluarga bahasa yang dikembangkan oleh \textit{World Wide Web Consortium (W3C)}, digunakan untuk mengembangkan dan berbagi ontology di dalam \textit{web}. OWL bertujuan untuk memberikan representasi yang efisien untuk ontology, pengecekan \textit{logical consistency} dan klasifikasi \textit{concept}, menggunakan RDF untuk \textit{linking} sehingga ontology dapat didistribusikan antar sistem \parencite{shadbolt2006semantic}.