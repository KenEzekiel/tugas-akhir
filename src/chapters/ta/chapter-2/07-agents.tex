\section{Agents}
\label{sec:agents}

Agents dalam bidang Artificial Intelligence (AI) adalah entitas yang dapat bertindak secara mandiri untuk mencapai tujuan tertentu. Dalam implementasinya, penggunaan agents digolongkan sebagai Agentic AI, dimana sistem AI dapat beroperasi secara mandiri dan melakukan tindakan yang kompleks tanpa intervensi manusia. Berikut merupakan komponen-komponen yang terdapat di dalam sistem Agentic AI:

\begin{enumerate}
	\item Goal: Sasaran yang ingin dicapai oleh agent, bisa berupa aksi spesifik (navigasi ke suatu lokasi) atau tujuan kompleks (optimasi suatu proses).
	\item Environment: Konteks di mana agent beroperasi, dapat berupa ruang fisik, lingkungan digital, atau konteks sosial. Lingkungan ini menyediakan informasi dan \textit{feedback} atas tindakan agent.
	\item Perception: Kemampuan agent untuk merasakan dan memahami lingkungan, baik melalui persepsi visual maupun bentuk lain seperti log, email, dokumen, atau data teks lainnya.
	\item Reasoning: Proses agent dalam menganalisis informasi dan membuat inferensi berdasarkan pengetahuan serta pengalamannya.
	\item Planning: Proses menentukan urutan tindakan terbaik untuk mencapai tujuan agent, meliputi evaluasi alternatif dan pemilihan opsi yang paling sesuai.
	\item Action \& Tool Use: Aksi fisik atau virtual yang dilakukan agent untuk mencapai tujuannya. Termasuk kemampuan berinteraksi dengan sistem eksternal (\textit{API, database, search engine}) agar agent dapat mempengaruhi lingkungan dan memperkaya informasinya.
	\item Memory: Kemampuan menyimpan dan mengambil kembali informasi tentang pengalaman, pengetahuan, dan tindakan masa lalu. Sangat penting untuk proses pembelajaran dan peningkatan kinerja.
	\item Learning: Kemampuan agent untuk beradaptasi dan meningkatkan kinerja seiring waktu, misalnya dengan memperbarui basis pengetahuan dan menyempurnakan proses pengambilan keputusan dari pengalaman sebelumnya.
	\item Communication: Kemampuan agent untuk berinteraksi dengan agent lain atau manusia, termasuk berbagi informasi, koordinasi tindakan, dan kolaborasi untuk mencapai tujuan bersama.
\end{enumerate}