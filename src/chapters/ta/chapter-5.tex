\chapter{Penutup}

Bab Kesimpulan dan Saran akan menjadi bagian akhir dan penutup dari penelitian tugas akhir ini. Bagian ini akan merangkum ketercapaian dari penelitian terhadap ukuran keberhasilan yang telah ditetapkan dan memberikan saran-saran yang dapat digunakan untuk mengembangkan penelitian ini lebih lanjut.

\section{Kesimpulan}

Berdasarkan keseluruhan proses penelitian, pengembangan sistem, perbaikan, pengujian, dan analisis hasil yang dilakukan pada penelitian tugas akhir ini, dapat ditarik beberapa kesimpulan sebagai berikut:

\begin{enumerate}
    \item Penelitian tugas akhir ini telah berhasil mengembangkan sebuah sistem pencarian Smart Contracts berbasis semantik dengan pendekatan LLM \textit{enrichment}.
    \item Sistem ini telah dievaluasi dengan menggunakan perbandingan dengan sistem pencarian berbasis teks pada \textit{source code} dan deskripsi semantik yang dihasilkan. Hasil pengujian menunjukkan bahwa pencarian pada deskripsi semantik memberikan peningkatan relevansi hasil pencarian dan pencarian berbasis semantik memberikan hasil yang terbaik dibandingkan dengan metode pencarian lainnya. Sistem pencarian berbasis semantik ini diimplementasikan dengan analisis \textit{source code} oleh LLM dan \textit{enrichment} data yang difasilitasi dengan prompt dan embeddings yang telah dikembangkan. Poin ini menjawab rumusan masalah pertama terkait metode \textit{enrichment} yang dapat memberikan pengertian semantik pada data.
    \item Sistem ini memiliki mengambil data dari blockchain Ethereum mainnet, menyimpan data tersebut dalam format yang telah ditentukan, melakukan \textit{enrichment} dengan LLM, menyimpan data yang sudah enriched dan melakukan pencarian berbasis semantik menggunakan cosine similarity pada embeddings yang dihasilkan atau teks deskripsi yang dihasilkan. Deskripsi semantik yang dihasilkan dari sistem ini dapat disesuaikan dengan mengubah format yang digunakan untuk menyimpan data dan prompt yang digunakan untuk melakukan \textit{enrichment}. Sehingga, sistem ini dapat dengan mudah di-\textit{extend} untuk meningkatkan relevansi hasilnya. Selain itu, sistem ini telah diuji dan dibandingkan dengan pencarian berbasis teks dan mendapatkan hasil presisi yang lebih baik. Poin ini menjawab rumusan masalah kedua terkait metode \textit{data retrieval} berbasis semantik yang efektif. 
    \item Fungsionalitas sistem ini untuk menerima input berupa kebutuhan pengguna dalam bahasa alami dan menghasilkan daftar Smart Contract yang relevan telah berhasil diimplementasikan. Sistem ini dapat menerima input berupa bahasa alami, kata kunci, atau teks deskripsi yang relevan dengan Smart Contracts yang dicari. Sistem ini juga dapat memberikan daftar hasil pencarian yang relevan dengan kebutuhan pengguna. Poin ini menjawab rumusan masalah ketiga terkait fungsionalitas sistem Smart Contract Discovery.
    \item Format semantik yang digunakan tersedia pada lampiran dari laporan ini. Format ini dapat mengakomodasi berbagai jenis fungsionalitas dan juga domain yang berbeda. Poin ini adalah pemenuhan tujuan pertama dari tugas akhir ini.
    \item Teknik \textit{semantic enrichment} yang digunakan adalah teknik dan prompt yang telah dikembangkan. Teknik yang digunakan adalah dengan menggunakan LLM untuk menganalisis \textit{source code} Smart Contract dan melakukan \textit{enrichment} pada data berdasarkan prompt yang disediakan. Prompt ini tersedia pada lampiran dari laporan ini. Poin ini adalah pemenuhan tujuan kedua dari tugas akhir ini.
    \item \textit{Data retrieval} yang digunakan adalah pencarian embeddings yang disediakan oleh Dgraph. Pencarian ini dilakukan dengan menggunakan cosine similarity pada embeddings yang dihasilkan dari proses \textit{enrichment}. Poin ini adalah pemenuhan tujuan ketiga dari tugas akhir ini.
    \item Mekanisme ektraksi data dan konversi menjadi format yang dapat digunakan oleh sistem telah berhasil diimplementasikan memanfaatkan eth2dgraph. Mekanisme ini dilakukan dengan menggunakan Archive Node untuk mendapatkan data dan Dgraph untuk menyimpan data dan melakukan query terhadap data yang disimpan. Poin ini adalah pemenuhan tujuan keempat dari tugas akhir ini.
    \item Sistem Smart Contract Discovery telah berhasil diimplementasikan dan dapat digunakan untuk mencari Smart Contracts yang relevan dengan kebutuhan pengguna. Sistem ini dapat diakses melalui API dan Web GUI yang telah disediakan. Poin ini adalah pemenuhan tujuan kelima dari tugas akhir ini.
\end{enumerate}
 
\section{Saran}
Meskipun sistem Smart Contract Discovery berbasis semantik ini telah berhasil dikembangkan, terdapat beberapa saran yang dapat diberikan untuk pengembangan penelitian ini lebih lanjut. Saran-saran ini berisi kekurangan sistem yang teridentifikasi serta peluang pengembangan yang dapat dilakukan untuk meningkatkan sistem ini.

\begin{enumerate}
    \item Format semantik yang digunakan masih dapat diperbaiki untuk meningkatkan relevansi hasil pencarian. Format yang digunakan saat ini masih cukup umum dan dapat diperluas untuk mengakomodasi fungsionalitas yang lebih spesifik. Pengembangan lebih lanjut dapat dilakukan untuk mengembangkan format semantik yang lebih spesifik secara domain dan relevan dengan kebutuhan pengguna.
    \item Prompt yang digunakan untuk melakukan \textit{enrichment} masih dapat diperbaiki untuk meningkatkan relevansi hasil pencarian. Prompt yang digunakan saat ini masih cukup umum dan dapat diperluas untuk mengakomodasi fungsionalitas yang lebih spesifik. Pengembangan lebih lanjut dapat dilakukan untuk mengembangkan prompt yang lebih spesifik secara domain dan relevan dengan kebutuhan pengguna.
    \item Model LLM yang digunakan saat ini adalah model OpenAI GPT4o mini, dimana sudah terdapat model-model yang lebih baik dan lebih murah yang dapat digunakan untuk melakukan \textit{enrichment}. Seiring model LLM yang lebih baik dan lebih murah berkembang, sistem ini dapat memanfaatkan perkembangan tersebut untuk meningkatkan relevansi hasil pencarian dan mengurangi biaya yang dikeluarkan untuk melakukan \textit{enrichment}.
    \item Pencarian berbasis semantik yang dilakukan masih terbatas pada kemampuan pengguna untuk memberikan query bahasa alami atau kata kunci yang sesuai dengan kebutuhan. Pada kenyataannya, tidak semua pengguna memiliki kemampuan tersebut. Terdapat peluang untuk mengembangkan sistem pencarian berbasis semantik yang dapat memahami kebutuhan pengguna dengan lebih baik, misalnya dengan menggunakan AI Agents untuk berinteraksi dengan pengguna dan memahami kebutuhan mereka secara lebih mendalam lalu melakukan pencarian berbasis semantik pada sistem menggunakan Model Context Protocol.
    \item Sistem ini dapat diperluas untuk memfasilitasi seluruh Smart Contracts, termasuk Smart Contracts yang tidak terverifikasi. Hal ini dilakukan dengan memanfaatkan penelitian analisis Bytecode yang tersedia. Namun, hal ini akan menambahkan risiko keamanan yang lebih tinggi karena sifat dari Smart Contracts yang tidak terverifikasi. Oleh karena itu, perlu dilakukan penelitian lebih lanjut untuk mengembangkan sistem yang dapat memfasilitasi Smart Contracts yang tidak terverifikasi dengan aman.
    \item Sistem retrieval dapat ditingkatkan dengan menggunakan teknik yang lebih canggih seperti \textit{knowledge graph} untuk meningkatkan relevansi semantik dari hasil pencarian. Hal ini dapat dilakukan dengan mengembangkan sistem yang dapat mengekstraksi entitas dan relasi dari Smart Contracts, sehingga dapat membangun \textit{knowledge graph} yang dapat digunakan untuk meningkatkan relevansi hasil pencarian.
    \item Sistem retrieval dapat ditingkatkan dengan mengekstraksi nilai-nilai yang ada dari data, membangun sebuah ontologi, yang lalu akan digunakan oleh sebuah model AI yang menerima input dari pengguna lalu mengubah input tersebut menjadi sebuah query yang spesifik menggunakan ontologi yang dimiliki. Hal ini dapat meningkatkan relevansi dan menghasilkan hasil yang lebih spesifik sesuai dengan kebutuhan pengguna.
    \item Sistem retrieval dapat ditingkatkan dengan menggabungkan beberapa teknik seperti teknik \textit{reranking} hasil dengan menggabungkan pendekatan berbasis teks dan berbasis semantik. Hal ini dapat meningkatkan relevansi hasil pencarian dengan menggabungkan kekuatan dari kedua pendekatan tersebut.
    \item Sistem retrieval dapat ditingkatkan dengan menambahkan mekanisme chunking untuk konversi embeddings pada \textit{source code} Smart Contracts. Hal ini dapat dimanfaatkan untuk melakukan pencarian semantik pada embeddings \textit{source code}.
\end{enumerate}

