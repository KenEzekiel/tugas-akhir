\chapter{Pendahuluan}

% Bab Pendahuluan secara umum yang dijadikan landasan kerja dan arah kerja penulis tugas akhir, berfungsi mengantar pembaca untuk membaca laporan tugas akhir secara keseluruhan.

Bab ini berisikan gambaran umum dan masalah yang akan diselesaikan dalam tugas akhir. Bab ini dimulai dari penjelasan latar belakang, yaitu kondisi aktual dari permasalahan yang akan diselesaikan, perumusan masalah yang didapatkan, tujuan yang ingin dicapai oleh tugas akhir dengan batasan masalah yang diangkat, dan juga metodologi serta jadwal pelaksanaan yang akan digunakan pada tugas akhir ini. 

\section{Latar Belakang}
\label{sec:latarbelakang}

% Latar Belakang berisi dasar pemikiran, kebutuhan atau alasan yang menjadi ide dari topik tugas akhir. Tujuan utamanya adalah untuk memberikan informasi secukupnya kepada pembaca agar memahami topik yang akan dibahas.  Saat menuliskan bagian ini, posisikan anda sebagai pembaca – apakah anda tertarik untuk terus membaca?
% ide dan kondisi aktual dari landscapenya sekarang
Konsep dasar blockchain berakar pada karya \cite{haber1991time} yang memperkenalkan sebuah metode untuk mencatat dokumen dengan cap waktu yang tidak dapat diubah. Konsep dasar tersebut kemudian dikembangkan oleh \cite{nakamoto2008bitcoin}, dalam karyanya, Bitcoin Whitepaper, sebuah dokumen yang memperkenalkan konsep Bitcoin kepada dunia yang menggabungkan konsep \textit{digital timestamping} yang diperkenalkan oleh \cite{haber1991time} dengan konsep Kriptografi \parencite{hellman1976new} \parencite{standard1995secure}, Merkle Tree \parencite{merkle1987digital}, Konsensus Proof-of-Work \parencite{dwork1992pricing}, dan Smart Contracts \parencite{szabo1997formalizing}. Setelah perilisan Bitcoin pada tahun 2009, terjadi peningkatan eksponensial dalam adopsi dan penelitian terkait blockchain dan cryptocurrency. Secara pasar finansial, dunia blockchain dan cryptocurrency sudah mencapai USD 26,91 miliar pada tahun 2024 \parencite{rosencrance2024top}, dan diprediksi untuk bertumbuh sampai USD 825,93 miliar pada tahun 2032. Pertumbuhan tersebut menandakan investasi yang tinggi pada teknologi blockchain, diiringi dengan perkembangan teknologi dan juga peluang yang bertumbuh pada teknologi blockchain.

% Dengan diperkenalkannya Ethereum oleh \cite{buterin2013ethereum}, teknologi blockchain memperbolehkan Smart Contract yang pertama diusulkan oleh \cite{szabo1997formalizing} untuk direalisasikan. Realisasi dari Smart Contract membuka jalan yang luas dengan mereduksi resiko, biaya administrasi, meningkatkan efisiensi dari proses bisnis, dan mendukung banyak spektrum dari pengembangan aplikasi \parencite{zheng2020overview}. Semenjak pengenalan Smart Contract, terdapat pertumbuhan yang drastis dalam pembuatan Smart Contract, di mana Smart Contract yang dibuat hanya pada kuarter 1 tahun 2022 adalah 1,45 juta Smart Contract, yang adalah peningkatan sejumlah 24,7\% dari kuarter 4 tahun 2021, yang adalah 1,16 juta \parencite{alchemy_ethereum_statistics}. Dengan jumlah Smart Contract yang terus bertambah, bertambah juga jumlah masalah yang terlihat di dalam Smart Contract, 
% seperti kesulitan pengguna untuk menemukan Smart Contract yang sesuai dengan kebutuhan,
% ketergantungan pada kata kunci yang spesifik atau tidak relevan dalam pencarian Smart Contract,
% kesulitan integrasi karena perancangan Smart Contract yang tidak \textit{interoperable},
% banyaknya Smart Contract dengan fungsionalitas yang serupa dengan implementasi berbeda yang menimbulkan kebingungan pada pengguna untuk memilih,
% sulitnya mengevaluasi keamanan dan kepercayaan/kredibilitas/kualitas pada Smart Contract,
% sulitnya menemukan versi yang lebih baru dari sebuah Smart Contract, 

Dengan diperkenalkannya Ethereum oleh \cite{buterin2013ethereum}, teknologi blockchain memungkinkan realisasi Smart Contracts yang pertama kali diusulkan oleh \cite{szabo1997formalizing}. Realisasi Smart Contracts ini membuka peluang besar dengan mereduksi risiko, menurunkan biaya administrasi, meningkatkan efisiensi proses bisnis, serta mendukung pengembangan aplikasi dalam berbagai spektrum \parencite{zheng2020overview}.

Smart Contracts bukan hanya dimanfaatkan untuk melakukan transaksi di dalam blockchain dengan lebih efisien, terpercaya, dan \textit{trustless}, tetapi juga menjadi dasar bagi pengembangan dApps (Decentralized Applications), sebuah aplikasi terdesentralisasi yang memanfaatkan Smart Contracts untuk menjalankan logika bisnis dan pemrosesannya. Smart Contracts menjadi konsep yang penting di dalam blockchain. Sejak pengenalannya, terjadi pertumbuhan drastis dalam jumlah kontrak yang dibuat. Pada kuartal 1 tahun 2022 saja, tercatat 1,45 juta Smart Contracts baru yang di-\textit{deploy}, meningkat sebesar 24,7\% dari kuartal 4 tahun 2021 yang mencatat 1,16 juta Smart Contracts \parencite{alchemy_ethereum_statistics}.

Namun, pertumbuhan ini juga menghadirkan berbagai tantangan, seperti:
\begin{itemize}
	%  \item Kesulitan bagi pengguna, baik pengguna kontrak maupun \textit{developer} dalam menemukan Smart Contracts.
	\item \textbf{Kesulitan dalam Menemukan Smart Contract yang Sesuai} \newline
	      Terdapat banyak Smart Contract publik, sehingga sulit bagi pengguna atau pengembang menemukan dan menggunakan Smart Contract yang sesuai dengan kegunaan yang diinginkan.
	\item \textbf{Hasil Pencarian yang Terlalu Spesifik atau Tidak Relevan} \newline
	      Metode pencarian berbasis kata kunci sering menghasilkan hasil yang tidak tepat, baik terlalu sempit maupun tidak relevan, sehingga menghambat proses pencarian Smart Contract yang sesuai.
	\item \textbf{Latensi Pencarian yang Meningkat} \newline
	      Pertumbuhan data di blockchain menyebabkan latensi yang semakin tinggi, sehingga proses pencarian Smart Contract menjadi lebih lambat.
	\item \textbf{Keterbatasan Informasi Karena Tidak Selalu Ada \textit{source code}} \newline
	      Tidak semua Smart Contract yang telah dideploy disertai dengan source code (beberapa hanya tersedia dalam bentuk EVM Bytecode), sehingga sulit untuk mendapatkan informasi semantik yang mendalam baik bagi pengguna maupun sistem.
	\item \textbf{Kurangnya Dokumentasi dan Informasi yang Jelas} \newline
	      Banyak Smart Contract yang kekurangan dokumentasi dan informasi penjelasan, sehingga menyulitkan pemahaman dan evaluasi Smart Contract tersebut.
	\item \textbf{Kurangnya Standarisasi dan Pengartian Semantik} \newline
	      Tidak adanya standarisasi serta definisi semantik yang jelas mengenai aspek dan fungsionalitas Smart Contract menyulitkan adopsi dan pemanfaatan yang optimal.
	\item \textbf{Redundansi Penulisan Smart Contract} \newline
	      Karena Smart Contract yang serupa sulit ditemukan, pengembang cenderung menulis ulang kontrak yang sebenarnya sudah ada, sehingga terjadi redundansi dalam blockchain.
	\item \textbf{Kurangnya Standar Keamanan dan Mekanisme \textit{Reuse}} \newline
	      Tidak terdapat mekanisme standar yang memudahkan penggunaan ulang Smart Contract, sehingga standar keamanan dalam penulisan Smart Contract pun belum optimal.
	\item \textbf{Hambatan Integrasi dan Kurangnya Interoperabilitas} \newline
	      Desain Smart Contract yang tidak mendukung interoperabilitas menghambat integrasi dan kolaborasi antar sistem atau Smart Contract lain.
	\item \textbf{Kesulitan Menilai Keamanan, Kredibilitas, dan Kualitas} \newline
	      Tidak adanya mekanisme evaluasi yang standar membuat penilaian terhadap keamanan, kredibilitas, dan kualitas Smart Contract menjadi sulit.
	\item \textbf{Sulit Mengidentifikasi Versi Terbaru dan Efisien} \newline
	      Pengguna sering mengalami kesulitan dalam menemukan versi Smart Contract yang terbaru dan lebih efisien, sehingga kontrak yang sudah usang atau tidak optimal tetap digunakan.
\end{itemize}

Tantangan-tantangan ini jika dibiarkan dengan laju pertumbuhan Smart Contracts yang terus naik, akan membuat ekosistem pengembangan dan juga penggunaan Smart Contracts tidak efisien. Hal ini juga didukung dengan data yang didapatkan oleh \cite{aimar2023extraction}, yang mendapatkan bahwa hanya 5,78\% dari Smart Contracts yang pernah setidaknya menerima sebuah transaksi dan mengemisi sebuah log. Di tengah banyaknya Smart Contracts ini, kesulitan pencarian akan membuat lebih banyak pengguna membuat kontrak baru yang lebih tidak terstandarisasi dan \textit{error-prone} dibandingkan menggunakan yang sudah ada, walaupun Smart Contracts yang sudah ada memiliki fungsionalitas yang sama persis dengan apa yang dibutuhkan pengguna dan lebih terjamin secara \textit{standard}. Sehingga menyebabkan inefisiensi secara usaha, waktu, dan juga besar data yang disimpan di dalam blockchain. Hal ini juga berarti akan sulit mengembangkan sistem yang lebih kompleks menggunakan Smart Contracts.

Terdapat beberapa penelitian yang dapat membantu menyelesaikan tantangan-tantangan tersebut, seperti yang dilakukan oleh \cite{third2017linked}, di mana dilakukan \textit{indexing} terhadap data di dalam blockchain, termasuk juga data Smart Contract, dan melakukan \textit{mapping} kedua data tersebut ke sebuah Ontology untuk memudahkan pencarian. Penelitian lainnya yang dilakukan oleh \cite{aimar2023extraction}, membuat sebuah perangkat lunak, eth2dgraph, yang dapat melakukan \textit{mapping} antara data di dalam Blockchain Ethereum menjadi struktur data Dgraph, dan juga beberapa penelitian lainnya seperti yang dilakukan oleh \cite{baqa2019semantic} dan \cite{cano2021toward} juga mencoba menyelesaikan tantangan-tantangan tersebut. Penelitian-penelitian ini, walau inovatif, belum menyelesaikan tantangan yang ada secara nyata, karena kurangnya implementasi pada sistem blockchain dan koneksi dengan penggunanya.

Sejak tahun 2017, yaitu setelah diperkenalkannya arsitektur transformer oleh Google yang menjadi tulang punggung dari \textit{Large Language Model} modern, terdapat perkembangan pesat dalam dunia \textit{Artificial Intelligence}, terutama dalam \textit{Natural Language Processing}. Perkembangan yang pesat ini membawa banyak inovasi baru yang dapat diintegrasikan dengan berbagai teknologi lainnya untuk menghasilkan solusi yang inovatif.

Pendekatan yang belum tereksplorasi untuk menjawab tantangan-tantangan pada ekosistem blockchain adalah menggunakan pendekatan \textit{semantic understanding} pada data dari Smart Contracts, seperti yang digunakan oleh penelitian seperti yang dilakukan oleh \cite{third2017linked}, \cite{shi2021semantic}, \cite{stan}, dan \cite{sopek2018graphchain}, tetapi dengan penggunaan bantuan \textit{Large Language Model} (LLM) dan model AI lainnya untuk melakukan \textit{semantic enrichment}, \textit{semantic understanding}, dan \textit{data retrieval}. Secara teori, dengan memanfaatkan LLM dan embeddings, sistem dapat memahami semantik dari kebutuhan pengguna dan isi serta konteks semantik dari Smart Contracts, sehingga memungkinkan pemetaan yang tepat secara semantik antara permintaan dan Smart Contracts yang tersedia.

% Penelitian-penelitian dan kapabilitas dari \textit{Artificial Intelligence} sekarang menjadi fondasi untuk solusi yang ditawarkan pada tugas akhir ini untuk menjawab tantangan-tantangan dalam ekosistem Blockhain, yaitu sebuah sistem Smart Contract Discovery yang dapat memudahkan pencarian Smart Contracts berdasarkan semantik untuk membantu pengguna untuk menggunakan dan mengembangkan Smart Contract di dalam ekosistem blockchain dengan memanfaatkan LLM dan RAG.

Pada tugas akhir ini, akan dieksplorasi dan dibangun sebuah prototipe sistem dengan pendekatan yang menggunakan LLM untuk melakukan \textit{semantic understanding}, terutama pada konteks Smart Contracts yang akan membantu \textit{data retrieval} berbasis semantik.

% Meskipun Smart Contracts memiliki potensi yang sangat besar untuk mengubah cara kerja proses bisnis konvensional dan mengefisiensikan sistem, masih terdapat banyak tantangan yang perlu dijawab, seperti masalah terkait \textit{privacy}, \textit{security}, \textit{interoperability}, dan lainnya.

% kayanya perlu ditambahin terkait Smart Contract itu banyak banget, dan susah discover yang sesuai, dan bisa interoperable, intinya introduce kebutuhan discoverability disini. kaya sesuai fungsionalitas. dan gimana inituh bisa sangat membantu, misal dengan Smart Contract yang discoverable, bakal lebih efisien atau interoperable, introduce juga usecase yang developernya disini

% Tugas Akhir ini berfokus untuk menjawab tantangan di dalam Smart Contract Discovery, yang akan berefek pada berbagai faktor seperti interoperabilitas, integritas, aksesibilitas, dan \textit{compliance}. Tujuan dari sistem tersebut adalah untuk memudahkan pencarian Smart Contracts berdasarkan semantik, sehingga hasil dari pencarian relevan dengan kebutuhan. 

% Sebagai hasil dari Tugas Akhir ini, akan dibangun sebuah sistem yang memanfaatkan indeks \textit{linked-data} di dalam blockchain, dan menggunakan sebuah ekstensi dari Semantic Smart Contract Language untuk mempermudah pencarian Smart Contract yang sesuai dengan fungsionalitas tertentu. Perangkat lunak ini diharapkan dapat memperbolehkan pencarian Smart Contract berdasarkan semantik, sebagai langkah pertama untuk mencapai interoperabilitas dan integrasi dari sistem pengembangan Smart Contract.

% ada indexingnya, gmn cara nyarinya itu query 

% nah yang gua mau bikin itu discoverynya, gimana dapetin sc yang tepat untuk kebutuhan tertentu

% dengan pembuatan sc discovery juga bisa dimanfaatkan untuk bikin package manager buat Smart Contract itu, tapi itu udah different topic, tapi harus dipikirin format yang jadi si sistem inituh bisa dipakai oleh banyak aplikasi dengan mudah, either dia json or something

% (abis ini ngomongin, banyak juga riset paper yang rilis, sekitar X jumlahnya, yang menandakan bahwa teknologinya terus berkembang) Dengan diperkenalkannya Ethereum oleh  (nah ethereum memperkenalkan Smart Contract, dan disitu jadi muncul banyak isu buat optimasi seperti skalabilitas, security, privacy, dan salah satunya adalah interoperabilitas, nah si discovery ini bukan cuma interoperabilitas, tetapi juga (nah pikirin benefit apa yang muncul dengan si Smart Contract discovery inituh??))

% setelah ini ngomongin terkait tren nya sekarang dan kenapa makin lama makin shift ke Smart Contract, kaya kenapa Smart Contract dipakai dan kenapa butuh, terus jadi ke gimana Smart Contract discovery itu akan dibutuhkan
\pagebreak
\section{Rumusan Masalah}
\label{sec:rumusan-masalah}

Berdasarkan kondisi saat ini dimana jumlah Smart Contracts terus bertambah dengan laju yang pesat tanpa infrastruktur yang baik dan mewadahi, berikut adalah rumusan masalah yang akan dijawab dalam penelitian ini:
\begin{enumerate}
	% \item Dengan banyaknya Smart Contracts yang ada, bagaimana menemukan Smart Contracts dengan fungsionalitas yang sesuai dengan kebutuhan dan spesifikasi yang diberikan?
	% \item Pencarian Smart Contracts masih menggunakan kata kunci, yang biasanya tidak relevan dengan arti atau fungsionalitas dari Smart Contract tersebut, sehingga bagaimana menemukan Smart Contracts dengan semantik yang sesuai?
	% \item Tidak semua Smart Contract dapat terjamin keamanannya, sehingga bagaimana mendapatkan Smart Contracts yang memiliki \textit{compliance} terhadap aturan yang baik?
	% \item Apakah ada cara yang lebih baik untuk memodelkan Smart Contracts dibandingkan model \textit{Minimal Service Model}?
	% \item Bagaimana mengevaluasi ketersesuaian hasil Smart Contracts Discovery berdasarkan semantik?

	% \item Bagaimana membangun arsitektur dan mengimplementasikan sistem Smart Contract Discovery berbasis semantik yang mampu menemukan Smart Contracts sesuai dengan kebutuhan pengguna secara lebih akurat dibandingkan metode berbasis kata kunci?

	% \item Bagaimana membangun sebuah sistem Smart Contract Discovery yang dapat memberikan hasil pencarian berupa Smart Contract yang sesuai bagi pengembang maupun pengguna Smart Contracts jika hanya diketahui kebutuhan fungsionalitas yang diinginkan?

	% \item Bagaimana membantu pengguna atau pengembang menemukan Smart Contracts yang sesuai dengan kebutuhan di tengah banyaknya Smart Contracts yang ada pada Blockchain? (RM1)
	\item Bagaimana melakukan \textit{enrichment} dengan LLM yang memberikan pengertian semantik dengan terhadap data Smart Contracts untuk membantu pengguna menemukan Smart Contracts yang sesuai dengan kebutuhan? (RM1)
	      % SEMANTIC ENRICHMENT
	      % dengan sistem yang akan dibangun
	      % menggunakan semantic search & functionaliy-based search, dan memanfaatkan RAG untuk memahami kebutuhan pengguna dan mengambil data
	      % menggunakan LLM 

	\item Bagaimana melakukan \textit{data retrieval} berbasis semantik terhadap Smart Contracts yang sudah \textit{semantically enriched}? (RM2)
	      % Data retrieval menggunakan cosine similarity

	\item Bagaimana merancang dan mengimplementasikan sistem Smart Contract Discovery yang mampu menerima input berupa kebutuhan fungsionalitas dari pengguna atau pengembang, menghasilkan daftar Smart Contract yang relevan melalui mekanisme pencarian yang efektif dan akurat, serta dapat diakses secara luas tanpa mengganggu kinerja jaringan blockchain yang digunakan? (RM3)
	      % DESIGN AND DEVELOPMENT
	      % mekanisme pencarian dan relevansi hasil.
	      % Langchain (difasilitasi LLM) -> mengerti kebutuhan fungsionalitas yang pengguna inginkan dan dapat memberikan hasil yang sesuai
	      % LLM untuk semantic enrichment
	      % RAG untuk retrieval

	      % sistem dapat diskalakan dan tidak mengganggu performa blockchain.
	      % Offchain dan dgraph

	      % \item Bagaimana mengekstrak informasi dari Smart Contract yang ada dan membangun model representasi yang efektif untuk mendukung pencarian dan klasifikasi? (RM4)
	      % (melalui metadata, source code, dan dokumentasi)
	      % ekstraksi data dan pembentukan model representasi dari smart contract.
	      % eth2dgraph mengektstrak informasi, lalu melakukan semantic enrichment, lalu melakukan query (fokus di semantic enrichment nya)

	      % \item Bagaimana mengembangkan modul yang dapat memahami semantik dari kebutuhan pengguna dan isi Smart Contract, sehingga memungkinkan pemetaan yang tepat antara permintaan dan Smart Contract yang tersedia? (RM5)
	      % mengintegrasikan pemahaman semantik untuk mencocokkan kebutuhan dengan konten smart contract.
	      % Ini fokus di GIMANA MEKANISME semantic enrichment nya
\end{enumerate}

% Rumusan Masalah berisi masalah utama yang dibahas dalam tugas akhir. Rumusan masalah yang baik memiliki struktur sebagai berikut:

% \begin{enumerate}
% 	\item Penjelasan ringkas tentang kondisi/situasi yang ada sekarang terkait dengan topik utama yang dibahas Tugas Akhir.
% 	\item Pokok persoalan dari kondisi/situasi yang ada, dapat dilihat dari kelemahan atau kekurangannya. \textbf{Bagian ini merupakan inti dari rumusan masalah}.
% 	\item Elaborasi lebih lanjut yang menekankan pentingnya untuk menyelesaikan pokok persoalan tersebut.
% 	\item Usulan singkat terkait dengan solusi yang ditawarkan untuk menyelesaikan persoalan.
% \end{enumerate}

% Penting untuk diperhatikan bahwa persoalan yang dideskripsikan pada subbab ini akan dipertanggungjawabkan di bab Evaluasi apakah terselesaikan atau tidak.

\section{Tujuan dan Ukuran Keberhasilan Pencapaian}
\label{sec:tujuan-ukuran-keberhasilan-pencapaian}

% Tuliskan tujuan utama dan/atau tujuan detail yang akan dicapai dalam pelaksanaan tugas akhir. Fokuskan pada hasil akhir yang ingin diperoleh setelah tugas akhir diselesaikan, terkait dengan penyelesaian persoalan pada rumusan masalah. Penting untuk diperhatikan bahwa tujuan yang dideskripsikan pada subbab ini akan dipertanggungjawabkan di akhir pelaksanaan tugas akhir apakah tercapai atau tidak. Tuliskan juga ukuran keberhasilan pencapaiannya.

Tujuan yang akan dicapai untuk tugas akhir ini adalah menghasilkan: 

\begin{enumerate}
  \item \textbf{Format Semantik yang Dapat Mengakomodasi Pencarian Semantik yang Spesifik (T1)} \newline
  Sebuah format yang dapat digunakan untuk menyimpan data deskripsi semantik dari Smart Contract yang dapat digunakan untuk melakukan pencarian semantik yang spesifik. 

  \item \textbf{Teknik \textit{Semantic Enrichment} Menggunakan LLM yang Efektif (T2)} \newline
  Implementasi teknik \textit{semantic enrichment} menggunakan \textit{Large Language Models} (LLM) untuk memperkaya deskripsi dan metadata Smart Contract, sehingga meningkatkan akurasi dan relevansi hasil pencarian.

  \item \textbf{Teknik \textit{Data Retrieval} Menggunakan RAG yang Efektif (T3)} \newline
  Penerapan teknik \textit{Retrieval-Augmented Generation} (RAG) yang efektif untuk mengambil informasi yang relevan dari basis data Smart Contract yang telah diekstraksi dan diperkaya, untuk menghasilkan daftar Smart Contracts yang sesuai dengan kebutuhan pengguna.

  \item \textbf{Mekanisme Ekstraksi Data dari Blockchain yang Terintegrasi (T4)} \newline
  Sebuah mekanisme yang terintegrasi dengan sistem yang dapat mengekstrak data dari Blockchain dan mengubahnya menjadi format yang dapat digunakan oleh sistem.

  \item \textbf{Sistem Smart Contract Discovery (T5)} \newline
  Sebuah sistem yang dapat digunakan oleh pengguna, yang dapat diakses dengan beberapa cara, untuk mencari sebuah Smart Contract yang spesifik dan sesuai dengan kebutuhan.
  
\end{enumerate}

% dapat menghubungkan kebutuhan dari pengguna kepada fungsionalitas dari Smart Contract yang sesuai menggunakan semantik dan memberikan hasil yang sesuai kepada pengguna. Sistem ini diharapkan dapat memudahkan pencarian Smart Contracts dengan mengutamakan keterkaitan semantik, serta dapat digunakan oleh pengguna tanpa mempengaruhi kinerja dari Blockchain yang digunakan.

Ukuran keberhasilan pencapaiannya adalah seberapa sesuai Smart Contract dapat ditemukan. Metrik dari masing-masing ukuran keberhasilan adalah sebagai berikut:

\begin{enumerate}
  \item \textbf{Relevansi Hasil Pencarian Smart Contract (UK1)} \newline 
  Sistem Smart Contract Discovery dapat menghasilkan Smart Contract yang relevan sesuai semantik dan fungsionalitas yang diinginkan oleh pengguna. Metrik yang digunakan adalah akurasi. 
  % Pengukuran:
  % Gunakan model pre-trained (misalnya, CodeBERT atau BERT) untuk menghasilkan representasi vektor dari query dan smart contract yang ditemukan. (BISA PAKAI LLM)
  % Hitung nilai cosine similarity antara vektor query dan vektor smart contract untuk memperoleh Semantic Similarity Score.
  % Jika dataset berlabel tidak tersedia, lakukan evaluasi manual dengan mengambil sampel hasil pencarian dan mengumpulkan umpan balik dari pengguna atau domain expert.
  % (Alternatif) Gunakan metrik ekstrinsik seperti Click-Through Rate (CTR) jika sistem sudah di-deploy.

  \item \textbf{Kualitas Semantik Data Smart Contract (UK2)} \newline
  Format semantik yang digunakan untuk menyimpan data dapat mengakomodasi pencarian Smart Contract berdasarkan semantik secara spesifik. Selain itu, data yang dihasilkan dari \textit{semantic enrichment} dapat mendeskripsikan Smart Contract dengan baik. Metrik yang digunakan adalah \textit{Semantic Expresiveness Score}.

  \item \textbf{Kemiripan Hasil Pencarian (UK3)} \newline
  Sistem Smart Contract Discovery dapat menghasilkan hasil pencarian yang memiliki kemiripan antar satu sama lain dengan query yang sama. Metrik yang digunakan adalah \textit{Semantic Similarity Score}.

  \item \textbf{Konsistensi Hasil Pencarian (UK4)} \newline
  Sistem Smart Contract Discovery dapat menghasilkan hasil pencarian yang konsisten dengan query yang serupa. Metrik yang digunakan adalah \textit{Jaccard Index} pada K-buah hasil teratas.
  
  % \item \textbf{Dampak Terhadap Kinerja Blockchain (UK2)} \newline 
  % Sistem Smart Contract Discovery dapat berjalan tanpa mempengaruhi kinerja dari Blockchain yang digunakan. Metrik yang digunakan adalah \textit{overhead} penggunaan sumber daya pada Blockchain.
  % Ini sudah pasti tercapai, karena didesain agar seperti itu, validasi nya gimana?
  %   Pengukuran:
  % Lakukan benchmarking pada jaringan blockchain dalam kondisi normal (tanpa integrasi sistem discovery) dan setelah integrasi.
  % Bandingkan metrik seperti waktu konfirmasi transaksi, beban jaringan, atau total penggunaan gas untuk mengidentifikasi apakah sistem discovery memberikan dampak negatif terhadap kinerja blockchain.
  % Pengukuran ini dapat dilakukan di lingkungan testnet dengan simulasi beban yang representatif.
  % \item \textbf{Kinerja dan Skalabilitas Sistem (UK3)} \newline 
  % Sistem Smart Contract Discovery dapat berjalan dengan latensi rendah dan skalabilitas yang baik. Metrik yang digunakan adalah \textit{Response Time}, \textit{Throughput}, dan \textit{Resource Utilization}.
  %   Pengukuran:
  % Gunakan tools performance testing (misalnya, Apache JMeter atau Gatling) untuk mensimulasikan beban pengguna dan mengukur response time sistem.
  % Ukur throughput dengan mencatat jumlah permintaan pencarian yang diproses dalam jangka waktu tertentu.
  % Pantau resource utilization menggunakan monitoring tools standar seperti Prometheus dan Grafana untuk memastikan sistem dapat menangani beban secara efisien, terutama pada kondisi puncak.
\end{enumerate}

\section{Batasan Masalah}
\label{sec:batasan-masalah}

% Tuliskan batasan-batasan yang diambil dalam pelaksanaan tugas akhir. Batasan ini dapat dihindari (tidak perlu ada) jika topik/judul tugas akhir dibuat cukup spesifik.

Batasan masalah yang diambil adalah sebagai berikut:

\begin{enumerate}
	\item \textbf{Pencarian Hanya pada Smart Contract dengan \textit{Source Code} yang Tersedia dan Terverifikasi (BM1)} \newline
	      Alasan dari batasan ini adalah untuk berfokus kepada \textit{proof-of-concept} dari sistem Smart Contract Discovery, bukan kepada aspek teknis dari \textit{decompilation} Bytecode. Selain itu, Smart Contracts yang tidak terverifikasi tidak dianjurkan untuk digunakan karena risiko keamanan yang tinggi.
	\item \textbf{Analisis dan Implementasi Difokuskan pada Blockchain Ethereum Mainnet (BM2)} \newline
	      Alasan dari batasan ini adalah karena Ethereum mainnet merupakan ekosistem yang paling banyak digunakan dan menyediakan data yang cukup. Pembatasan ini bertujuan untuk menyederhanakan analisis dan pengembangan.
	% \item \textbf{Pemodelan Fungsionalitas Smart Contract Hanya pada Fungsionalitas Umum (BM3)} \newline
	%       Alasan dari batasan ini adalah untuk mengembangkan model pencarian yang lebih efisien dan relevan untuk berbagai jenis Smart Contract. Hal ini dilakukan untuk menghindari kerumitan yang timbul dari variasi fitur yang sangat spesifik dan memudahkan standarisasi dalam model pencarian.
	      % \item Sistem Smart Contract Discovery dirancang untuk menghasilkan sejumlah hasil yang relevan tanpa perlu simulasi besar atau pengujian dengan berbagai variasi fungsionalitas. 
	\item \textbf{Fokus Tugas Akhir berada pada Pengembangan Sistem Pencarian (BM4)} \newline
	      Untuk memastikan lingkup yang jelas, maka fokus dari Tugas Akhir ini adalah pada mekanisme pencarian data Smart Contracts, bukan pada pengembangan antarmuka pengguna, pengujian keamanan Smart Contracts, atau aspek lain yang tidak terkait langsung dengan pencarian. Mekanisme pencarian yang dimaksud adalah bagaimana sistem dapat menemukan Smart Contracts yang relevan dengan kebutuhan.
	      % \item \textbf{Pengujian Kinerja Sistem Dilakukan dalam Lingkungan Terkontrol} \newline
	      % Alasan dari batasan ini adalah untuk menghindari dampak variabilitas eksternal dan memastikan data yang diperoleh valid dan akurat.
\end{enumerate}

\section{Metodologi}
\label{sec:metodologi}

% Tuliskan semua tahapan yang akan dilalui selama pelaksanaan tugas akhir. Tahapan ini spesifik untuk menyelesaikan persoalan tugas akhir. Tahapan studi literatur tidak perlu dituliskan karena ini adalah pekerjaan yang harus Anda lakukan selama proses pelaksanaan tugas akhir.

Terdapat beberapa tahapan untuk melaksanakan tugas akhir secara sistematis, berikut merupakan tahapan-tahapan yang dilakukan:

\begin{enumerate}
	\item \textbf{Identifikasi Permasalahan} \newline
	      Pada tahapan ini, dilakukan eksplorasi dan pengumpulan informasi terkait sistem pencarian Smart Contract untuk mengidentifikasi permasalahan utama yang akan diselesaikan.
	\item \textbf{Identifikasi Kebutuhan} \newline
	      Permasalahan utama akan dielaborasikan menjadi kebutuhan-kebutuhan yang harus dipenuhi oleh solusi yang dirancang.
	\item \textbf{Analisis dan Desain Sistem} \newline
	      Analisis kebutuhan untuk membuat rancangan dari sistem.
	\item \textbf{Pengembangan prototipe} \newline
	      Hasil dari rancangan akan diimplementasikan menggunakan teknologi yang sesuai.
	\item \textbf{Pengujian Prototipe} \newline
	      Hasil implementasi, yaitu prototipe, akan diuji dengan berbagai kasus nyata yang sesuai dengan kebutuhan yang akan dipenuhi untuk menjamin ketercapaian kebutuhan.
	\item \textbf{Evaluasi dan Analisis Kinerja Prototipe} \newline
	      Hasil dari pengujian akan dianalisis dan dievaluasi, untuk memberikan saran perbaikan dan juga kesimpulan dari penelitian.
\end{enumerate}

% \section{Sistematika Pembahasan}
% \label{sec:sistematika-pembahasan}

% \section{Jadwal Pelaksanaan Tugas Akhir}

% Tuliskan rencana kegiatan dan jadwal (dirinci sampai per minggu) mulai dari awal pelaksanaan Tugas Akhir I s.d. sidang tugas akhir berikut milestones dan deliverables yang harus diberikan. Jadwal ini dapat dibantu dengan membuat sebuah tabel timeline.

\section{Sistematika Pembahasan}

Pembahasan pada laporan ini akan dibagi menjadi lima bagian yang terdiri dari:

\begin{enumerate}
  \item \textbf{Pendahuluan} \\
  Bab I akan menjelaskan gagasan utama dari topik tugas akhir ini yang terdiri dari latar belakang dan tantangan yang ada pada saat ini, perumusan masalah dari tantangan-tantangan yang ada, penentuan tujuan, batasan permasalahan, metodologi pengembangan, dan sistematika pembahasan mengenai proses pengembangan solusi.
  \item \textbf{Studi Literatur} \\
  Bab II akan menjelaskan terkait studi literatur yang menjadi landasan untuk pengetahuan dan pengembangan dari solusi.
  \item \textbf{Analisis Persoalan dan Rancangan Solusi} \\
  Bab III akan menjelaskan hasil analisis dari masalah yang ditemukan pada kondisi yang ada pada saat ini diikuti dengan dampaknya, kebutuhan yang muncul dari analisis tersebut, dan ajuan solusi yang diusulkan untuk menyelesaikan masalah yang ada. Selain itu, bab ini juga akan menjelaskan alternatif solusi yang dipertimbangkan dan alasan mengapa solusi yang diusulkan dipilih.
  \item \textbf{Implementasi dan Pengujian} \\
  Bab IV akan menjelaskan terkait hasil implementasi dan rancangan yang telah dibuat, baik dari segi arsitektur dan teknologi yang digunakan, hingga hasil pengujian yang dilakukan untuk memastikan kualitas dari solusi yang diusulkan.
  \item \textbf{Kesimpulan dan Saran} \\
  Bab V menjadi penutup dari laporan ini, yang akan menjelaskan kesimpulan dari hasil pengembangan solusi yang telah dilakukan, serta saran dari penulis untuk pengembangan selanjutnya.
\end{enumerate}
