\chapter*{Kata Pengantar}
\addcontentsline{toc}{chapter}{KATA PENGANTAR}

Puji dan syukur penulis panjatkan kepada Tuhan Yang Maha Esa 

% \begin{enumerate}
% 	\item 
% \end{enumerate}

\begin{flushright}
	\vspace{0.5cm}
	Bandung, \tanggalpengesahan

	\vspace{1.5cm}

	Kenneth Ezekiel Suprantoni
\end{flushright}

% Puji dan syukur penulis panjatkan kepada Tuhan Yang Maha Esa atas berkat dan rahmatnya, laporan tugas akhir yang berjudul "\thetitle" dapat diselesaikan dalam rangka memenuhi syarat kelulusan tingkat sarjana. Perlu diakui pengerjaan tugas akhir ini didukung oleh banyak pihak. Khususnya, penulis ingin mengucapkan terima kasih kepada:

% \begin{enumerate}
% 	\item Bapak Dr.techn. Muhammad Zuhri Catur Candra, S.T., M.T., selaku dosen pembimbing atas segala bentuk dukungan yang telah diberikan dan kesabarannya dalam membimbing penulis serta memberikan saran dalam pengerjaan tugas akhir.
% 	\item Bapak Yudistira Dwi Wardhana Asnar, S.T, Ph.D dan Dr. Agung Dewandaru, S.T., M.Sc., selaku dosen penguji atas segala masukan dan kritik yang telah diberikan terhadap tugas akhir penulis.
% 	\item Dicky Prima Satya, S.T, M.T., Bapak Adi Mulyanto, S.T, M.T., Robithoh Annur, S.T., M.Eng., Ph.D., dan Tricya Esterina Widagdo, ST., M.Sc. selaku dosen koordinator tim tugas akhir atas usahanya mengingatkan mahasiswa program studi Teknik Informatika untuk mengerjakan tugas akhirnya.
% 	\item Seluruh dosen program studi Teknik Informatika ITB yang telah memberikan ilmu pengetahuan yang sangat berharga bagi penulis.
% 	\item Ibu Rini Liani dan Bapak Edhie Hikmat selaku kedua orangtua penulis atas dukungan yang diberikan
% 	\item Rumah Amal Salman ITB serta tim Beasiswa Perintis yang telah membantu penulis sehingga penulis dapat menempuh pendidikan di Institut Teknologi Bandung dengan mudah.
% 	\item Teman-teman INIT 2020 yang telah menemani, memberikan inspirasi, serta dukungan moral kepada penulis dalam menempuh kuliah pada program studi Teknik Informatika.
% 	\item Teman-teman penulis khususnya anggota dari grup "temenin ngerjain TA", "Koordinasi penonton sempro", "Para Ajudan Pecinta Sedekah", "Kos Aufa Enjoyer", "kaliMANTAN", serta "karimun" yang telah memberikan kenangan berharga, motivasi, hiburan, serta bantuan untuk segala situasi.
% 	\item Sahabat Penulis khususnya Erik dan Rachel yang telah pantang menyerah berjuang bersama dalam berkompetisi CTF.
% 	\item Sahabat terdekat penulis khususnya Marcho, Aira, Gagas, Rio, Dhika, Kinan, Anca, Dipa, Ubai, Azka, Sarah, Dea, Fay, Epi, Aufa, Syahrul, Rifqi dan Malik yang telah menemani perjuangan dari TPB hingga saat ini, menjadi \textit{emotional support} di segala situasi, membantu penulis dalam proses pengejaan tugas akhir, serta membuat hari - hari menjadi lebih berwarna.
% 	\item Seluruh pihak lain yang tidak bisa disebutkan disini yang telah membantu dalam proses pengerjaan tugas akhir.
% \end{enumerate}

% Akhir kata, penulis mengucapkan terima kasih kepada semua pihak yang telah terlibat dalam pengerjaan tugas akhir ini. Penulis juga ingin menyampaikan mohon maaf apabila terdapat kesalahan maupun kekurangan dalam laporan tugas akhir ini. Penulis berharap semoga tugas akhir ini dapat bermanfaat bagi pembaca dan riset-riset kedepannya.