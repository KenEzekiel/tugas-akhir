% \chapter{Implementasi dan Pengujian}
\chapter{Rencana Pelaksanaan}

Bab ini akan menjelaskan proses implementasi dari rancangan solusi yang telah dikaji pada Bab III. Setelah pembahasan terkait implementasi, akan dilanjutkan dengan pemaparan hasil uji terkait implementasi yang telah dibuat.


% Bab Rencana Pelaksanaan digunakan untuk mendeskripsikan rencana pelaksanaan berupa jadwal dan risiko-risiko yang mungkin dihadapi dan rencana mitigasinya. Tujuan bab ini  adalah:
% 1.	Mahasiswa memiliki rencana yang jelas mengenai pelaksanaan TA
% 2.	Mahasiswa mengenali risiko-risiko yang mungkin dihadapi dan sudah menyiapkan diri untuk mengantisipasi risiko tersebut.

% \section{Lingkungan}
% \blindtext

% \section{Implementasi}
% \blindtext

% \section{Pengujian}
% \blindtext

\section{Jadwal}
\label{sec:jadwal}

% Cantumkan jadwal pengerjaan tugas akhir lengkap dengan uraiannya.

\section{Risiko}
\label{sec:risiko}

% misal kegiatan apa yang akan jadi tantangan, misal ada kepengurusan, jadi lebih baik dituliskan agar diperhatikan
% misal kurang banyak artikel yang dituliskan

% Cantumkan 5 risiko tertinggi yang mungkin dihadapi dalam pengerjaan tugas akhir. Risiko yang dicantumkan dapat merupakan risiko dari sisi teknis, risiko dari sisi operasional, risiko dari metode yang dipilih, dan sebagainya. Cantumkan pula rencana mitigasi dari risiko-risiko tersebut.